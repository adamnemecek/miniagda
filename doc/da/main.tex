\chapter{Basic Mugda}

% Inference rules
\newcommand{\rulename}[1]{\ensuremath{\mbox{\sc#1}}}
\newcommand{\ru}{\dfrac}
\newcommand{\rux}[3]{\ru{#1}{#2}\ #3}
\newcommand{\nru}[3]{#1\ \ru{#2}{#3}}
\newcommand{\nrux}[4]{#1\ \ru{#2}{#3}\ #4}
\newcommand{\dstack}[2]{\begin{array}[b]{c}#1\\#2\end{array}}
\newcommand{\xstack}[3]{\begin{array}[b]{c}#1\\#2\\#3\end{array}}
\newcommand{\ndru}[4]{#1\ \ru{\dstack{#2}{#3}}{#4}}
\newcommand{\nxru}[5]{#1\ \ru{\xstack{#2}{#3}{#4}}{#5}}
\newcommand{\ndrux}[5]{#1\ \ru{\dstack{#2}{#3}}{#4}\ #5}


% rule names
\newcommand{\infvar}{\rulename{inf-var}}
\newcommand{\infset}{\rulename{inf-set}}
\newcommand{\infsize}{\rulename{inf-size}}
\newcommand{\infinfty}{\rulename{inf-infty}}
\newcommand{\infappa}{\rulename{inf-app-i}}
\newcommand{\infappb}{\rulename{inf-app-ii}}
\newcommand{\infcon}{\rulename{inf-con}}
\newcommand{\infdef}{\rulename{inf-fun}}
\newcommand{\infconst}{\rulename{inf-const}}
\newcommand{\infdata}{\rulename{inf-data}}

\newcommand{\chklam}{\rulename{chk-lam}}
\newcommand{\chkpi}{\rulename{chk-pi}}
\newcommand{\chksucc}{\rulename{chk-succ}}
\newcommand{\chkinf}{\rulename{chk-inf}}

\newcommand{\eqnf}{\rulename{eq-nf}}
\newcommand{\eqlf}{\rulename{eq-lf}}
\newcommand{\eqrf}{\rulename{eq-rf}}

\newcommand{\eqset}{\rulename{eq-set}}
\newcommand{\eqsize}{\rulename{eq-size}}
\newcommand{\eqsuc}{\rulename{eq-succ}}
\newcommand{\eqinfty}{\rulename{eq-infty}}
\newcommand{\eqapp}{\rulename{eq-app}}
\newcommand{\eqgen}{\rulename{eq-gen}}
\newcommand{\eqpi}{\rulename{eq-pi}}
\newcommand{\eqlam}{\rulename{eq-lam}}
\newcommand{\eqdef}{\rulename{eq-fun}}
\newcommand{\eqcon}{\rulename{eq-con}}
\newcommand{\eqdata}{\rulename{eq-data}}

\newcommand{\chkpvar}{\rulename{chkp-var}}
\newcommand{\chkpcon}{\rulename{chkp-con}}
\newcommand{\chkpsucc}{\rulename{chkp-succ}}
\newcommand{\chkpinacc}{\rulename{chkp-inacc}}

\newcommand{\leqnf}{\rulename{leq-nf}}
\newcommand{\leqlf}{\rulename{leq-lf}}
\newcommand{\leqrf}{\rulename{leq-rf}}

\newcommand{\leqset}{\rulename{leq-set}}
\newcommand{\leqsize}{\rulename{leq-size}}
\newcommand{\leqsuc}{\rulename{leq-succ}}
\newcommand{\leqinfty}{\rulename{leq-infty}}
\newcommand{\leqappco}{\rulename{leq-co}}
\newcommand{\leqappind}{\rulename{leq-ind}}
\newcommand{\leqappd}{\rulename{leq-data}}
\newcommand{\leqapp}{\rulename{leq-app}}
\newcommand{\leqgen}{\rulename{leq-gen}}
\newcommand{\leqpi}{\rulename{leq-pi}}
\newcommand{\leqlam}{\rulename{leq-lam}}
\newcommand{\leqdef}{\rulename{leq-fun}}
\newcommand{\leqcon}{\rulename{leq-con}}

\newcommand{\sleqinfty}{\rulename{sleq-infty}}

\newcommand{\sleqsucci}{\rulename{sleq-succ-i}}
\newcommand{\sleqsuccii}{\rulename{sleq-succ-ii}}
\newcommand{\sleqgen}{\rulename{sleq-gen}}

\newcommand{\dd}{\mathsf{d}}
\newcommand{\DD}{\mathsf{D}}
\newcommand{\cc}{\mathsf{c}}
\newcommand{\ff}{\mathsf{f}}

\newcommand{\cll}{\mathbb{L}}
\newcommand{\cdd}{\mathbb{D}}
\newcommand{\ccc}{\mathbb{C}}
\newcommand{\cff}{\mathbb{F}}
\newcommand{\cV}{\mathbb{V}}

\newcommand{\cS}{\mathbb{S}}
\newcommand{\cN}{\mathbb{N}}
\newcommand{\cB}{\mathbb{B}}
\newcommand{\cI}{\mathbb{I}}
\newcommand{\cF}{\mathbb{F}}
\newcommand{\nF}{\mathbf{N}}
\newcommand{\lF}{\mathbf{L}}
\newcommand{\rF}{\mathbf{R}}

\newcommand{\sz}{\mathbf{sz}}
\newcommand{\ns}{\mathbf{nosz}}
\newcommand{\co}{\mathbf{co}}
\newcommand{\ind}{\mathbf{ind}}

\newcommand{\Expr}{\mathrm{Expr}}
\newcommand{\Val}{\mathrm{Val}}
\newcommand{\vlam}{\mathrm{Lam}}
\newcommand{\vpi}{\mathrm{Pi}}
\newcommand{\Clos}{\mathrm{Clos}}
\newcommand{\cl}[2]{#1^{#2}}
\newcommand{\clp}[2]{\cl{(#1)}{#2}}
\newcommand{\Env}{\mathrm{Env}}
\newcommand{\wh}{\searrow}
\newcommand{\ap}{\mathrm{app}}
\newcommand{\apf}{\mathrm{app_{fun}}}
\newcommand{\sinfty}{\mathrm{s_{\infty}}}
\newcommand{\lkup}{\mathrm{lkup}}
\newcommand{\ins}[3]{#1,#2=#3}
\newcommand{\forc}{\mathrm{force}}
\newcommand{\matchcls}{\mathrm{match_{cls}}}
\newcommand{\matchcl}{\mathrm{match_{cl}}}
\newcommand{\match}{\mathrm{match}}
\newcommand{\matchf}{\mathrm{match_f}}
\newcommand{\matchList}{\mathrm{match_{list}}}
\newcommand{\matchInfty}{\mathrm{match_{\infty}}}
\newcommand{\stuck}{\;\Uparrow}
\newcommand{\Clause}{\mathrm{Clause}}
\newcommand{\Decl}{\mathrm{Decl}}
\newcommand{\Pat}{\mathrm{Pat}}

\newcommand{\Sig}{\mathrm{\Sigma}}


In the following sections, we will describe in detail the syntax, semantics and type checking algorithms for \mugda.
The starting point for this chapter was the simple dependent type checking algorithm that is given in \cite{coquand96algorithm}.

\noindent First, we have to define some basics:
\paragraph*{Natural numbers}
$ \cN = \{ 0 , 1 , ... \} $ is the set of natural numbers.
\paragraph*{Booleans}
$ \cB = \{ \top, \bot \} $ is the set of boolean values.
\paragraph*{Vectors}
For some set $B$ we write $\ves{B}$ for the set of finite vectors over $B$.
For the empty vector we write $\diamond$.
We denote vectors with $\vec{b}$ or $(b_1, \ldots b_n)$.
The length of a vector is defined as $|(b_1,\ldots, b_n|)| := n $. 
\section{Syntax of Mugda}

\paragraph*{Identifier}
First we assume the following disjunct sets of identifiers.
\[
\begin{array}{ll}
\ccc \ni \cc & \mbox{constructors}\\
\cdd \ni \DD & \mbox{data types}\\
\cff \ni \ff & \mbox{functions}\\
\cll \ni \dd & \mbox{constants}\\
\cV \ni x,y,i & \mbox{variables}
\end{array}
\]

\paragraph*{Expressions}
\[
\begin{array}{lrll}
\Expr \ni e,A,B & ::= & \lam{x}{e} & \mbox{abstraction}\\
& \mid & \EPi{x}{A}{B} & \mbox{dep function type}\\
& \mid & e \; \vec{e} & \mbox{application}\\
& \mid & \Set & \mbox{universe of small types}\\
& \mid & \Size & \mbox{size type} \\
& \mid & \s e  & \mbox{size succesor}\\ 
& \mid & \infty & \mbox{limit size}\\
& \mid & x & \mbox{variable}\\
& \mid & \cc & \mbox{constructor}\\
& \mid & \DD & \mbox{data type}\\
& \mid & \ff & \mbox{function}\\
& \mid & \dd & \mbox{const}
\end{array}
\]

\paragraph*{Patterns}
\[
\begin{array}{lrll}
\Pat \ni p & ::= & x & \mbox{variable pattern}\\
& \mid & \cc \; \vec{p} & \mbox{constructor pattern}\\
& \mid & \s p & \mbox{size succesor pattern}\\
& \mid & \inacc{e} & \mbox{inaccessible pattern}\\
\end{array}
\]

\paragraph*{Constructor definitions}
\[
\begin{array}{lrll}
\gamma &  ::= & \cc \; : \; A & 
\end{array}
\]

\paragraph*{Clause definitions}
\[
\begin{array}{lrll}
\Clause \ni \kappa & ::= & \ff \; \vec{p} = e & 
\end{array}
\]

\paragraph*{Telescopes}
\[
\begin{array}{lrll}
\tau & ::= & \diamond & \mbox{empty telescope}\\
& \mid & ( x : t ) \; \tau & \mbox{parameter}\\
& \mid & ( + \; x : t ) \; \tau & \mbox{strictly positive paramter}
\end{array}
\]

Note that in a telescope, which is basically a sequence of type bindings, the types of later declared parameters can depent on earlier parameters. The $+$ annotation will be explained in the section on \emph{type checking}. 

\paragraph*{Declarations}
\[
\begin{array}{lrll}
\Decl \ni \delta & ::= & \data \DD \tau : A \vec{\gamma} & \mbox{inductive datatype}\\ 
& \mid & \codata \DD \tau : A \vec{\gamma} & \mbox{coinductive data type} \\
& \mid & \fun \ff : A \; \vec{\kappa} & \mbox{recursive function}\\
& \mid & \cofun \ff : A \; \vec{\kappa} & \mbox{corecursive function}\\ 
& \mid & \const \dd : A \; = e & \mbox{constant}\\
& \mid & \mutual \vec{\delta} & \mbox{mutual declaration}
\end{array}
\]



\paragraph*{Syntactic sugar}
We write $A \ra B$ for $\EPi{x}{A}{B}$ where $x$ is some dummy variable not occuring in $B$.
For a telescope $\tau = (x_1 : A_1) \ldots (x_n : A_n)$ (ignoring the $+$ annotations) we write
$\tau \ra B$ for the expression  $\EPi{x_1}{A_1}{\ldots \ra \EPi{x_n}{A_n}{B}}$.
Accordingly, we write $\Gamma \ra B$ for an expression of the form $\EPi{x_1}{A_1}{\ldots \ra \EPi{x_n}{A_n}{B}}$.

A Mugda program $\vec{\delta}$ is a list of declarations.
For $\mutual$ declarations, we allow either multiple $\fun$ or multiple $\cofun$ declarations.
Each declaration adds newly defined constructs to the signature $\Sig$, which is empty at the beginning.

\section{Semantics of Mugda expressions}
As noted in introduction, we need computation during type checking.
Thus, we need a semantic to be used during type checking.
The main ideas of the type checking algorithm of \cite{coquand96algoritm} are based on the idea of closures and generic values.
Closures, already introduced for computer science in \cite{Landin63}, are used for explicit substitution and to delay evaluation.
We simultanously define values, closures and environemnts.
 
\paragraph{Value}
\[
\begin{array}{lrll}
\Val \ni v & ::= & v \; \vec{c} & \mbox{application}\\
& \mid & \vlam \; x \; c & \mbox{abstraction}\\
& \mid & \vpi \; x \; c & \mbox{dependent function space}\\
& \mid & k & \mbox{generic value}\\
& \mid & \Set & \mbox{universe of small types} \\
& \mid & \Size & \mbox{size type value}\\
& \mid & \s v & \mbox{size succesor}\\
& \mid & \infty & \mbox{size limit}\\
& \mid & \cc & \mbox{constructor}\\
& \mid & \ff & \mbox{function symbol}\\
\end{array}
\]
A generic value $k \in \cN$ represents the computed value of a variable during type checking.
\paragraph{Closure}
\[
\begin{array}{lrll}
\Clos \ni c & ::= & \cl{e}{\rho} & 
\end{array}
\]

\paragraph{Environment}
\[
\begin{array}{lrll}
\Env \ni \rho & ::= & \diamond & \mbox{empty environment}\\
& \mid & \rho,x = v & \mbox{extended with value binding}\\
& \mid & \rho,x = c & \mbox{extended with closure binding} 
\end{array}
\]

Values can be seen as partially evaluated expressions.
They are only partially evaluated because they may contain closures.
A closure is an expression together with an environment that provides bindings for the free variables in the expression.
\paragraph*{Signature}
The signature carries information about all user-defined constants.
We define it as a overloaded function:
\begin{itemize}
\item
$ \Sig : \cff \ra \times \Val \times \ves{\Clause} \times \cB $\\
recording for a function symbol the type, the clauses and wether the clauses have been typechecked yet.
\item
$ \Sig : \cll \ra \Expr \times \Val $ \\
recording for a const definition the expression and its type. 
\item
$ \Sig : \ccc \ra \Val $
recording for a constructor its type.
\item
$ \Sig : \cdd \ra \Val \times \cN $\\
recording for a datatype its type and the number of parameters.
\end{itemize}

Now we define evaluation of a closure $\cl{e}{\rho}$ to a value. 
A closed expression can be evaluated in an empty environment.
We simultaneously define evaluation as a function $\wh$ along with other functions needed by $\wh$.     
These functions are partial because only type-correct expressions should be evaluated.
A fixed signature $\Sig$ is assumed.

\paragraph*{Evaluation}

\[
\begin{array}{ll}
\multicolumn{2}{l}{\wh \; : \Clos \ra \Val}\\
\wh \clp{\lam{x}{e}}{\rho} & = \vlam \; x \; \cl{e}{\rho} \\
\wh \clp{\EPi{x}{a}{b}}{\rho} &= \vpi \; x \; v_a \; \cl{b}{\rho} \mbox{ where } v_a = \wh \cl{a}{\rho} \\
\wh \clp{e \; e_1 \ldots e_n}{\rho} &= \ap \; v \; \cl{e_1}{\rho} \ldots \cl{e_n}{\rho} \mbox{ where } v = \wh \cl{e}{\rho} \\
\wh \cl{\Set}{\rho} & = \Set \\ 
\wh \cl{\Size}{\rho} & = \Size \\
\wh \cl{\infty}{\rho} & = \infty  \\
\wh \clp{\s e}{\rho} & = \sinfty \; v \mbox{ where } \; v = \wh \cl{e}{\rho} \\ 
\wh \cl{\cc}{\rho} & = \cc  \\
\wh \cl{\ff}{\rho} & = \ff \\
\wh \cl{\dd}{\rho} & = \wh \cl{e}{\diamond} \mbox{ where } \; \Sig \; \dd = (e,\_)\\
\wh \cl{x}{\rho} & = \lkup \; \rho \; x \\
\end{array}
\]
We note that the closure $c$  in $\vlam \; x \; \cl{e}{\rho}$
and $\vpi \; x \; v_a \; \cl{b}{\rho}$ do not have a binding for $x$.
The missing binding has to be provided before the closure can be evaluated.
This might be a concrete value (for example during $\beta$-reduction) or a
fresh generic value $k$ during type checking.

\paragraph*{Size succesor}
We should take into account that $\infty$ is the limit size, i.e the equation $\s \infty = \infty$:
\[
\begin{array}[t]{ll}
\multicolumn{2}{l}{\sinfty : \Val \ra \Val}\\ 
\sinfty \; \infty & = \infty\\
\sinfty \;  v & = \s v  \mbox{ otherwise}\\ 
\end{array}
\]

\paragraph*{Environment lookup}
\[
\begin{array}{ll}
\multicolumn{2}{l}{\lkup : \Env \ra \cV \ra \Val }\\
\lkup \; (\ins{\rho}{x}{v}) \; x & = v \\
\lkup \; (\ins{\rho}{x}{c}) \; x & = \wh c \\
\lkup \; (\ins{\rho}{y}{\_}) \; x & = \lkup \; \rho \; x \mbox{ if } y \neq x  
\end{array}
\]

\paragraph*{Application}
does $\beta$-reduction and inductive function application. 
\[
\begin{array}{ll}
\multicolumn{2}{l}{\ap : \Val \ra \ves{\Clos} \ra \Val}\\
\ap \; u \; \diamond & = u \\
\ap \; (u \; \vec{c_1}) \; \vec{c_2} & = \ap \; u \; (\vec{c_1}\;\vec{c_2})\\
\ap \; (\vlam \; x \; \cl{e}{\rho}) \; (c \; \vec{c}) & = \ap \; v \; \vec{c} \mbox{ where } v = \wh \cl{e}{\ins{\rho}{x}{c}} \\ 
\ap \; \ff \; \vec{c} & = \apf \; \ff \; \vec{c} \mbox{ if } \ff \mbox{ is a } \fun \\ 
\ap \; v \; \vec{c} & = v \; \vec{c} \mbox{ otherwise } \\
\end{array}
\]
In the following, as pattern matching can fail, we use $\stuck$ to symbolize this. 
\paragraph*{Inductive funcion application}
We always reduce application of an inductive function if possible:\[
\begin{array}{ll}
\multicolumn{2}{l}{\apf : \cV \ra \ves{\Clos} \ra \Val}\\
\apf \; \ff \; \vec{c} & = v \mbox{ if } \Sig \; \ff = (\vec{\gamma},\top) \mbox { and } \matchcls \; \vec{\gamma} \; \vec{c} = v \neq \stuck \\ 
\apf \; \ff \; \vec{c} & = \ff \; \vec{c} \mbox { otherwise } 
\end{array}
\]


\paragraph*{Corecursive unrolling}
An application of a corecursive definiton is lazily unrolled when needed: 
The following unrolls a corecursive definition once, if possible :
\[
\begin{array}{ll}
\multicolumn{2}{l}{\forc : \Val \ra \Val }\\
\forc \; \ff & = \matchcls \; \vec{\gamma} \; \diamond \mbox{ if } \Sig \; \ff = (\vec{\gamma},\top) \mbox{ and } \ff \mbox { is a } \cofun \\
\forc \; (\ff \; \vec{c}) & = \matchcls \; \vec{\gamma} \; \vec{c} \; \mbox{ if } \Sig \; \ff = (\vec{\gamma},\top) \mbox { and } \ff \mbox { is a } \cofun\\ 
\forc \; v & = \stuck \mbox{ otherwise}
\end{array}
\]


\paragraph*{Pattern matching}
Pattern matching returns an environment that binds the variables in the patterns to values.
When matching against a coinductive constructor, the value is forced.
\[
\begin{array}{ll}
\multicolumn{2}{l}{\match : \Env \ra \Pat \ra \Val \ra \Env}\\
\match \; \rho \; (\cc \; \vec{p}) \; v & = \matchf \; \rho \; p \; v' \spc \mbox{if  } \cc  \mbox{ is a coinductive constructor}\\
& \spcx \spcx \spcx \mbox{and } v' = \forc \; v \neq \stuck \\
\match \; \rho \; p \; v & = \matchf \; \rho \; p \; v \mbox{ otherwise}
\vs\\
\multicolumn{2}{l}{\matchf : \Env \ra \Pat \ra \Val \ra \Env}\\
\matchf \; \rho \; \inacc{e} \; v & = \rho \\
\matchf \; \rho \; x \; v & = \ins{\rho}{x}{v} \\
\matchf \; \rho \; (\cc \; \diamond) \; \cc & = \rho \\
\matchf \; \rho \; (\cc \; \vec{p}) (\cc \; \vec{v}) & = \matchList \; \rho \; \vec{p} \; \vec{v} \\
\matchf \; \rho \; (\s p) \; \infty & = \match \; \rho \; p \; \infty \\
\matchf \; \rho \; (\s p) \; (\s v) & = \match \; \rho \; p \; v \\
\matchf \; \_ \; \_ \; \_ & = \stuck \mbox{ otherwise}
\vs\\
\multicolumn{2}{l}{\matchList : \Env \ra \ves{\Pat} \ra \ves{\Clos} \ra \Env }\\
\matchList \; \rho \; \diamond \; \diamond & = \rho \\
\matchList \; \rho \; (p \; \vec{p}) \; (c \; \vec{c}) & = \matchList \; \rho_2 \; \vec{p} \; \vec{c} \mbox{ where } \rho_2 \; = \match \rho \; p \; v \neq \stuck \\
\matchList \; \_ \; \_ \; \_ & = \stuck \mbox { otherwise } \\
\end{array}
\]

\paragraph*{Clause matching}
For a single clause, if all patterns of a clause match against the arguments, then the right hand side can be evaluated: 
\[
\begin{array}{ll}
\multicolumn{2}{l}{\matchcl : \Env \ra \ves{\Pat} \ra \Expr \ra \ves{\Clos} \ra \Val}\\
\matchcl \; \rho \; \diamond \; e \; \vec{c} & = \ap \; v \; \vec{c} \mbox{ where } v = \wh e \; \rho \\

\matchcl \; \rho \; (p \; \vec{p}) \; e \; (c \; \vec{c}) & = \matchcl \; \rho_2 \; \vec{p} \; \vec{c} \mbox{ if } \rho_2 = \match \; p (\wh c) \neq \stuck \\  
\matchcl \; \_ \; \_\; \_ \; \_ & = \stuck \mbox{ otherwise}\\  
\end{array}
\]

\paragraph*{Clauses matching} 
Now we define how a set of clauses matches against arguments. 
Each clause is tried until one is matched or there are no clauses left. We assume that clauses do not overlap, so the order in which the clauses are tried is not relevant. as at most one will match:
Note that matching for all clauses can fail even when the coverage is complete: First, there can be too few arguments.Furthermore, generic values (that are introduced during type checking) do not match against a constructor pattern. 
\[
\begin{array}{ll}
\multicolumn{2}{l}{\matchcls : \ves{\Clause} \ra \ves{\Clos} \ra \Val}\\
\matchcls \; \diamond \; \_ & = \stuck \\
\matchcls \;  ((\vec{p} \; e)\;\vec{\gamma}) \; \vec{c} & = v \mbox{ if } v = \matchcl \; \diamond \; \vec{p} \; e \;  \vec{c} \neq \stuck \\
\matchcls \; ((\vec{p} \; e)\;\vec{\gamma}) \; \vec{c} & = \matchcls \; \vec{\gamma} \; \vec{c} \mbox{ otherwise}  
\end{array}
\]


\section{Examples}
\subsection{Identity function}
Without defining new data types, we still can define some useful non recursive functions.
One example is the polymorphic identity function. 
Mugda is monomorphic, so this is not really a polymorphic function, as you have to supply the
type as the first argument.
\begin{bsp}
$\const id : \EPi{A}{\Set}{A \ra A} = \lam{a}\lam{x}x$   
\end{bsp}

\subsection{Booleans}
The booleans are an example of a non-inductive enumeration type.
\begin{bsp}
$\data \Bool : \Set$  \\
$\spc \ttt : \Bool $\\
$\spc \fff : \Bool $
\end{bsp}

\subsection{Natural Numbers}
First we declare the type of natural numbers:
\begin{bsp}
$\data \Nat : \Set$ \\
$\spc \zero : \Nat $\\
$\spc \suc : \Nat \ra \Nat$
\end{bsp}
then we declare the addition function by pattern matching:
\begin{bsp}
$\fun \add : \Nat \ra \Nat \ra \Nat$\\
$\spc \add x \; \zero = x $\\
$\spc \add x \; (\suc y) = \suc (\add x \; y)  $
\end{bsp}
\subsection{Lists}
Lists are an example of a parameterized data type:
\begin{bsp}
$\data \List ( A : \Set ) : \Set $ \\
$ \spc \nil : \List A  $\\
$ \spc \cons : A \ra \List A \ra \List A $
\end{bsp}
\subsection{Binary Trees}
The following introduces binary trees with leaf labeled with elements of $A$ and nodes labeled with elements of $B$:
\begin{bsp}
$\data \Tree ( A : \Set ) ( B : \Set ) : \Set $ \\
$ \spc \leaf : A \ra \Tree A \; B $\\
$ \spc \node : B \ra \Tree A \; B \ra \Tree A \; B $
\end{bsp}

\subsection{Vectors}
The previous data types didn't use any dependent function space. 
Now things get more interesting. Vectors are an example of an inductive family of types.
They can be imagined as lists that store their length.
\begin{bsp}
$\data \vVec ( A : \Set ) : \Nat \ra \Set $ \\
$\spc \nil : \vVec A \; \zero  $\\
$ \spc \cons : \EPi{n}{\Nat}{ A \ra \vVec A \; n \ra \vVec A \; (\suc n)} $
\end{bsp}
Now let's define the $\head$ function that returns the first element of an vector. This operation should only be allowed for a non-empty vector.
With dependent types, we can express this in the type signature
\begin{bsp}
$ \fun \head : ( A : \Set ) \ra ( n : \Nat ) \ra \vVec A \; (\suc n) \ra A $
\end{bsp}
Now to the clause definition. We can explain the use inaccessible patterns here.
We need to pattern match to get the head element.
\begin{bsp}
$\head ? \; ? \; (\cons \; B \; m \; x \; xl) = x $
\end{bsp}
First, let's note that we don't need to pattern match against $\nil$.
Now what about the ? above. 
One might be inclined to use non-linear patterns:
\begin{bsp}
$\head \; B \; m \; (\cons  B \; m \; x \; xl) = x $
\end{bsp}
But we really don't have to check at runtime that the values at the corresponding arguments match -- it's guaranteed for a well-typed program.
So to really capture the notion that the first two arguments are automatically instantiated by pattern matching against the inductive family argument, we use the inaccessible pattern notation. The final definition is:
\begin{bsp}
$ \fun \head : ( A : \Set ) \ra ( n : \Nat ) \ra \vVec A \; (\suc n) \ra A $\\
$ \spc \head \inacc{B} \; \inacc{m} \; (\cons \; B \; m \; x \; xl) = x $
\end{bsp}
\subsection{Equality}
The following is called Martin-L\"of equality, another important example of an inductive family.
This equality relation is defined by a single constructor for reflexivity:
\begin{bsp}
$\data \Eq ( A : \Set ) : A \ra A \ra \Set $ \\
$\spc \refl : \EPi{a}{A}{\Eq A \; a \; a}$
\end{bsp}
As a simple example, we can now prove $ x + 0 = x $:
\begin{bsp}
$ \const \prof : \EPi{x}{\Nat}{\Eq \Nat (\add x \; \zero)\; x} $ \\ 
$ \spcx = \lam{y}\refl \Nat y$ 
\end{bsp}
Here one can see that proof checking is type checking.
Without having formally introduced type checking, we want to give some intuition how it is done.
This proofs passes the type checker because $\add x \; \zero$ reduces to $\zero$ and the type checker knows that $\zero$ is equal to itself.
Now if we want to prove $ 0 +  x = x $, we need more effort.
The following does not pass the type checker:
\begin{bsp}
$ \const \tprof : \EPi{x}{\Nat}{\Eq \Nat (\add \zero x)\; x} $ \\ 
$ \spcx \lam{y} \refl \Nat y$ 
\end{bsp}
The reason why is that the type checker can only see definitional equality.
Note that addition was defined by recursion on the second argument.
Technically, the generic value introduced for $y$ does neither match against the pattern $\zero$ in the first clause
nor against the pattern $\suc y$ in the second clause of $\add$, so $\add \zero x$ does not reduce to $x$ during type checking.
So what we need is a recursive proof, manually doing case distinction on x:
\begin{bsp}
$\fun \eqsucc : \EPi{x}{\Nat}\EPi{y}{\Nat} \Eq \Nat x \; y \ra \Eq \Nat (\suc x) \; (\suc y)$ \\
$\spc \eqsucc \inacc{x} \; \inacc{x} \; (\refl \inacc{\Nat} \; x) = \refl \Nat (\suc x)$
\vs\\
$\fun \tprof : ( x : \Nat ) \ra \Eq \Nat (\add \zero x) x$\\
$\spc \tprof \zero = \refl \Nat \zero$\\
$\spc \tprof (\suc x) = \eqsucc (\add \zero x)\; x \; (\tprof x)$
\end{bsp}

Recursive proofs need to be total (complete pattern coverage and termination) to be correct.

\section{Missing features}
Due to its use for interactivly building proofs, the full Agda has meta-variables that represent parts of tge proof that are still misssing. These 
Also, many bookkeeping arguments can be easily infered by the type checker,easying tedious pain for the user.
The user can mark these arguments as \emph{implicit}. The type checker will automatically infer them.
To keep Mugda simple, these features were not implemented.
Furthermore, we are limited to one universe $\Set$.
We could easily have $\Set$ as the type of all types ($\Set : \Set$), but then we are prone to paradoxoi and inconsitencies (\cite{coquand86analysis,coquand92pattern}).
The solution is to apply universe stratification and move to a predicative hierary of universes $ \Set = \Set_0 , \Set_1 , \Set_2 $ and so on. 
\section{Type Checking}

\renewcommand\Check[5]{#1;#2;#3\vdash#4\leftleftarrows#5}
\newcommand\IsType[4]{#1;#2;#3\vdash#4\;\mathbf{Type}}
\newcommand\DataType[5]{#1;#2;#3;#4\vdash#5\;\mathbf{DataType}}
\newcommand\ConType[5]{#1;#2;#3;#4\vdash#5\;\mathbf{ConType}}
\newcommand\Infer[5]{#1;#2;#3\vdash#4\rightrightarrows#5}
\newcommand\EqVal[4]{#1;#2\vdash#3\Leftrightarrow#4}
\newcommand\FeqVal[4]{#1;#2\vdash#3{\;\leftrightarrow\;}#4}
\newcommand\DeqVal[2]{\vdash#1\leftrightarrow#2}
\newcommand\LeqVal[4]{#1;#2;\vdash#3\leq#4}
\newcommand\FleqVal[4]{#1;#2\vdash#3{\;\ll\;}#4}
\newcommand\DLeqVal[2]{\vdash#1\leq#2}
\newcommand\SzLeq[2]{\vdash#1\sqsubseteq#2}

\subsection{Bidirectional type checking} 
For depentend types, a practice called \emph{bidirectional type checking} is used.
This means that the type checker has two modes:
one for checking that an expression has a certain type and one for infering the type of an expression.

\begin{description}
\item[check expression]:
\[\Check{k}{\rho_1}{\rho_2}{e}{v} \subseteq \cN \times \Env \times \Env \times \Expr \times \Val\]
\item[infer expression]:
\[\Infer{k}{\rho_1}{\rho_2}{e}{v} : \cN \times \Env \times \Env \times \Expr \ra \Val\]
\item[is type]:
\[\IsType{k}{\rho_1}{\rho_2}{e} \subseteq \cN \times \Env \times \Env \times \Expr \]
\end{description}

The environment $\rho_1$ will be used to bind fresh generic values to variables, and $\rho_2$ will have the type correspoding to this variables. in checking mode, the type checker might have to infer the type of the expression and then verify that the infered value is equal to the one to be checked against.
So one needs equality between two values.
We want to strengthen equality of coinducive values by allowing the type checker to unroll ($\forc$) a corecusive definition. We cannot allow unlimit unfolding, thus we keep track of unfolding and only allow one of the values to be unrolled.
We introduce the set $ \cF = \{\lF, \rF, \nF \} $ to keep track if we have unrolled the left or right value or not at all.
Thus we introduce simulataneously two judgements, the latter one operating on already unrolled values.
\[\EqVal{f}{k}{v_1}{v_2} \subseteq \cF \times \cN \times \Val \times \Val \]
\[\FeqVal{f}{k}{v_1}{v_2} \subseteq \cF \times \cN \times\Val \times \Val \]

\begin{figure}[p]

\begin{gather*}
%%Lam
\nru{\chklam}
{\Check{k+1}{\ins{\rho_1}{x}{k}}{\ins{\rho_2}{x}{v_a}}{e}{\wh \cl{t}{\ins{\rho}{x}{k}}}}
{\Check{k}{\rho_1}{\rho_2}{\lam{x}{e}}{\vpi \; x \; v_a \; \cl{t}{\rho}}}
%% Pi
\\[2ex]
\nru{\chkpi}
{\Check{k}{\rho_1}{\rho_2}{t_1}{\Set}
\qquad
\Check{k+1}{\ins{\rho_1}{x}{k}}{\ins{\rho_2}{x}{\cl{t_1}}{\rho}}{t_2}{\Set}
}
{\Check{k}{\rho_1}{\rho_2}{\EPi{x}{t_1}{t_2}}{\Set}}
\\[2ex]
%% Succ
\nru{\chksucc}
{\Check{k}{\rho_1}{\rho_2}{e}{\Size}}
{\Check{k}{\rho_1}{\rho_2}{\s e}{\Size}} 
%% Inf
\qquad
\nru{\chkinf}
{\Infer{k}{\rho_1}{\rho_2}{e}{v_2}
\qquad
\EqVal{\nF}{k}{v_2}{v_1}} 
{\Check{k}{\rho_1}{\rho_2}{e}{v_1}}
\end{gather*}
\caption{Checking expressions}
\end{figure}

\begin{figure}[p]
\begin{gather*}
%% Var
	  \nru{\infvar}{\lkup \; \rho_2 \; x \; v}{\Infer{k}{\rho_1}{\rho_2}{x}{v}}
\\[2ex]
%% Size
	  \nru{\infsize}{}{\Infer{k}{\rho_1}{\rho_2}{\Size}{\Set}}
\qquad
%% Infty
	  \nru{\infinfty}{}{\Infer{k}{\rho_1}{\rho_2}{\infty}{\Size}} 
\\[2ex]
%% App-1
\nru{\infappa}
{\Infer{k}{\rho_1}{\rho_2}{e_1}{\vpi \; x \; v_a \; \cl{b}{\rho}}
\qquad \Check{k}{\rho_1}{\rho_2}
{e_2}{v_a}}
{\Infer{k}{\rho_1}{\rho_2}{e_1 \; e_2}{\wh \cl{b}{\ins{\rho}{x}{\cl{e_2}{\rho_1}}}}}
%% App-2
\\[2ex]
\nru{\infappb}
{\Infer{k}{\rho_1}{\rho_2}{(e_1 \; e_2)\;\vec{e}}{v}}
{\Infer{k}{\rho_1}{\rho_2}{e_1 \; (e_2 \; \vec{e})}{v}}
\qquad
%% Def	
\nru{\infdef}
{\Sig \; \ff \; = (v,\_) }
{\Infer{k}{\rho_1}{\rho_2}{\ff}{v}}
\\[2ex]
\qquad
%% Con 
\nru{\infcon}
{\Sig \; \cc \; = v}
{\Infer{k}{\rho_1}{\rho_2}{\cc}{v}} 
%% Data
\nru{\infdata}
{\Sig \; \cc \; = (v,\_)}
{\Infer{k}{\rho_1}{\rho_2}{\DD}{v}} 
%% Const
\qquad
\nru{\infconst}
{\Sig \; \dd \; = (\_,v)}
{\Infer {k}{\rho_1}{\rho_2}{\dd}{v}}
\end{gather*}
\caption{Infering type of expressions}
\end{figure}

\begin{figure}
\begin{gather*}
\nru{}
{\IsType{k}{\rho_1}{\rho_2}{t_1}
\qquad
\IsType{k+1}{\ins{\rho_1}{x}{k}}{\ins{\rho_2}{x}{\cl{t_1}{\rho}}}{t_2}}
{\IsType{k}{\rho_1}{\rho_2}{\EPi{x}{t_1}{t_2}}}
\\[2ex]
\nru{}{}{\IsType{k}{\rho_1}{\rho_2}{\Set}}
\qquad
\nru{}
{\Check{0}{\rho_1}{\rho_2}{t}{\Set}}
{\IsType{k}{\rho_1}{\rho_2}{t}}
\end{gather*}
\caption{Type judgment}
\end{figure}

\begin{figure}[p]
\begin{gather*}
%% Force L
\nru{\eqlf}
{\forc \; v_1 = u_1 \qquad \forc \; v_2 = \stuck 
\qquad
f \neq \rF \qquad \FeqVal{\lF}{k}{u_1}{v_2}}
{ \EqVal{f}{k}{v_1}{v_2}}
%% Force R
\\[2ex]
\nru{\eqrf}
{\forc \; v_1 = \stuck \qquad \forc \; v_2 = u_2 
\qquad
f \neq \lF \qquad \FeqVal{\rF}{k}{v_1}{u_2}}
{ \EqVal{f}{k}{v_1}{v_2}}
\\[2ex]
%% Force N
\nru{\eqnf}
{\FeqVal{f}{k}{v_1}{v_2}}
{ \EqVal{f}{k}{v_1}{v_2}}
\qquad
%% Set
\nru{\eqset}
{}
{\FeqVal{f}{k}{\Set}{\Set}}
%% Size
\\[2ex]
\nru{\eqsize}
{}
{\FeqVal{f}{k}{\Size}{\Size}}
%% Infty
\qquad
\nru{\eqinfty}
{}
{\FeqVal{f}{k}{\infty}{\infty}}
%% Succ
\\[2ex]
\nru{\eqsuc}
{\EqVal{f}{k}{v_1}{v_2}}
{\FeqVal{f}{k}{\s v_1}{\s v_2}}
%% App
\qquad
%% Gen
\nru{\eqgen}
{}
{\EqVal{f}{k}{k'}{k'}}
\\[2ex]
\nru{\eqapp}
{\EqVal{f}{k}{v_1}{v_2} \qquad
\EqVal{f}{k}{c_1}{d_1} \ldots \EqVal{f}{k}{c_n}{d_n}}
{\FeqVal{f}{k}{v_1 (c_1 \ldots c_n)}{v_2 (d_1 \ldots d_n)}}
%% Pi
\\[2ex]
\nru{\eqpi}
{\EqVal{f}{k}{a_1}{a_2}
\qquad
\EqVal{f}{k+1}{\wh \cl{b_1}{\ins{\rho_1}{x_1}{k}}}{\wh \cl{b_2}{\ins{\rho_2}{x_2}{k}}}}
{\FeqVal{f}{k}{\vpi \; x_1 \; a_1 \; \cl{b_1}{\rho_1}}{\vpi \; x_2 \; a_2 \; \cl{b_2}{\rho_2}}}
%% Lam
\\[2ex]
\nru{\eqlam}
{\EqVal{f}{k+1}{\wh \cl{b_1}{\ins{\rho_1}{x_1}{k}}}{\wh \cl{b_2}{\ins{\rho_2}{x_2}{k}}}}
{\FeqVal{f}{k}{\vlam \; x_1 \; \cl{b_1}{\rho_1}}{\vlam \; x_2 \; \cl{b_2}{\rho_2}}}
%% Def
\\[2ex]
\nru{\eqdef}
{}
{\FeqVal{f}{k}{\ff}{\ff}}
%% Con
\qquad
\nru{\eqcon}
{}
{\FeqVal{f}{k}{\cc}{\cc}}
\end{gather*}
\caption{equality checking}
\end{figure}

After having defined the basic building blocks, now it has to be defined how individual declarations are checked.
\subsubsection{Constant declaration}
Checking a constand declaration is straightforward.
The steps are:
\paragraph*{Typechecking $\const \dd : A = e $}
\begin{enumerate}
\item
$\IsType{0}{\diamond}{\diamond}{A}$
\item
$ v_a := \wh \cl{A}{\diamond}$
\item
$\Check{0}{\diamond}{\diamond}{e}{v_a}$
\item
$\Sig := \Sig \cup \{\dd \ra (e,v_a)\}$
\end{enumerate}
We check that $A$ is a valid type, then check that $e$ has this type and then we update the signature.
\subsubsection{Datatype declarations}
We have to check that data type declarations follow the scheme of inductive families.
Actually, we here stay closer to how data type declarations are checked in the current version Agda.
It is a little more liberal then the schematas given in the original paper \cite{dubjier}.
Although this is not essential, we don't require that non-recursive arguments appear before recursive ones.
Second, as will we see, our notion of strict positiy is a little more lenient than in \cite{dubjier}.

A data type declarations need to be of the following scheme:
\begin{bsp}
$\data \DD ( \{+\} p_1 : P_1) \ldots ( \{+\} p_n : P_n) : \Gamma \ra \Set $\\
$\spc \ldots$\\
$\spc \cc : \Delta \ra \DD p_1 \ldots p_n t_1 \ldots t_m$\\
$\spc \ldots$\\   
\end{bsp}

\newcommand{\pos}{\mbox{pos}}
We define $\pos(\DD) := \{ i \vert p_i \mbox{ is declared strictly positive} \}. $ 
Note that the parameters $p_i$ have to be written down in the target of each constructor exactly as declared in the telescope.  
The parameters can be arbitrary types, while the types in $\Gamma$ and $\Delta$ have to be set-types, i.e of type $\Set$.
We require that datatype declarations are strictly positive.
Otherwise there are inconsistencies (\cite{paulinmohring93inductive}), i.e one can define recursion without using a $\fun$.
Strict positivy rougly means that in the type of the arguments $\Delta$ of every constructor , every recursive argument $\DD$ is not allowed in the domain of a function type or in a argument of an application.
An exception to this is that it may occur as a parameter to a type that is itself declared as strictly positive.
Of course it has to be checked that all strictly positive parameters are indeed themselves strictly positive.

\newcommand{\sposc}[3]{#1\vdash#2\mbox{ \textbf{sposc} }#3}
\newcommand{\spos}[3]{#1\vdash#2\mbox{ \textbf{spos} }#3}

\paragraph*{Typechecking}
For a data declaration  
\[\begin{array}{l}
\data \DD (\{+\} p_1 : P_1) \ldots (\{+\} p_n : P_n) : A \\
\spcx \vec{\gamma}
\end{array}
\]
the following steps are performed:
\begin{enumerate}
\item
$\DataType{0}{\diamond}{\diamond}{n}{(p_1 : P_1) \ra \ldots (p_n : P_n) \ra A}$
\item
$v_D := \wh \cl{(p_1 : P_1) \ra \ldots (p_n : P_n) \ra B}{\diamond}$
\item
$\Sig := \Sig \cup \{\DD \ra v_D\}$
\item
for every constructor $\cc : B  \in \vec{\gamma}$:
\begin{enumerate} 
\item
$\ConType{0}{\diamond}{\diamond}{\DD p_1 \ldots p_n}{(p_1 : P_1) \ra \ldots (p_n : P_n) \ra B}$
\item
$v_C := \wh \cl{(p_1 : P_1) \ra \ldots (p_n : P_n) \ra B}{\diamond}$
\item
$\pos := \{ i | p_i \mbox{ is declared strictly positive } \} $ 
\item
$\sposc{0}{v_C}{i}$ for every $i \in \pos(\DD) $ 
\item
$\sposc{0}{v_C}{\DD}$
\item
$\Sig := \Sig \cup \{\cc \ra v_C\}$
\end{enumerate}
\end{enumerate}

\newcommand{\nocc}[3]{#1\vdash#2\mbox{ \textbf{nocc} }#3}

\begin{figure}
\begin{gather*}
\nxru{}
{\mbox {if } k < n \mbox{ then } \IsType{k}{\rho_1}{\rho_2}{A}}
{\mbox {if } k >= n \mbox { then } \Check{k}{\rho_1}{\rho_2}{A}{\Set}}
{\DataType{k+1}{\ins{\rho_1}{x}{k}}{\ins{\rho_2}{x}{\cl{A}{\rho}}}{n}{B}}
{\DataType{k}{\rho_1}{\rho_2}{p}{\EPi{x}{A}{B}}}
\\[2ex]
\nru{}
{}
{\DataType{k}{\rho_1}{\rho_2}{n}{\Set}}
\end{gather*}
\caption{Data type judgement}
\end{figure}

\begin{figure}
\begin{gather*}
\ndru{}
{\mbox {if } k >= n \mbox { then } \Check{k}{\rho_1}{\rho_2}{A}{\Set}}
{\ConType{k+1}{\ins{\rho_1}{x}{k}}{\ins{\rho_2}{x}{\cl{A}{\rho}}}{\DD p_1 \ldots p_n}{B}}
{\ConType{k}{\rho_1}{\rho_2}{\DD p_1 \ldots p_n}{\EPi{x}{A}{B}}}
\\[2ex]
\nru{}
{\IsType{k}{\rho_1}{\rho_2}{\DD p_1 \ldots p_n t_1 \ldots t_m}}
{\ConType{k}{\rho_1}{\rho_2}{\DD p_1 \ldots p_n}{\DD p_1 \ldots p_n t_1 \ldots t_m}}
\end{gather*}
\caption{Constructor type judgment}
\end{figure}


\begin{figure}
\begin{gather*}
\nru{}
{\nocc{k}{i}{v_a} \qquad \nocc{k+1}{i}{\wh \cl{b}{\ins{\rho}{x}{k}}}}
{\nocc{k}{i}{\vpi \; x \; v_a \; \cl{b}{\rho}}  }
\\[2ex]
{\nocc{k+1}{i}{\wh \cl{b}{\ins{\rho}{x}{k}}}}
{\nocc{k}{i}{\vlam \; x \; \cl{b}{\rho}}  }
\\[2ex]
\nru{}
{\nocc{k}{i}{v} \qquad \nocc{k}{i}{\wh c_j} \mbox{ for all } j \in \{1 \ldots n \}}
{\nocc{k}{i}{v \; c_1 \ldots c_n}}
\qquad
\nru{}
{\nocc{k}{i}{v}}
{\nocc{k}{i}{\s v}}
\\[2ex]
\nru{}
{i \neq i'}
{\nocc{k}{i}{i'}}
\qquad
\nru{}
{}
{\nocc{k}{i}{v}}
\end{gather*}
\caption{Nonoccurence of generic value i}
\end{figure}

\begin{figure}
\begin{gather*}
\nru{}
{ \nocc{k}{i}{v_a} \qquad \spos{k+1}{i}{\wh \cl{b}{\ins{\rho}{x}{k}}}}
{\spos{k}{i}{\vpi \; x \; v_a \; \cl{b}{\rho}}  }
\\[2ex]
\nru{}
{\spos{k+1}{i}{\wh \cl{b}{\ins{\rho}{x}{k}}}}
{\spos{k}{i}{\vlam \; x \; \; \cl{b}{\rho}}  }
\\[2ex]
\ndru{}
{\nocc{k}{i}{\wh v_j} \mbox{ for all } j \in \{1 \ldots m \}} 
{\spos{k}{i}{\wh p_j} \mbox{ for all } j \in \pos(\DD) \qquad \nocc{k}{i}{\wh p_j} \mbox{ for all } j \notin \pos(\DD) } 
{\spos{k}{i}{\DD p_1 \ldots p_n \; c_1 \ldots c_m}}
\\[2ex]
\nru{}
{\spos{k}{i}{v} \qquad \nocc{k}{i}{\wh c_j} \mbox{ for all } j \in \{1 \ldots n \}}
{\spos{k}{i}{v \; c_1 \ldots c_n}}
\\[2ex]
\nru{}
{\spos{k}{i}{v}}
{\spos{k}{i}{\s v}}
\qquad
\nru{}
{}
{\spos{k}{i}{v}}
\end{gather*}
\caption{Strictly positive occurence of generic value i}
\end{figure}


\begin{figure}
\label{sposcf}
\begin{gather*}
\nru{}
{ \spos{k}{i}{v_a} \qquad \sposc{k+1}{i}{\wh \cl{b}{\ins{\rho}{x}{k}}}}
{\sposc{k}{i}{\vpi \; x \; v_a \; \cl{b}{\rho}}  }
\nru{}
{}
{\sposc{k}{i}{v}}
\end{gather*}
\caption{Strict positivity test for generic value i in constructor arguments}
\end{figure}

The strict positivity test for a generic variable i is given in figure \ref{sposcf}.
These judgement can be defined analougsly for the occurence of a data identifier $\DD$.
Later the information about positive parameters will be usefull for subtyping.

\begin{bsp}
$\data \Bad : \Set $ \\
$\spc \ok : \Bad$\\
$\spc \bad : ((\Nat \ra \Bad) \ra \Bad) \ra \Bad  $
\end{bsp}
is not allowed because of the non-positive occurence of $\Bad$ in the argument of $\bad$.

\begin{bsp}
$\data \Bad : ( + A : \Set ) : \Set $\\
$\spc \ok : \Bad A$\\
$\spc \bad : (A \ra \Bad) \ra \Bad A$
\end{bsp}
is not allowed because of the negative occurence of $A$ in the argument of $\bad$.

\subsubsection{Function declarations}

\newcommand\checkcl[2]{#1\mbox{ \textbf{checkcl}}#2}
\newcommand\checkp[5]{(#1)\vdash#2\mbox{ \textbf{checkp} }#3\Rightarrow (#4),#5}
\newcommand\checkps[5]{(#1)\vdash#2\mbox{ \textbf{checkps} }#3\Rightarrow (#4),#5}
\newcommand\checkinacc[2]{(#1)\vdash\mbox{ \textbf{checkinaccs} }#2}
\newcommand\inst[5]{#1,#2\vdash#3\mbox{ \textbf{inst} }#4\Rightarrow #5}
\newcommand\instlist[5]{#1,#2\vdash#3\mbox{ \textbf{instList} }#4\Rightarrow #5}


\paragraph*{Overview}
We need to check every clause $\ff \vec{p} e$ of a function against the delcared type.
For each clause, we have to check the patterns $\vec{p}$ and then the right hand side $e$.
When checking the patterns, we first skip the inacessible patterns, 
replacing them with \emph{flexible} generic values.
These variables get instantiated to concrete values when matching
the indices of a constructor pattern.
After we are done with the accesible patterns, 
we then check that the inacessible patterns written down 
by the user was correct.
\newcommand{\subst}[2]{#1\{#2\}}
\newcommand{\substClos}[2]{#1\{#2\}}
\newcommand{\substEnv}[2]{#1\{#2\}}
\newcommand{\compS}[2]{\mathrm{compS}#1\;#2}


\newcommand\ptc{\mathrm{p2c}}
\newcommand\inaccenv{\mathrm{inaccEnv}}
\newcommand\pte{\mathrm{p2e}}
\newcommand\pstes{\mathrm{ps2es}}


\paragraph*{Checking functions}
We here handle the case of a mutual $\fun$ declaration.
A mutual $\cofun$ block is typechecked just the same, and non-mutual declaration ist just a special case of a mutual declaration.
To check a mutual declaration
\begin{bsp}
$\mutual$\\
$\spcx \fun f_1 : t_1 $\\
$\spcx \spc \vec{\gamma}_1$\\
$\spcx \ldots$\\
$\spcx \fun f_n : t_n $\\
$\spcx \spc \vec{\gamma}_n$
the following steps are performed:
\end{bsp}
\begin{enumerate}
\item
for every $i \in \{1 \ldots n \} $
\begin{enumerate}
\item
$\IsType{0}{\diamond}{\diamond}{t_i}$
\item
$v_i := \wh \cl{t_i}{\diamond}$
\item
$\Sig := \Sig \cup \{f_i \ra (v_i,\vec{\gamma_n},\bot)\}$
\end{enumerate}
We check every declared type. The signature is updated, 
but only the type can be used as the clauses have yet to be checked. 
\item
for every $i \in \{ 1 \ldots n \} $
\begin{enumerate}
\item
$\checkcl{t_i}{\gamma}$ for every $ \gamma \ni \vec{\gamma}_i$
\item
$\Sig := \Sig \cup \{f_i \ra (v_i,\vec{\gamma_n},\top)\}$
\end{enumerate}
Now that the clauses are checked, we can update the signature to enable reduction of the defined functions.
\end{enumerate}

As noted, flexible variables get instantiated by $\insŧ{}{}{}{}{}$ , returning a substitution
\begin{definition}
A substitution $\sigma : \ves{(\cN \times \Val)} $ denotes a partial mapping of flexible generic values to values.
$\diamond$ is the empty substitution.
$\sigma[k \rightarrow v] $ maps $k$ to $v$ and all other variables according to $\sigma$
\end{definition}

Now we define simultaneously the application of a substitution on values,closures and environments.

\paragraph*{Substution application}
\[
\begin{array}{ll}
\multicolumn{2}{l}{\subst{\_}{\_} : \ves{(\cN \times \Val)} \ra \Val \ra \Val} \\
\subst{\sigma}{k} & = v \mbox{ if } (k,v) \in \sigma \\
\subst{\sigma}{v cl_1 \ldots cl_n} & = \subst{\sigma}{v} \substClos{\sigma}{c_1} \ldots \substClos{\sigma}{c_n}\\
\subst{\sigma}{\s v} & = \s \subst{\sigma}{v} \\
\subst{\sigma}{\vpi x v_a \cl{b}{\rho}} & = \vpi x \subst{\sigma}{a_v} \cl{b}{\subst{\sigma}{\rho}} \\
\subst{\sigma}{\vlam x \cl{b}{\rho}} & = \vpi x \cl{b}{\subst{\sigma}{\rho}} \\
\subst{\sigma}{v} & = v \mbox{ othewise }
\end{array}
\]

\[
\begin{array}{ll}
\multicolumn{2}{l}{\substClos{\_}{\_} : \ves{(\cN \times \Val)} \ra \Clos \ra \Clos} \\
\substClos{\sigma}{\cl{e}{\rho}} & = \cl{e}{\substEnv{\sigma}{\rho}} 
\end{array}
\]


\[
\begin{array}{ll}
\multicolumn{2}{l}{\substEnv{\_}{\_} : \ves{(\cN \times \Val)} \ra \Env \ra \Val} \\
\substEnv{\sigma}{\diamond} & = \diamond \\
\substEnv{\sigma}{(x,v)\rho} & = (x,\subst{\sigma}{v}) \substEnv{\sigma}{\rho} \\
\substEnv{\sigma}{(x,c)\rho} & = (x,\substClos{\sigma}{c}) \substEnv{\sigma}{\rho}
\end{array}
\]

Now we define composition of two substitutions $\sigma_1$ and $\sigma_2$.
It is assumed that the domains of the substituions are disjunct and
ther are no occurences of the variables of $\sigma_1$ in the values of the codomain of $\sigma_2$.  
\[
\begin{array}{ll}
\multicolumn{2}{l}{\compS : \ves{(\cN \times \Val)} \ra \ves{(\cN \times \Val)} \ra \ves{(\cN \times \Val)}} \\
\compS{\sigma_1}{\sigma_2} & =  \sigma'_1 \sigma_2 \mbox{ where } (x,\subst{\sigma_2}{v}) \in \sigma'_1 \mbox{ iff } (x,v) \in \sigma_1 \\
\end{array}
\]



\[
\checkp{(k,f,\sigma,\rho_1,\rho_2)}{p}{v}{(k',f',\sigma',\rho'_1,\rho'_2,v')}
\]

\[
\begin{array}{ll}
\multicolumn{2}{l}{\ptc : \cN \times \Env \times \ves{\cN} \times \Pat \ra \Clos}\\
\ptc k \rho f p & = \cl{e}{\rho_2} \mbox{ where } \\
& (e,k') = \pte k p \\
& \rho_2 = \inaccenv \rho k f \\
\multicolumn{2}{l}{\pte : \cN \times \Pat \ra (\Expr,k)}\\
\pte k x & = (x,k+1) \\
\pte k (\cc \diamond) & = (\cc,k) \\
\pte k (\cc \vec{p}) & = (\cc \vec{e},k') \mbox{ where } (\vec{e},k') = \pstes k \vec{p}\\
\pte k (\s p) & = (\s e,k') \mbox{ where } (e,k') = \pte k p\\
\pte k \inacc{e} & = (i_k,k+1) \\
\multicolumn{2}{l}{\inaccenv : \Env \times \cN \times \ves{cN} \ra \Env}\\
\inaccenv \rho k \diamond & = \rho \\
\inaccenv \rho k (f \vec{f}) & = \inaccenv \rho k \vec{f} \mbox{ if } f < k  \\
\inaccenv \rho k (f \vec{f}) & = \ins{(\inaccenv \rho k \vec{k})}{i_f}{f} \mbox { if } f \geq k 
\end{array}
\]

\begin{figure}
\begin{gather*}
\ndru{}
{\checkps
{0,\diamond,\diamond,\diamond,\diamond}
{\vec{p}}{v_t}{k,f,\sigma,\rho_1,\rho_2}{v_e}}
{\checkinacc{k,\sigma,\rho_1,\rho_2}{f} 
\qquad
\Check{k}{\rho_1}{\rho_2}{e}{v_e}
}
{
\checkcl{v_t}{\ff \vec{p} e}
}
\end{gather*}
\caption{Checking a clause}
\end{figure}

\begin{figure}
\begin{gather*}
% VAR
\nru{\chkpvar}
{v_b = \wh \cl{b}{\ins{\rho}{x}{k}}}
{
\checkp
{k,f,\sigma,\rho_1,\rho_2}
{y}
{\vpi x v_a \cl{b}{\rho}}
{k+1,f,\sigma,\ins{\rho_1}{y}{k},\ins{\rho_2}{y}{v_a}}
{v_b}}
% CON
\\[2ex]
\nxru{\chkpcon}
{\Sig{\cc}=v_c \qquad
\checkps{k,f,\sigma,\rho_1,\rho_2}
{\vec{p}}
{v_c}
{k',f',\sigma',\rho'_1,rho'_2}
{v'_c}
}
{\inst{k}{f}{v'_c}{v_a}{\sigma_2} \qquad
\ptc k \rho'_1 f p =  c}
{\sigma'' = \compS{\sigma'}{\sigma_2} \qquad
\rho''_2 = \substEnv{\sigma''}{\rho'_2} \qquad
v_b = \subst{\sigma''}{\wh \cl{b}{\ins{\rho}{x}{c}}}
}
{\checkp
{k,f,\sigma,\rho_1,\rho_2}
{\cc \vec{p}}
{\vpi x v_a \cl{b}{\rho}}
{k',f',\sigma'',\rho'_1,\rho''_2}
{v_b}
}
% SUCC
\\[2ex]
\ndru{\chkpsucc}
{
\checkp{k,f,\sigma,\rho_1,\rho_2}{p}{\wh \clp{\Size \ra \Size}{\diamond}}
{k',f',\sigma',\rho'_1,\rho'_2}{v'}}
{
\EqVal{k'}{\nF}{v_a}{v'} \qquad
\ptc k \rho'_1 f p = c
}
{
\checkp
{k,f,\sigma,\rho_1,\rho_2}
{\s p}
{\vpi x v_a \cl{b}{\rho}}
{k',f,\sigma,\ins{\rho_1}{y}{k},\ins{\rho_2}{y}{v_a}}
{v_b}}
% DOT
\\[2ex]
\nru{\chkpinacc}
{v_b = \wh \cl{b}{\ins{\rho}{x}{k}}}
{
\checkp
{k,f,\sigma,\rho_1,\rho_2}
{\inacc{e}}
{\vpi x v_a \cl{b}{\rho}}
{k+1,(k,e)f,\sigma,\rho_1,\rho_2}
{v_b}}
\end{gather*}
\caption{Checking a pattern}
\end{figure}

\begin{figure}
\begin{gather*}
\nru{}
{}
{
\checkps
{k,f,\sigma,\rho_1\rho_2}
{\diamond}
{v}
{k,f,\sigma,\rho_1,\rho_2}
{v}}
\\[2ex]
\ndru{}
{\checkp
{k,f,\sigma,\rho_1,\rho_2}
{p}
{v}
{k',f',\sigma',\rho'_1,\rho'_2}
{v'}
}
{\checkps
{k',f',\sigma',\rho'_1,\rho'_2}
{\vec{p}}
{v'}
{k'',f'',\sigma'',\rho''_1,\rho''_2}
{v''}
}
{
\checkps
{k,f,\sigma,\rho_1\rho_2}
{p \vec{p}}
{v}
{k'',f'',\sigma'',\rho''_1,\rho''_2}
{v''}}
\end{gather*}
\caption{Checking multiple patterns}
\end{figure}

\begin{figure}
\begin{gather*}
\nru{}
{}
{
\checkinacc
{k,\sigma,\rho_1,\rho_2}
{\diamond}
}
\\[2ex]
\ndru{}
{
\Check{k}{\rho_1}{\rho_2}{e}{\subst{\sigma}{t_e}} \qquad (i,v) \in \sigma 
}
{
\EqVal{k}{\nF}{\wh \cl{e}{\rho_1}}{v} 
\qquad
\checkinacc
{k,\sigma,\rho_1,\rho_2}
{\vec{f}}
}
{
\checkinacc
{k,\sigma,\rho_1,\rho_2}
{(i,(e,t_e)) \vec{f}}
}
\end{gather*}
\caption{Checking inaccesible patterns}
\end{figure}


\begin{figure}
\begin{gather*}
\nru{}
{}
{\inst{k}{f}{k'}{v}{(k',v)}}
%
\nru{}
{}
{\inst{k}{f}{v}{k'}{(k',v)}}
%
\\[2ex]
\nru{}
{{\instlist{k}{f}{\wh{c_1} \ldots \wh{c_n}}{\wh{d_1} \ldots \wh{d_n}}{\sigma}}}
{\inst{k}{f}{\DD c_1 \ldots c_n}{\DD d_1 \ldots d_n}{\sigma}
}
\\[2ex]
\nru{}
{{\instlist{k}{f}{\wh{c_1} \ldots \wh{c_n}}{\wh{d_1} \ldots \wh{d_n}}{\sigma}}}
{\inst{k}{f}{\cc c_1 \ldots c_n}{\cc d_1 \ldots d_n}{\sigma}
}
\\[2ex]
\nru{}
{\inst{k}{f}{v_1}{v_2}{\sigma}}
{\inst{k}{f}{\s v_1}{\s v_2}{\sigma}
}
\nru{}
{\inst{k}{f}{v_1}{\infty}{\sigma}}
{\inst{k}{f}{\s v_1}{\infty}{\sigma}
}
\\[2ex]
\nru{}
{\EqVal{k}{N}{v_1}{v_2}}
{\inst{k}{f}{v_1}{v_2}{\diamond}
}
\nru{}
{}
{\instlist{k}{f}{\diamond}{\diamond}{\diamond}}
\\[2ex]
\ndru{}
{\inst{k}{f}{v_1}{w_1}{\sigma}}
{\instlist{k}{f}{\subst{\sigma}{v_2} \ldots \subst{\sigma}{v_n}}{\subst{\sigma}{w_2} \ldots \subst{\sigma}{w_n}}{\sigma'}}
{\instlist{k}{f}{v_1 v_2 \ldots v_n}{w_1 w_2 \ldots w_n}{\compS \sigma \sigma'}}
\end{gather*}
\caption{Instantiating flexible variables}
\end{figure}

\section{Pattern matching}
Dependend pattern matching was introduced by Thierry Coquand in \cite{coquand92pattern}.
The patterns for all the clauses of a function are assumed to be complete. This means that during runtime
it is guaranteed that matching always succeeds.

\chapter{Termination Checking with the size-change principle}

\section{Matrix notation}

\newcommand{\diag}{\mathrm{diag}}

For a set $R$ we write $R^{m,n}$ for the set of matrices with $n$ rows and $m$ columns and elements in $R$.
We write $\alpha_{i,j}$ for the element in the $j$th row and $i$th column.
For a matrix $\alpha \in R^{m,n}$ we define the length $|\alpha| := (m,n)$.
For a square matrix $\beta \in R^{m,m}$ we define the set of diagonal elements $\diag \beta := \{ \beta_{i,i} \vert i \leq m\}$.
We write $\alpha_{i,j}$ for the element in the $i$-th row and $j$-th column.


\newcommand{\order}{\mathrm{Order}}
\section{Structural recursion}

\begin{definition}[Order]
\[ O := \{ < , \leq , ? \} \]
\end{definition}
Axiom1:
\[ e \; < \; C \; \vec{e_1} \; e \; \vec{e_2} \]
\\
Axiom2:
\[ f \; \vec{e} \; \leq \; f \] 

Axiom 2 is needed for higher order data types like $\Ord$.

\begin{definition}[Order Addition]
\[
\begin{array}{l | c |c | c }

 +    &  <   &  \leq   &  ? \\
 \hline 
 <    &  <   & <       &  <   \\
 \leq &  <   & \leq    &  \leq \\
 ?    &  <   & \leq    &  ?
\end{array}
\]

\end{definition}


\begin{definition}[Order multiplication]
\[
\begin{array}{l | c |c | c}
 *   &  <  &  \leq & ? \\
\hline
  <   & <    & <    &   ? \\
 \leq & <    & \leq &   ?  \\
 ?    & ?    & ?    &   ?
\end{array}
\]
\end{definition}

For the following definiton, we need to enforce that all clauses of a function definition have the same number of patterns.
\begin{definition}
For function $f$ we define the arity as The arity $ar(f) = \# \vec{p} $ where $f \vec{p} = e $ is some clause.   
\end{definition}

\begin{definition}{Relating patterns and arguments}
\end{definition}

The following definitons are taken from \cite{abelAltenkirch:predStRec}

\begin{definition}
An expression of the form $\ff \vec{e}$ is called a call.
\end{definition}

\newcommand{\mino}{\mathrm{min}\;}
\newcommand{\maxo}{\mathrm{max}\;}
\newcommand{\supo}{\mathrm{sup}\;}
\newcommand{\info}{\mathrm{inf}\;}

\begin{definition}[Order minimum]
\[
\begin{array}{l | c |c | c}
 \mino &  <      &  \leq    & ? \\
\hline
  <    & <       &  \leq    &   ? \\
 \leq  & \leq    &  \leq    &   ?  \\
 ?     & ?       &  ?       &   ?
\end{array}
\]
\end{definition}

\begin{definition}[Order maximum]
\[
\begin{array}{l | c |c | c}
 \maxo & <      &  \leq    &   ? \\
\hline
  <    & <       & <       &   <    \\
 \leq  & <       & \leq    &   \leq  \\
 ?     & <       & \leq    &   ?
\end{array}
\]
\end{definition}

\begin{definition}[Order supremum]
\[
\begin{array}[t]{ll}
\multicolumn{2}{l}{\supo : \ves{\order} \ra \order}\\ 
\supo \diamond = ? \\
\supo (o \vec{o} ) = \maxo o \; (\supo \vec{o}) \\
\end{array}
\]
\end{definition}

\begin{definition}[Order infimum]
\[
\begin{array}[t]{ll}
\multicolumn{2}{l}{\info : \ves{\order} \ra \order}\\ 
\info \diamond = < \\
\info (o \vec{o} ) = \mino o \; (\info \vec{o}) \\
\end{array}
\]
\end{definition}

\newcommand{\cmp}{\mathrm{cmp}\;}
\newcommand{\cmpv}{\mathrm{cmpv}}
\newcommand{\cmps}{\mathrm{cmps}}
\newcommand{\cmpsv}{\mathrm{cmpsv}}
\newcommand{\etp}{\mathrm{etp}\;}


\[
\begin{array}[t]{ll}
\multicolumn{2}{l}{\cmp : \Expr \times \Pat \ra \order}\\ 
\cmp e_1 \; \inacc{e_2} & = \cmp e_1 \; p \mbox{ if } \etp e_2 = p \\
\cmp x \; p & = \cmpv x \;p  \mbox{ where } x \in \cV \\
\cmp (x \vec{e}) \; p & = \cmpv x \;p  \mbox{ where } x \in \cV \\
\cmp (\cc e_1 \ldots e_i) \; (\cc p_1 \ldots p_j ) & = \info (\cmp e_1 p_1) \ldots (\cmp e_m p_m)\\
& \; \mbox { where } m = min\{i,j\}\\ 
\cmp \cc \; (\cc \diamond) & = \leq \\
\cmp (\s e_1) \; (\s e_2) & = \cmp e_1 \; e_2 \\
\cmp (f \vec{e}) \; f & = \leq \mbox{ where } f \in \cV \\
\cmp e \; p & = ? \mbox{ otherwise}
\end{array}
\]

\[
\begin{array}[t]{ll}
\multicolumn{2}{l}{\etp : \Expr \ra \Pat}\\ 
\etp x & = x \mbox{ where } x \in \cV \\
\etp (\cc e_1 \ldots e_n ) & = \cc (\etp e_1) \ldots (\etp e_n) \\
\etp \cc  & = \cc \diamond \\
\etp (\s e) = \s (\etp e) \\
\etp e = \stuck \mbox{ otherwise}
\end{array}
\]

\[
\begin{array}[t]{ll}
\multicolumn{2}{l}{\cmpv : \cV \ra \Pat \ra \order}\\ 
\cmpv x \; x & = \leq \\
\cmpv x \; (\cc p_1 \ldots p_n ) & = < * \; \supo (\cmpv p_1) \ldots (\cmpv p_n) \mbox{ if } \cc \mbox{ is inductive }\\   \\
\cmpv x \; \inacc{e} & = \cmpv x p \mbox{ if } \etp e = p \\
\cmpv x \; \s p & = ? \\
\cmpv x \; (\cc \vec{p}) & = ? \mbox{ where } \cc \mbox{ is coinductive}\\
\cmpv x \; p & = ? \mbox{ otherwise}
\end{array}
\]

\[
\begin{array}[t]{ll}
\multicolumn{2}{l}{\cmps : \Expr \ra \Pat \ra \order}\\ 
\cmps e_1 \; \inacc{e_2} & = \cmp e_1 \; p \mbox{ if } \etp e_2 = p \\
\cmps x \; p & = \cmpsv x \;p  \mbox{ where } x \in \cV \\
\cmps (\s e_1) \; (\s e_2) & = \cmps e_1 \; e_2 \\
\cmps e \; p & = ? \mbox{ otherwise}
\end{array}
\]

\[
\begin{array}[t]{ll}
\multicolumn{2}{l}{\cmpsv : \cV \ra \Pat \ra \order}\\ 
\cmpsv x \; x & = \leq \\
\cmpsv x \; \inacc{e} & = \cmpv x p \mbox{ if } \etp e = p \\
\cmpsv x \; \s p & = < * \; \cmpsv x \; p \\
\cmpsv x \; p & = ? \mbox{ otherwise}
\end{array}
\]

Note that coinductive objects are not well-founded. So we do have $\cmp x (\cc x) = ? $ when $\cc$ is a coinductive construcotor.
Also we have $\cmp x \; (\s x) = ? $ because the size type is not well-founded.
We will later allow the judgment $x < \s x$ for admissble size arguments.


\begin{definition}[Call matrix]
Let $f (p_1 \ldots p_n) e$ be a function clause and $g (e_1 \ldots e_n$ be a call in $e$.
The associated call matrix   is defined by $(f,M,g)$ where $M$ is a $(ar f \times ar g)$ order matrix defined by:
\[ 
\begin{array}[t]{l}
M(i,j) = \cmp e_j p_i \mbox{ if } j <= m \\
M(i,j) = ? \mbox{ otherwise} 
\end{array}
\]
\end{definition}
In the previous definition, we took into consideration that there can be calls $f e_1 \ldots e_n$ where $n < ar f$ because a function might be used in a higher order fashion. We fill up the missing elements with $?$. 

\begin{definition}[Call Matrix multiplication]
\[(f,M_1,g) \times (g,M_2,h) := (f,M_1 \times M_2,h)\]
\end{definition}

\begin{definition}[Call Graph]
A finite set of Call matrices is called a \emph{call graph}
\end{definition}

\begin{definition}[Call Graph completion]
Call graph completion is closing the set under multiplication.
Completion of a finite set is also finite. 
\end{definition}

\section{Size change principle}

The termination criterion in \cite{abelAltenkirch:predStRec} can be replaced with the
simpler and more powerful criterion from \cite{lee01sizechange}
Their formalism is based on bipartit graphs instead of call matrixes, but these are
equivalent representations.

\begin{definition}
A callmatrix (f,M,f) is \emph{idempotent} if $ M \times M = M $ 
\end{definition}

\begin{definition}[Size change principle]
A function f is terminating, if every idempotent callmatrix  $ \in \mathrm{complete g} $ from f to f has a decreasing element $<$ on
the diagonal.
\end{definition}

David Wahlstedt proofs the soundness of the size-change principle for a depently typed language 
for first order data types (no function space in constructor arguments, so Axiom 2 is not needed ).
Andreas Abel has shown the soundness of the structual order defined by Axiom 1 and 2 for the simple typed
language foetus.
Note that the order from Axiom 1 and 2 is actually not valid in impredicative systems.





\section{Examples}

Recall the addition function on natural numbers:

\begin{bsp}
$\fun \add : \Nat \ra \Nat \ra \Nat$\\
$\spc \add x \; \zero = x $\\
$\spc \add x \; (\suc y) = \suc (\add x \; y)  $
\end{bsp}
The single recursive call is $\add x \; y$ in the second clause of $add$.
To build the call matrix we need to calculate
\begin{bsp}
$\cmp x \; x = \leq $\\
$\cmp x \; (\suc y) = ? $\\
$\cmp y \; x = ? $\\
$\cmp y \; (\suc y) = < $ 
\end{bsp}

The call set is thus the singleton set with the matrix
\[
M_1 := \add\begin{pmatrix}
\leq & ? \\
?    & < \\
\end{pmatrix}\add
\]

To complete the call graph, we compose $M_1$ with itself.

\[
\add\begin{pmatrix}
\leq & ? \\
?    & < \\
\end{pmatrix}\add
* 
\add\begin{pmatrix}
\leq & ? \\
?    & < \\
\end{pmatrix}\add
=
\add\begin{pmatrix}
\leq & ? \\
?    & < \\
\end{pmatrix}\add
\]

which yields an identical matrix.
Thus, we are finished completing and $M_1$ is an idempotent matrix.
As it has a $<$ on the diagonal, the definition for add is terminating.

Now suppose we switch the arguments in the recursive call:
\begin{bsp}
$\fun \add : \Nat \ra \Nat \ra \Nat$\\
$\spc \add x \; \zero = x $\\
$\spc \add x \; (\suc y) = \suc (\add y \; x)  $
\end{bsp}

The call set is then 
\[
M_2 := \addt\begin{pmatrix}

? & \leq \\
< & ?   \\
\end{pmatrix}\addt
\]
Composing $M_2$ with itself yields
\[
M_3 := \addt \begin{pmatrix}
< & ?    \\
? & <    \\
\end{pmatrix} \addt
\] 
which is an idempotent matrix.No further matrix are added during completion.
As $M_3$ has an $<$ on the diagonal, the definiton for $\addt$ is also accepted.
Definition with permuted arguments are a streng of the size-change principle.  

\subsection{A mutual definition}

\begin{bsp}
$\mutual$\\
$\spcx \fun \even : \Nat \ra \Bool$\\
$\spcx \spc \even \zero = \tt $\\
$\spcx \spc \even (\suc x) = \odd x $\\
$\spcx \fun \odd : \Nat -> \Bool$\\
$\spcx \spc \odd \zero = \ff$\\
$\spcx \spc  odd (\suc x) = \even x$
\end{bsp}
The call set is 
\[
\even \begin{pmatrix}
< \\
\end{pmatrix}\odd
, 
\odd \begin{pmatrix}
< \\
\end{pmatrix}\even
\] 
In the completed call set includes the idempotent matrices
\[
\even \begin{pmatrix}
< \\
\end{pmatrix}\even
, 
\odd \begin{pmatrix}
< \\
\end{pmatrix}\odd
\] 
which have an decreasing element.
\subsection{Brouwer ordinals}
The so called Brouwer ordinal notations can be defined with
\begin{bsp}
$\data \Ord : \Set$ \\
$\spc \ozero : \Ord $\\
$\spc \olim : (\Nat \ra \Ord ) \ra \Ord $
\end{bsp}
This is an example of a higher-order data type, because the argument to $\olim$ contains a function space.
Now let's define ordinal addition:
\begin{bsp}
$\fun \addOrd : \Ord \ra \Ord \ra \Ord$\\
$\spc \addOrd x \; \ozero = x $\\
$\spc \addOrd x (\olim f) = \olim ( \lam{y} \addOrd  x \; (f \; y))  $
\end{bsp}
The interesting comparison is $(f \; y)$ against $\olim f$.
We have  $f \; y \leq f$ (Axiom 2)and $\olim f < f $ (Axiom 1) thus with transitivity $ f\; y < \olim f$.
And indeed we get:
\begin{bsp}
$\cmp x \; x = \leq $\\
$\cmp x \; (\olim f) = ? $\\
$\cmp (f \; y) \; x = ? $\\
$\cmp (f \; y) \; (\olim f) = \cmpv f (\olim f) = < * \supo (\cmpv f \; f) = < * \leq = < $ 
\end{bsp}
The corresponding call set is thus shown terminating as in the $\add$ case shown above.
\section{Extending the Order}
Definition of pair data type.
But now, the uncurried versions do no longer termination check.
The reason is that we need to keep the information.

\newcommand{\coll}{\mathrm{collapse}\;}

\begin{definition}[Extended Order]
We define the set $O^{+}$ by   
\begin{itemize}
\item
$ O \subseteq O^{+}$
\item
if $ M \subseteq O^{+} , n \in \cN , m \in \cN $ then $M^{n,m} \in O^{+}$
\end{itemize}
and nothing else is in $O^{+}$.
\end{definition}

\begin{definition}[Extended Order operations]
We extend the $ *, \mino , \maxo $ given in the previous section by the following additional clauses, using the new operation $\coll : M(O^{+}) \ra O^{+} $
\[
\begin{array}{ll}
\alpha * \beta = \left\{ \begin{array}{l}
                          \alpha \times \beta \mbox{ if} |\alpha| = | \beta | \\
                          (\coll \alpha) * (\coll \beta) \mbox{ otherwise}  
                          \end{array}\right. \\
\alpha * \leq = \alpha \\
\alpha * ? = ? \\
\alpha * < = (\coll \alpha) * < \\

\maxo \alpha  \beta =  \left\{ \begin{array}{l}
                          \gamma \mbox{ with } \gamma_{i,j} = \maxo \alpha_{i,j} \beta_{i,j} \mbox{ if} |\alpha| = |\beta| \\
                          \maxo (\coll \alpha) (\coll \beta) \mbox{ otherwise}  
                          \end{array}\right. \\
\maxo \alpha o = \maxo (\coll \alpha) o \mbox{ for } o \in O\\
\maxo o \alpha = \maxo o (\coll \alpha) \mbox{ for } o \in O\\

\mino \alpha  \beta =  \left\{ \begin{array}{l}
                          \gamma \mbox{ with } \gamma_{i,j} = \mino \alpha_{i,j} \beta_{i,j} \mbox{ if} |\alpha| = |\beta| \\
                          \mino (\coll \alpha) (\coll \beta) \mbox{ otherwise}  
                          \end{array}\right. \\
\mino \alpha o = \mino (\coll \alpha) o \mbox{ for } o \in O\\
\mino o \alpha = \mino o (\coll \alpha) \mbox{ for } o \in O\\

\coll \alpha = \info (\diag \alpha) 
\end{array}
\]
\end{definition}

In the definiton of $\cmp$ we change the clause 
\[
\begin{array}[t]{ll}
\cmp (\cc e_1 \ldots e_i) \; (\cc p_1 \ldots p_j ) & = \info (\cmp e_1 p_1) \ldots (\cmp e_m p_m)\\
& \; \mbox { where } m = min\{i,j\}\\ 
\end{array}
\]
to
\[
\begin{array}[t]{ll}
\cmp (\cc e_1 \ldots e_i) \; (\cc p_1 \ldots p_j ) & = 
\left\{\begin{array}{l} \gamma \mbox{ with } \gamma_{i,j} = \cmp e_i e_j \\
\; \mbox{ if } i = j \geq 2 \\
\info (\cmp e_1 p_1) \ldots (\cmp e_m p_m)\\
\; \mbox { where } m = min\{i,j\} \mbox{ otherwise } 
\end{array}
\right.
\end{array}
\]
so we can use more information about the componenets during termination checking.

\begin{definition}[Completion with extended order]
is also finite because matrices are only getting smaller or a collapsed to a single value.
\end{definition}

\section{Examples}


\subsection{Huet list reversion}
First, we present a peculiar list reversion algorithm.

The algorithm cannot be verified terminating.
Note the call $\rev A \; (\revb A \; x \; xs)$.
It is not structural recursive.
We can reason that $\revb$ does not return a bigger list.  
A termination checker would need the information that revb does not increase the size of the list to ensure termination.
We can fix this by using vectors instead of lists:
Now the termination checker can verify the termination.
So we note that better type information helps the termination checker by making more definitions structurally recursive.

But using vectors instead of lists is not always easily possible.
Consider a filter function on Lists. We only know that this function does not increase the size, but
we don't know the exact length.
This is one of the reasons we will introduce a special size type in the next chapter.

\begin{figure}
$\mutual $  \\
$ \spc \fun \rev : \EPi{A}{\Set}{\List A \ra  \List A} $ \\
$ \spc \spc \rev .A \; (\nil A ) = \nil A$\\
$ \spc \spc \rev .A \; (\cons A \;x  \; xs) = \cons A \; (\reva A \; x \; xs) (\revb A \; x \; xs)$\\
$\ \spc \fun \reva : \EPi{A }{\Set}{A  \ra \List A \ra A}$\\
$ \spc \spc \reva .A \; a \; (\nil A ) = a$ \\
$\spc \spc \reva .A \; a (\cons A \; x \; xs) = \reva A \; x \; xs$\\
$\spc \fun \revb : \EPi{A}{\Set}{ A \ra \List A \ra \List A} $\\
$\spc \spc \revb .A \; a \; (\nil A ) = \nil A $\\
$\spc \spc \revb .A \; a (\cons A\;  x \; xs) = \rev A (\cons A \; a \; (\rev A \; (\revb A \; x \; xs)))$

\caption{reversion on Lists}  
\end{figure}
\begin{figure} 
$\mutual $ \\
$\spc \fun \rev : \EPi{n}{\Nat }{\EPi{A}{\Set }{\vVec A \; n \ra \vVec A \; n}}$\\
$\spc \spc \rev .\zero \; .A  \; (\nil A) = \nil A $\\
$\spc \spc \rev .(\suc n) .\; A (\cons A \; n \; x \; xs) = $\\
$\spcx \cons A \; n \; (\reva n \; A \; x \; xs) (\revb n \; A \; x \; xs) $ \\
$\spc \fun \reva : \EPi{n}{\Nat}\EPi{A}{\Set}{A \ra Vec A \; n \ra A}$\\
$\spc \spc \reva .\zero  \; .A \; a \; (\nil A) = a $\\
$\spc \spc \reva .(\suc n) .A \; a \; (\cons A \; n \; x \; xs) = \reva n \; A \; x \; xs $\\
$\spc \fun \revb : \EPi{n}{\Nat}{\EPi{A}{\Set}{A \ra \vVec A \; n \ra \vVec A \; n}}$\\
$ \spc \spc \revb .\zero \; .A  \; a \; (\nil A) = \nil A $\\
$ \spc \spc \revb .(\suc n) \; .A \; a (\cons A \; n \; x \; xs) = $\\
$ \spcx \rev (\suc n) A (\cons A \; n \; a (\rev \; n A \; (\revb \; n A \; x \; xs))$
\caption{reversion on vectors}  
\end{figure}  

\chapter{Using the Size Type}
Traditionally, size annotation have been used in the typing rules.
We take a different approach.
Wie add a primitive size type, and then use the current termination checker to see that the size is 
decreasing.
\section{Syntax}

Expressions:
\begin{itemize}
\item
size type: $ \Size $ 
\item
size succesor: $\s e $ 
\item
infinity: $\infty$ 
\end{itemize}
Patterns:
\begin{itemize}
\item
size succesor pattern:$ \s p $ 
\end{itemize}
Note that we also have variables of type $\Size$.
As such, expressions of type size form the stage expression of Barthe et. al :  $ i , \s i $ and $ \infty $ .
\section{Sized types definition}
We now declare what a sized type is.
\begin{definition}
A datatype D is a sized type, if $ \Size \ra A $ is its type
For every constructor 
\begin{itemize}
\item
it begins with $(i: Size)$
\item
the result type is $D \dots (\s i) \dots $
\item
every rec. argument is of the form $ D \dots i \dots $
\end{itemize}
\end{definition}
\section{Matching against infinity}
The equation 
$[ s \infty = \infty $
needs to be taken into account when evaluating a function application.
So whhen the pattern $ s \; x $ is matches against $ \infty $ , $x$ will be bound to $ \infty $  
\section{Interpretation of the Size type}
One may see the size type as a coniductive data type - the dual to natural numbers, the conatural numbers.
\section{Examples}
\subsection{Sized Lists}
\begin{bsp}
$\data \List ( A : \Set ) : \Size \ra \Set $ \\
$\spc \nil : \EPi{i}{\Size}{ \List A \; (\s i)} $\\
$\spc \cons : \EPi{i}{\Size}{ A \ra \List A \; i \ra \List A \; (\s i)} $
\end{bsp}

What is the differnce between vectors and sized lists ?
The size index is only used for relativ comparisons, its not an exact number, but an upper bound.
This enables subtyping, and one can compromise by forgetting the exact size ($\infty$).
For example, splitting a vector of length n, we need to show that there exist j,k such that $j+k = n$.
With sized type, we don't need to be that exact.
\subsection{Sized natural numbers}
\begin{bsp}
$\data \SNat : \Size \ra \Set $ \\
$\zero : \EPi{i}{\Size}{ \SNat \; (\s i)} $\\
$\suc : \EPi{i}{\Size}{ \SNat \; i \ra \SNat \; (\s i)} $
\end{bsp}

\subsection{Great common divisor}

\subsection{Quicksort}
\begin{bsp}
$\data \ProdT ( + A : \Set) : \Set$\\
$\spc  \prd : A \ra A \ra \ProdT A$\\
$\fun \pra : ( A : \Set ) \ra \ProdT A\; \ra A$\\
$\spc \pra \inacc{A}\; (\prd A\; a b) = a$\\
$\fun \prb : ( A : \Set ) \ra \ProdT A\; \ra A$\\
$\spc \prb \inacc{A}\; (\prod A\; a b) = b$\\
$\fun \ite : (A : \Set ) \ra \Bool \ra A \ra A \ra A$\\
$\spc \ite A\; \tt x\; y\; = x$\\
$\spc \ite A\; \ff x\; y\; = y$\\
%%
$\fun \pivot : (i : \Size ) \ra (A : \Set ) \ra ( \lleq : A \ra A \ra \Bool )$\\ 
$\spcx	\ra A \ra \List A\; i\; \ra \ProdT (\List A\; i)$\\
$\spc\pivot \inacc{\s i}\;     \inacc{A}\; \lleq a\; (\nil A\; i) = \prd (\List A\; (\s i))\;(\nil A\; i)\;(\nil A\; i)$\\
$\spc \pivot \inacc{\s i}\;    \inacc{A}\; \lleq a\; (\cons A\; i x\; xs) =$\\ 
$\spcx     \ite (\ProdT (\List A\; (\s i)))\; (\lleq a\; x)$\\ 
$\spcx    (\prd (\List A\; (\s i))$\\
$\spcx \spc       (\pra (\List A\; i)\; (\pivot i\; A\; \lleq a\; xs)) $\\
$\spcx \spc	(\cons A\; i\; x\; (\prb (\List A\; i) \; (\pivot i\; A\; \lleq a\; xs))))$\\ 
$\spcx     (\prd (\List A\; (\s i))$\\
$\spcx	\spc (\cons A\; i\; x\; (\pra (\List A\; i)\; (\pivot i\; A\; \lleq a\; xs)))$\\
$\spcx  \spc (\prb (\List A\; i)\; (\pivot i\; A\; \lleq a\; xs)))$\\
%%
$ \fun \qsapp : (i : \Size ) \ra ( A : \Set ) \ra ( \lleq : A \ra A \ra \Bool )$\\ 
$\spcx	\ra \List A\; i \ra \List A\; \infty \ra \List A\; \infty$\\
$\spc \qsapp \inacc{\s i}\; \inacc{A}\; \lleq (\nil A\; i) \;      ys = ys;$\\
$\spc \qsapp \inacc{\s i}\; \inacc{A}\; \lleq (\cons A\; i\; x\; xs) \; ys = \qsapp i\; A\; \lleq$\\ 
%%
$\spcx	(\pra (\List A\; i) (\pivot i\; A\; \lleq x\; xs))$\\
%%    	
$\spcx	(\cons A\; \infty \; x\; $\\
$\spcx	\spc   (\qsapp i\; A\; \lleq (\prb (\List A\; i) \; (\pivot i\; A\; \lleq x\; xs))\; ys))$\\
%%
$\fun \quicksort : (i : \Size ) \ra (A : \Set ) \ra (\lleq : A \ra A \ra \Bool)$\\ 
$\spcx	\ra \List A\; i \ra \List A\; \infty$\\
$\spc \quicksort i\; A\; \lleq l = \qsapp i\; A\; \lleq l (\nil A\; \infty)$ 
\end{bsp}

\subsection{A higher order function}
While somewhat artifical, the following example is interesting for two reasons:
\begin{bsp}
$\fun \addWith : ((k : \Size ) \ra \SNat k \ra \SNat k ) \ra (i : \Size ) \ra (j : \Size ) $\\
$\spcx \spcx \ra  \SNat i \ra \SNat j \ra \SNat \infty $\\
$\spc \addWith f \; \inacc{\s i} \; j \; (\zero i) \; y = y $ \\
$\spc \addWith f \; \inacc{\s i} \; j \; (\suc i \; x) \; y = \suc \infty \; (\addWith f \; j \; i \; y \; (f \; i \; x))$
\end{bsp}
$\addWith$ resembles the permutating size add function we've discussed in the previous chapter.
But it has one additional parameter that is declared to be a size-preserving function.
This is one strength of sized types: Any function of type  $ ((k : \Size ) \ra \SNat k \ra \SNat k $ can now
be passed along to $\addWith$.
The second reason is that the size arguments are permutating. 
So this is really one example that plays to the strength of our system: both the size type and the size change principle are needed.
\subsection{Sized ordinal numbers}
Note that sized types are quite powerful. 
We can show the termination of ordinal addition with a decreasing size argument.
First, sized Brouwer Ordinals:
\begin{bsp}
$\data \Ord : \Set$ \\
$\spc \ozero : \Ord $\\
$\spc \olim : (\Nat \ra \Ord ) \ra \Ord $
\end{bsp}
Now we don't kneed Axiom 2 of the structual ordering to see that addition is terminating:
\begin{bsp}
$\fun \addOrd : \Ord \infty \ra \EPi{i}{\Size}{\Ord i \ra \Ord \infty }$\\
$\spc \addOrd x\; \inacc{\s i}\; (\ozero i) = \ozero i $\\
$\spc \addOrd x\; \inacc{\s i}\; (\olim f)  = \olim \infty ( \lam{y} \addOrd  x \; i \; (f \; y))  $
\end{bsp}

\section{Admissibility}
\subsection{Motivation}
The use of sizes needs to be constraint.
Consider for example the function definition

\begin{bsp}
$\fun \bad : \EPi{i}{\Size}{\Nat}$\\
$\spc \bad (\s i) = \bad i $ 
\end{bsp}
The termination checker will hapilly tell you that $\bad$ is terminating.
The issue here is that the size argument is not really bound to any sized data type.
But there are more subtile difficulties with using the size type.
The type of the function $\lop$
\begin{bsp}
$\EPi{i}{\Size}{\SNat i \ra \Maybe (\SNat (\s i)) \ra (Nat \ra \Maybe (\SNat i))} $ 
\end{bsp}
needs to be rejected by an admissibility checker.
\begin{figure}
$\data \Maybe (A : \Set ) : \Set$\\
$\spc  \nothing : \Maybe A$\\
$\spc  \just : A \ra\Maybe A$\\
$\const \Nat : \Set = \SNat \infty$\\
$\fun \shiftcase : (i : \Size) \ra\Maybe (\SNat (\s i)) \ra\Maybe (\SNat i)$\\
$\spc \shiftcase i (\nothing .(\SNat (\s i))) = \nothing (\SNat i)$\\
$\spc \shiftcase .i (\just .(\SNat (\s i)) (\zero i)) = \nothing (\SNat i)$\\
$\spc \shiftcase .i (\just .(\SNat (\s i)) (\suc i x)) = \just (\SNat i) x$\\
$\const \shift : (i : \Size) \ra(\Nat \ra\Maybe (\SNat (\s i))) \ra\Nat \ra\Maybe (\SNat i) =$\\ 
$\spcx \lam{i}{\lam{f}{\lam{n}{\shiftcase i (f (\suc \infty n))}}}$\\
$\const \inc : \Nat \ra\Maybe \Nat = \lam{n}{\just \Nat (\suc \infty n)}$\\
$\mutual$\\
$\spc \fun \lop : (i : \Size ) \ra\SNat i \ra(\Nat \ra\Maybe (\SNat i)) \ra\Bool$\\
$\spc \spc \lop .(\s i) (\zero i) f = \lopcase (\s i) (\zero i) f (f (\zero i))$\\
$\spc \spc \lop .(\s i) (\suc i n) f = \lop i n (\shift i f)$\\
$\spc \fun \lopcase : (i : \Size ) \ra\SNat i \ra(\Nat \ra\Maybe (\SNat i)) \ra\Maybe (\SNat i) \ra\Bool$\\
$\spc \spc \lopcase i       x f (\nothing .(\SNat i)) = \tt$\\
$\spc \spc \lopcase .(\s i)  x f (\just .(\SNat (\s i))  (\zero i)) = \tt$\\
$\spc \spc \lopcase .(\s i)  x f (\just .(\SNat (\s i)) (\suc i y)) = \lop i y (\shift i f) $\\
$\const diverge : Unit = \lop \infty\; (\zero \infty) \; \inc$\\
\caption{Loop example}
\label{loop}
\end{figure}

Otherwise, the non terminating term diverge can be constructed.

\subsection{Admissibility criterion}
Gimenez and Barthe deal with monotonicity criterions.
Abel and Pareto have worked on criterions based on continuity.

\newcommand{\mon}[3]{#1\vdash#2\mbox{ \textbf{mon} }#3}
\newcommand{\anti}[3]{#1\vdash#2\mbox{ \textbf{anti} }#3}
\newcommand{\indu}[3]{#1\vdash#2\mbox{ \textbf{ind} }#3}
\newcommand{\coind}[3]{#1\vdash#2\mbox{ \textbf{coind} }#3}
\newcommand{\admIndSize}[3]{#1\vdash#2\mbox{ \textbf{admIndSize} }#3}
\newcommand{\admCoSize}[3]{#1\vdash#2\mbox{ \textbf{admCoSize} }#3}
\newcommand{\admArg}[3]{#1\vdash#2\mbox{ \textbf{admArg} }#3}
\newcommand{\admIndRes}[3]{#1\vdash#2\mbox{ \textbf{admIndRes} }#3}
\newcommand{\admCoRes}[3]{#1\vdash#2\mbox{ \textbf{admCoRes} }#3}

\begin{figure}
\begin{gather*}
\nru{}
{\LeqVal{k}{\nF}{v}{\subst{(i,\s i)}{v}}}
{\mon{k}{v}{i}}
\qquad
\nru{}
{\LeqVal{k}{\nF}{\subst{(i,\s i)}{v}}{v}}
{\anti{k}{v}{i}}
\\[2ex]
\ndru{}
{\DD \mbox{ sized inductive type}}
{\EqVal{k}{N}{\wh s_i}{i}}
{\indu{k}{\DD p_1 \ldots p_n s_i v_1 \ldots v_m}{i}}
\qquad
\ndru{}
{\DD \mbox{ sized coinductive type}}
{\EqVal{k}{N}{\wh s_i}{i}}
{\coind{k}{\DD p_1 \ldots p_n s_i v_1 \ldots v_m}{i}}
\\[2ex]
\nru{}
{\anti{k}{v}{i} \; \mbox{ or } \indu{k}{v}{i}}
{\admArg{k}{v}{i}}
\\[2ex]
\nru{}
{\mon{k}{v}{i} \; \mbox{ or } \coind{k}{v}{i}}
{\admIndRes{k}{v}{i}}
\qquad
\nru{}
{\coind{k}{v}{i}}
{\admCoRes{k}{v}{i}}
\\[2ex]
\nru{}
{\admArg{k}{v_a}{i} 
\qquad
{\admIndSize{k+1}{\wh \cl{b}{\ins (x,k) \rho}}{i}}}
{\admIndSize{k}{\vpi x v_a \cl{b}{\rho}}{i}}
\qquad
\nru{}
{\admIndRes{k}{v}{i}}
{\admIndSize{k}{v}{i}}
\\[2ex]
\nru{}
{\admArg{k}{v_a}{i} 
\qquad
{\admCoSize{k+1}{\wh \cl{b}{\ins (x,k) \rho}}{i}}}
{\admCoSize{k}{\vpi x v_a \cl{b}{\rho}}{i}}
\qquad
\nru{}
{\admCoRes{k}{v}{i}}
{\admCoSize{k}{v}{i}}
\end{gather*}
\caption{Admissiblity}
\end{figure}

\begin{definition}
A Type $A$ is \emph{inductive in i} if it is of the form $ \DD \vec{p} \; i \: \vec{v} $ and $\DD$ is a sized inductive type
\end{definition}
\begin{definition}
A type $A$ is \emph{monotone in i} if $ \DLeqVal{A}{A[i \ra \s i ]}$ 
\end{definition}
\begin{definition}
A type $A$ is \emph{antitone in i} if $ \DLeqVal{A[i \ra \s i ]}{A}$ 
\end{definition}
Note that if a type is inductive in $i$, it is also monotone in $i$. 
Now we give a criterion for when a $\fun$ declaration is admissible.
\begin{definition}
A recursive function with type $ A_1 \ra \dots A_n \ra R $ is admissible, if
for every $A_j$ of the form $(i : \Size )$ , for $A_{j + 1} \ra \dots \ra R $ the following holds 
\begin{itemize}
\item
every argument $A_k$ is antitone or inductive in $i$
\item
the result $R$ is monotone in $i$  
\item
at least one argument $A_k$ is inductive in $i$
\end{itemize}
\end{definition}
We note that for the rejected example(Figure \ref{loop}) the argument \[\Maybe (\SNat (\s i)\] is not antitone in $i$, which rrenders the functions $\lop$ (and both $ \shiftcase$ and $\lopcase$) non-admissible.

\subsection{Size Pattern completness}
It is necessary to check that the clauses of a definition cover all possible cases.
This is quite complicated for inductive families and will not be described here.
But a special case is pattern completeness for the built-in $\Size$ type.
As we have seen, without this check we would accept non-terminating function definitions.
This is like the \emph{bottom-check} in the work of Pareto and Sabry.
The issue is: does the pattern \[ (\s i) \] cover all possible cases ? 

\subsection{Examples}
We first give an example to show why we limit the form of what a sized type is.
Had we definied the sized natural numbers this way:
\begin{bsp}
$\data \BadNat ( A : \Set ) : \Size \ra \Set $ \\
$\spc \zero : \EPi{i}{\Size}{ \BadNat A \; i} $\\
$\spc \suc : \EPi{i}{\Size}{\BadNat i \ra \BadNat \; (\s i)} $
\end{bsp}
If the systemn accepted this as a sized type, then the following function would be seen terminating and admissible, while obviously looping on $\zero$.
\begin{bsp}
$\fun \foo : \EPi{i}{\Size}{\BadNat i \ra \Empty}$\\
$\spc \foo \inacc{(\s i)} \; (\zero (\s i) = \foo i \; (\zero i)$\\
$\spc \foo \inacc{(\s i)}\; (\suc i x) = \foo i \; x$
\end{bsp}
In our system, the $\BadNat$ type above is not a sized type, so the function $\foo$ is not admissible because $\BadNat i$ is not inductive in $i$.\\
For the same reason, the following function definition is also not admissible:
\begin{bsp}
$\fun \foo : \EPi{i}{\Size}{\SNat (\s i) \ra \Empty}$\\
$\spc \foo \inacc{(\s i)} \; (\zero (\s i) = \foo i \; (\zero i)$\\
$\spc \foo \inacc{(\s i)}\; (\suc i x) = \foo i x; l$
\end{bsp}
While $\SNat$ is a sized type, $\SNat (\s i)$ is not inductive in $i$.
\section{Subtyping for size}
It would be quite nice to have subtyping on sized data types.
As we are only interested in size arguments getting smaller, 
In one call to the function, the size might decrease by one, in another clause, by two.
It is enough for termination to decrease by one.
For inductive sized types like $\List$ we want for every size $i$
\begin{bsp}
$\List A\; i$ is a subtype of $\List A \; (\s i)$
\end{bsp}
and
\begin{bsp}
$\List A\; i$ is a subtype of $\List A \; \infty$.
\end{bsp}
because for inductive types the size is an upper bound for its height.
For coinductive sized types like $\Stream$ we want for every size $i$
\begin{bsp}
$\Stream (\s i) $ is a subtype of $\Stream i$
\end{bsp}
and
\begin{bsp}
$\Stream \infty$ is a subtype of $\Stream i$.
\end{bsp}
because for coinductive types the size is a lower bound for its definedness.
There is natural order on size expressions.
We define a substage relation on size values:
\[\SzLeq{v_1}{v_2} \subseteq Val \times Val \]
Now we the subtype relation similar to the equality relation by two simulatemous judgements:
\[\LeqVal{f}{k}{v_1}{v_2} \subseteq \cF \times \cN \times \Val \times \Val \]
\[\FleqVal{f}{k}{v_1}{v_2} \subseteq \cF \times \cN \times\Val \times \Val \]
Of special interest are the rules:
\begin{itemize}
\item
$\leqappind$: subtyping for sized inductiv type 
\item
$\leqappco$: subtyping for sized coinductive type
\item
$\leqappd$: subtyping for other data types
\item
$\leqpi$: subtyping for function space. 
\end{itemize}
\begin{figure}
\begin{gather*}
\nru{\sleqinfty}
{}
{\SzLeq{v}{\infty}}
\qquad
\nru{\sleqgen}
{}
{\SzLeq{k}{k}}
\\[2ex]
\nru{\sleqsucci}
{\SzLeq{v_1}{v_2}}
{\SzLeq{\s v_1}{\s v_2}}
\qquad
\nru{\sleqsuccii}
{\SzLeq{v_1}{v_2}}
{\SzLeq{v_1}{\s v_2}}
\end{gather*}
\caption{Size comparison}
\end{figure}

\begin{figure}[p]
\begin{gather*}
%% Force L
\nru{\leqlf}
{\forc \; v_1 = u_1 \qquad \forc \; v_2 = \stuck 
\qquad
f \neq \rF \qquad \FleqVal{\lF}{k}{u_1}{v_2}}
{ \LeqVal{f}{k}{v_1}{v_2}}
%% Force R
\\[2ex]
\nru{\leqrf}
{\forc \; v_1 = \stuck \qquad \forc \; v_2 = u_2 
\qquad
f \neq \lF \qquad \FleqVal{\rF}{k}{v_1}{u_2}}
{ \LeqVal{f}{k}{v_1}{v_2}}
\\[2ex]
%% Force N
\nru{\leqnf}
{\FleqVal{f}{k}{v_1}{v_2}}
{ \LeqVal{f}{k}{v_1}{v_2}}
\qquad
%% Set
\nru{\leqset}
{}
{\FleqVal{f}{k}{\Set}{\Set}}
%% Size
\\[2ex]
\nru{\leqsize}
{}
{\FleqVal{f}{k}{\Size}{\Size}}
%% Infty
\qquad
\nru{\leqinfty}
{}
{\FleqVal{f}{k}{\infty}{\infty}}
%% Succ
\\[2ex]
\nru{\leqsuc}
{\LeqVal{f}{k}{v_1}{v_2}}
{\FleqVal{f}{k}{\s v_1}{\s v_2}}
\qquad
%% Gen
\nru{\leqgen}
{}
{\FleqVal{f}{k}{k'}{k'}}
\\[2ex]
%% Ind Data
\ndru{\leqappind}
{\forall i \in Pos: \LeqVal{f}{k}{p_i}{q_i} 
\qquad
\forall j \notin Pos : \EqVal{f}{k}{p_j}{q_j} 
}
{\forall k \in \{1,..m\}:\EqVal{f}{k}{v_k}{w_k}
\qquad
\SzLeq s t }
{\FleqVal{f}{k}{\DD p_1 \ldots p_n s v_1 \ldots v_m}{\DD q_1 \ldots q_n t w_1 \ldots w_m}}
%% Co Data
\\[2ex]
\ndru{\leqappco}
{\forall i \in Pos: \LeqVal{f}{k}{p_i}{q_i} 
\qquad
\forall j \notin Pos : \EqVal{f}{k}{p_j}{q_j}}
{\forall k \in \{1,..m\}:\EqVal{f}{k}{v_k}{w_k}
\qquad
\SzLeq t s }
{\FleqVal{f}{k}{\DD p_1 \ldots p_n s v_1 \ldots v_m}{\DD q_1 \ldots q_n s w_1 \ldots w_m}}
%% Data
\\[2ex]
\ndru{\leqappd}
{\forall i \in Pos: \LeqVal{f}{k}{p_i}{q_i} 
\qquad
\forall j \notin Pos : \EqVal{f}{k}{p_j}{q_j}}
{\forall k \in \{1,..m\}:\EqVal{f}{k}{v_k}{w_k}}
{\FleqVal{f}{k}{\DD p_1 \ldots p_n v_1 \ldots v_m}{\DD q_1 \ldots q_n w_1 \ldots w_m}}
%%App
\\[2ex]
\nru{\leqapp}
{\LeqVal{f}{k}{v_1}{v_2} \qquad
\EqVal{f}{k}{c_1}{d_1} \ldots \EqVal{f}{k}{c_n}{d_n}}
{\FleqVal{f}{k}{v_1 (c_1 \ldots c_n)}{v_2 (d_1 \ldots d_n)}}
%% Pi
\\[2ex]
\nru{\leqpi}
{\LeqVal{f}{k}{a_2}{a_1}
\qquad
\LeqVal{f}{k+1}{\wh b_1 (\ins \; \rho_1 \; (x_1,k))}{\wh b_2 (\ins \; \rho_2 \; (x_2,k))}}
{\FleqVal{f}{k}{\vpi \; x_1 \; a_1 \; \cl{b_1}{\rho_1}}{\vpi \; x_2 \; a_2 \; \cl{b_2}{\rho_2}}}
%% Lam
\\[2ex]
\nru{\leqlam}
{\LeqVal{f}{k+1}{\wh b_1 (\ins \; \rho_1 \; (x_1,k))}{\wh b_2 (\ins \; \rho_2 \; (x_2,k))}}
{\FleqVal{f}{k}{\vlam \; x_1 \; \cl{b_1}{\rho_1}}{\vlam \; x_2 \; (\cl{b_2}{\rho_2}}}
%% Def
\\[2ex]
\nru{\leqdef}
{}
{\FleqVal{f}{k}{\ff}{\ff}}
%% Con
\qquad
\nru{\leqcon}
{}
{\FleqVal{f}{k}{\cc}{\cc}}
\end{gather*}
\caption{subtype checking}
\end{figure}

\subsection{On the neccesity of subtyping}
We note that one could possible live without subtyping in our system.
For example 
\begin{bsp}
$\fun \wkSNat : (i : \Size ) \ra \SNat i \ra \SNat (\s i)$\\
$\spc \wkSNat \inacc{\s i} \; (\zero i) = \zero (\s i)$\\
$\spc \wkSNat \inacc{\s i} \; (\suc i \; x) = \suc (\s i) \; (\wkSNat i \; x) $
\end{bsp}
could be used as a manual type casting function, altough it is not very efficient at run-time.
\chapter{Adding coinductive types}

Coinductive types allow the introduction of (possibly) infinite objects in type theory.
The analogon to list are colists. This type includes all finite and infinite lists.
The list of all natural numbers is an example of an infinite colist.
While we can't look at all natural numbers in a finite time, we should expect to be able to
compute any finite part of a colist in finite time.
Functions that define corecursive objects are corecustive functions.
This is possible for \emph{productive} coinductive definitions.
There are syntactic criteria for checking productivity of coinductive definitons.
But in the following, we will use the size type introduced earlier to prove termination.

\begin{itemize}
\item
coinductive datatype:
\[\codata \; \DD \; \tau : t \; \vec{\gamma}\]  

\item
coinductive function:
\[\cofun \; \ff \; : \; t \; \vec{\kappa}\]
\end{itemize}

\section{Syntax}

\section{Lazy values}
Unrolling of corecursive functions needs to be restricted.
Force is used in specific places.

\section{Examples}

\subsection{Streams}
Now we introduce streams of natural numbers.
\begin{bsp}
$\codata Stream : \Set $ \\
$\spc \cons : \Nat \ra \Stream \ra \Stream $
\end{bsp}
Note that because $\Stream$ has only one constructor, there are no finite $\Stream$ objects.
A stream of zeroes can be generated by:
\begin{bsp}
$\cofun \zeroes: \Stream $ \\
$\spc \zeroes = : \cons \zero \zeroes$
\end{bsp}

An example of an unproductive stream is $\unp$:
\begin{bsp}
$\cofun \unp : \Stream $ \\
$\spc \unp = : \unp$
\end{bsp}
We can look at any element of $\zeroes$ .
But looking at a element $\unp$ is not defined. 
A syntactic criterion for productivity of coinductive definitions is the guardness condition:
Every (co)recursive call must be guarded by a constructor.
In the above example , the call to $\zeroes$ is guarded by $\cons$
But we want to use sized types for productivity checking.
\subsection{Sized Streams}
Let us now explain how sized types , in comnbination with the current terminatin criterion, can be used for showing productivity.
First we introduce the sized coinductive type 
\begin{bsp}
$\codata \Stream : \Size \ra \Set $ \\
$\spc \cons : \EPi{i}{\Size}{\Nat \ra \Stream i \; \ra \Stream (\s i) } $
\end{bsp}
For the productive stream $\zeroes$, we turn it into
\begin{bsp}
$\cofun \zeroes: \EPi{i}{Size}{\Stream i} $ \\
$\spc \zeroes (\s i) = \cons i \zero (\zeroes i)$\\
\end{bsp}
The termination checker sees that the recursive call happens on a smaller size.
For the unproductive stream, we have two bad choices:
The first one
\begin{bsp}
$\cofun \unp : \EPi{i}{\Size}{\Stream i}$ \\
$\spc \unp i =  \unp i$
\end{bsp}
is type correct, but the call is not on a smaller size.
The second option is
\begin{bsp}
$\cofun \unp : \EPi{i}{\Size}{\Stream i}$ \\
$\spc \unp (\s i) = \unp i $
\end{bsp}
is accepted by the termination checker, but thankfully is not type correct.
\subsection{Fibonacci stream}

Helper functions head, tail:
\begin{bsp}
$\fun \tail : \EPi{i}{\Size}\Stream (\s i) \ra \Stream i$\\
$\spc \tail \; .i \; (\cons A \; i \; n \; ns) = ns$\\
$\fun \head : \EPi{i}{\Size}\Stream (\s i) \ra \Nat$\\
$\spc \head \; .i \; (\cons A \; i \; n \; ns) = n$
\end{bsp}
We can now define looking up the nth element of a Stream.
\begin{bsp}
$\fun nth : \Nat \ra \Stream \infty \ra \Nat$\\
$\nth \zero ns = \head \infty ns$\\
$\nth (\suc n) \; ns = nth n (\tail \infty ns)$\\
\end{bsp}
Then, we introduce the helper function $\zipWith$:
\begin{bsp}
$\cofun \zipWith : (\Nat \ra \Nat ) \ra \EPi{i}{\Size}{\Stream i \ra \Stream i \ra \Stream i}$ \\
$\spc \zipWith f \; (\s i) \; as bs = \cons i \; (f \; a \; b) (\zipWith f \; i \; (\tail i \; as) (\tail i \; bs)$
\end{bsp}
Finally we can define the stream of all fibonacci streams:
\begin{bsp}
$\cofun \fib : \EPi{i}{\Size}\Stream i$\\
$\spc \fib (\s \s i) = \cons (\s i) \; \zero (\cons i \; (\suc \zero)$\\
$\spcx (\zipWith \add \; i \; (\fib i) \; (\tail i (\fib (\s i)))))$
\end{bsp}
and get the fourth fibonacci number by:
\begin{bsp}
$\const \fibf : \Nat = \nth (\suc (\suc (\suc (\suc \zero )))) \; (\fib \infty)$
\end{bsp}
\subsection{Stream processors}

\begin{bsp}
$\data \ispd (+ K : \Set ) : \Set$\\
$\spc\putd : \Nat \ra K \ra \ispd K$\\
$\spc \getd : (\Nat \ra \ispd K) \ra \ispd K$\\ 
$\codata \spd : \Size \ra \Set$\\
$\spc \isp : (i : \Size ) \ra \ispd (\spd i) \ra \spd (\s i)$\\
\end{bsp}
Note that this is not a mutual definition, and that $\isp$ is sized type

We can now define $\eat$ for the execution of a stream processor.
\begin{bsp}
$\mutual $\\
$\spc \cofun \eat : (i : \Size ) \ra \spd i \ra \Stream \infty \ra \Stream i$ \\
$\spc \spc \eat \inacc{\s i} \; (\isp i \; sp) \; as = \ieat i \; sp \; as $ \\
$\spc \fun \ieat : (i : \Size ) \ra \ispd (\spd i) \ra \Stream \infty \ra \Stream ($ i$)$\\
$\spc \spc \ieat i\; (\getd \inacc{\spd i} \; f) \; (\cons \inacc{\infty} \; a \; as) = \ieat i \; (f a) \; as$\\
$\spc \spc \ieat i\; (\putd \inacc{\spd i} \; b \; sp) \; as  = \cons i \; b \; (\eat i \; sp \; as) $\\
\end{bsp}
$\ieat$ is structurally recursive on the $\ispd$ argument.
$\eat$ is shown productive because the size argument is admissible.
\subsection{Stream equality}
The following is a valid definition:
\begin{bsp}
$\const \eqo : \Eq (\Stream \infty) (\zeroes \infty) (\cons \infty \zero (zeroes \infty))$\\
$\spc = \refl (\Stream \infty) (zeroes \infty)$
\end{bsp}
The infered type of $\refl (\Stream \infty) \; (\zeroes \infty)$ is 
\[\Eq (\Stream \infty) \; (\zeroes \infty) \; (\zeroes \infty)\]
The type checker has to unroll $(\zeroes \infty)$ once to see that the type is equal to 
\[
\Eq (\Stream \infty) \; (\zeroes \infty) \; (\cons \infty \; \zero (\zeroes \infty)
\]
Now let's define another stream
\begin{bsp}
$\cofun \zeroest: \EPi{i}{Size}{\Stream i} $ \\
$\spc \zeroest (\s \s i) = \cons (\s i) \zero (\cons i \zero (\zeroest i))$\\
\end{bsp}
We cannot prove \[\Eq (\Stream \infty) (\zeroes \infty)) (\zeroest \infty)\]
because the type checker can't unroll both $\zeroes \infty)$ and $\zeroest \infty$ as
this would lead to no progress.
But we can define coinductive equality  on streams \cite{coquand-infinite}
\begin{bsp}
$\codata \beqd : \Size \ra \Stream \infty \ra \Stream \infty \ra \Set$\\
$ \spcx \beq : (i : \Size ) \ra ( n : \Nat ) \ra (s_1 : \Stream \infty ) \ra (s_2 : \Stream \infty )$\\
$ \spcx \spc  \ra \beqd i \; s_1 \; s_2 \ra \beqd (\cons n \; s_1) \; (\cons n \; s_2)$
\end{bsp}
Note that $\beqd$ is a sized coinductive type.
and now the following is a valid infinite proof:
\begin{bsp}
$\cofun \eqt : (i : \Size ) -> \beqd i (\zeroes \infty) (\zeroest \infty)$\\
$\spc \eqt (\s \s i) = $\\
$\spc \spcx \beq (\s i) \; \zero (\cons \infty \; \zero (\zeroes \infty)) \; (\cons \infty \zero (\zeroest \infty))$\\ 
$\spc \spcx \spc (\beq i \; \zero (\zeroes \infty) \; (\zeroest \infty) \; (\eqt i))$
\end{bsp}
\section{Admissibility}
We now have to define when a $\cofun$ declaration is admissible.
\begin{definition}
A corecursive function with type $ A_1 \ra \dots A_n \ra R $ is admissible, if
for every $A_j$ of the form $(i : \Size )$ , for $A_{j + 1} \ra \dots \ra R $ the following holds 
\begin{itemize}
\item
every argument $A_k$ is antitone or inductive in $i$
\item
the result $R$ is coinductive in $i$  
\end{itemize}
\end{definition}
It follows from the definition that one can only use on size argument, because it has to be coinductive in the result type.
\chapter{Implementation}
Mugda was implemented in the function language Haskell \cite{haskell}.
\section{Overview}
\begin{itemize}
\item
\texttt{Lexer.x} : The alex lexer file
\item
\texttt{Parser.y} : The happy parser file
\item
\texttt{Concrete.hs} : Concrete syntax produced by the parser
\item
\texttt{ScopeChecker.hs} : turns concrete into abstract syntax
\item
\texttt{Abstract.hs} : produced by the scope checker
\item
\texttt{Values.hs} : Values and related functions: eval, whnf 
\item
\texttt{TypeChecker.hs} : TypeChecking with Admissiblity checking
\item
\texttt{Termination.hs} : structural Termination checker
\item
\texttt{Completness} : size pattern completness check
\item
\texttt{SPos.hs} : strict positivity checker
\texttt{Main.hs} : the main module
\item
\text{example} dir: example input files
\end{itemize}

\section{Usage}

\begin{itemize}
\item
lists of constructors and clauses are grouped with brackets \texttt{\{},\texttt{\}} and separated with semicolon \texttt{;}
\item
$\EPi{x}{A}{B}$ is written \texttt{(x : A) -> B }
\item
$A \ra B$ is written \texttt{A -> B}
\item
$ \lam{x}e$ is written \texttt{\\ x -> e }
\item
$ \inacc{e}$ is written \texttt {.e} 
\item
$ \infty$ is written \texttt{\#} 
\item
$\s$ is written \texttt{\$}
\end{itemize}
\texttt{const} can be replaced with {\texttt {eval const}}. Then the value will be evaluated after type checking is done.
As an example showing all syntactial features, here is the fibonacci stream example (\texttt{examples/fib.ma}) in ASCII format:
\begin{verbatim}
data Nat : Set {
  zero : Nat;
  succ : Nat -> Nat 
}

fun add : Nat -> Nat -> Nat {
  add zero = \y -> y;
  add (succ x) = \y -> succ (add x y)
}

codata Stream : Size -> Set {
  cons : (i : Size) -> Nat -> Stream i -> Stream ($ i)
}
 
fun tail : (i : Size) -> Stream ($ i) -> Stream i {
  tail .i (cons i n ns) = ns
}

fun head : (i : Size) -> Stream ($ i) -> Nat {
  head .i (cons i n ns) = n
}

cofun zipWith :  (Nat -> Nat -> Nat ) -> ( i : Size ) 
		-> Stream i -> Stream i -> Stream i {
  zipWith f ($ i) as bs = 
	cons i (f (head i as) (head i bs))  (zipWith f i (tail i as) (tail i bs)) 
}

fun nth : Nat -> Stream # -> Nat {
  nth zero ns = head # ns;
  nth (succ x) ns = nth x (tail # ns) 
}

cofun fibs : ( i : Size ) -> Stream i
{
  fibs ($ $ i) = cons ($ i) zero (cons i (succ zero)
	 	(zipWith add i (fibs i) (tail i (fibs ($ i)))))
}

const 4 : Nat = (succ (succ (succ (succ zero))))
-- fib(4) = 3 
eval const fib4 : Nat = nth 4 (fibs ) 

\end{verbatim}
Running \texttt{Main examples/fib.ma} yields the console output:
\begin{verbatim}
./Main examples/fib.ma
***** Mugda v1.0 *****
--- scope checking ---
--- termination checking ---
--- type checking ---
Stream is a sized type 
--- evaluating constants ---
fib4 evaluates to (succ (succ (succ zero)))

\end{verbatim}

\section{Parsing}
The alex \cite{alex} and happy \cite{happy} tools where used for generating lexer and parser.

The checkers use monad transformers (\cite{Grabmueller2006MonadTransformers}).

\section{Scope checking}
The first step after parsing is scope checking. This checks that names are used correctly:
all declared constants should have a uniquie name and variables bound by 
The use of linear patterns is also restricted at scope checking.
Thus, all used identifiers can be categorized as refering to a constructor, a variable or a definied data type or
function.
After scope checking, it is guaranteed that looking up in signature or environments is guaranteed to succeed.
\subsection{Type Checking}
Type checking was implemented as explained above.
\subsection{Termination Checking}
Termination checking was implemented as explained.


%%% Local Variables: 
%%% mode: latex
%%% TeX-master: "da"
%%% End: 
