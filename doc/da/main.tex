\chapter{Mugda : Syntax and Semantics}

% Inference rules
\newcommand{\rulename}[1]{\ensuremath{\mbox{\sc#1}}}
\newcommand{\ru}{\dfrac}
\newcommand{\rux}[3]{\ru{#1}{#2}\ #3}
\newcommand{\nru}[3]{#1\ \ru{#2}{#3}}
\newcommand{\nrux}[4]{#1\ \ru{#2}{#3}\ #4}
\newcommand{\dstack}[2]{\begin{array}[b]{c}#1\\#2\end{array}}
\newcommand{\xstack}[3]{\begin{array}[b]{c}#1\\#2\\#3\end{array}}
\newcommand{\ndru}[4]{#1\ \ru{\dstack{#2}{#3}}{#4}}
\newcommand{\nxru}[5]{#1\ \ru{\xstack{#2}{#3}{#4}}{#5}}
\newcommand{\ndrux}[5]{#1\ \ru{\dstack{#2}{#3}}{#4}\ #5}


% rule names
\newcommand{\infvar}{\rulename{inf-var}}
\newcommand{\infset}{\rulename{inf-set}}
\newcommand{\infsize}{\rulename{inf-size}}
\newcommand{\infinfty}{\rulename{inf-infty}}
\newcommand{\infappa}{\rulename{inf-app-i}}
\newcommand{\infappb}{\rulename{inf-app-ii}}
\newcommand{\infcon}{\rulename{inf-con}}
\newcommand{\infdef}{\rulename{inf-fun}}
\newcommand{\inflet}{\rulename{inf-let}}
\newcommand{\infdata}{\rulename{inf-data}}

\newcommand{\chklet}{\rulename{chk-let}}
\newcommand{\chklam}{\rulename{chk-lam}}
\newcommand{\chkpi}{\rulename{chk-pi}}
\newcommand{\chksucc}{\rulename{chk-succ}}
\newcommand{\chkinf}{\rulename{chk-inf}}
\newcommand{\chkinfs}{\rulename{chk-inf-sub}}

\newcommand{\eqforce}{\rulename{eq-force}}

\newcommand{\eqsuc}{\rulename{eq-succ}}
\newcommand{\eqapp}{\rulename{eq-app}}
\newcommand{\eqpi}{\rulename{eq-pi}}
\newcommand{\eqlam}{\rulename{eq-lam}}
\newcommand{\eqatom}{\rulename{eq-atom}}

\newcommand{\chkpvar}{\rulename{chkp-var}}
\newcommand{\chkpcon}{\rulename{chkp-con}}
\newcommand{\chkpsucc}{\rulename{chkp-succ}}
\newcommand{\chkpinacc}{\rulename{chkp-inacc}}

\newcommand{\unifyl}{\rulename{inst-flex-l}}
\newcommand{\unifyr}{\rulename{inst-flex-r}}
\newcommand{\unifyc}{\rulename{inst-con}}
\newcommand{\unifyd}{\rulename{inst-data}}
\newcommand{\unifye}{\rulename{inst-eq}}
\newcommand{\unifys}{\rulename{inst-succ}}
\newcommand{\unifyi}{\rulename{inst-infty}}

\newcommand{\leqforce}{\rulename{leq-force}}

\newcommand{\leqsuc}{\rulename{leq-succ}}
\newcommand{\leqappco}{\rulename{leq-co}}
\newcommand{\leqappind}{\rulename{leq-ind}}
\newcommand{\leqappd}{\rulename{leq-data}}
\newcommand{\leqapp}{\rulename{leq-app}}
\newcommand{\leqpi}{\rulename{leq-pi}}
\newcommand{\leqlam}{\rulename{leq-lam}}
\newcommand{\leqatom}{\rulename{leq-atom}}

\newcommand{\sleqinfty}{\rulename{sleq-infty}}

\newcommand{\sleqsucci}{\rulename{sleq-succ-i}}
\newcommand{\sleqsuccii}{\rulename{sleq-succ-ii}}
\newcommand{\sleqgen}{\rulename{sleq-gen}}

\newcommand{\dd}{\mathsf{l}}
\newcommand{\DD}{\mathsf{D}}
\newcommand{\cc}{\mathsf{c}}
\newcommand{\ff}{\mathsf{f}}
\newcommand{\fg}{\mathsf{g}}

\newcommand{\cDD}{\mathbb{D}}
\newcommand{\cdd}{\mathbb{L}}
\newcommand{\ccc}{\mathbb{C}}
\newcommand{\cff}{\mathbb{F}}
\newcommand{\cV}{\mathbb{V}}

\newcommand{\cS}{\mathbb{S}}
\newcommand{\cN}{\mathbb{N}}
\newcommand{\cB}{\mathbb{B}}
\newcommand{\cI}{\mathbb{I}}
\newcommand{\cF}{\mathbb{F}}
\newcommand{\nF}{\mathbf{N}}
\newcommand{\lF}{\mathbf{L}}
\newcommand{\rF}{\mathbf{R}}

\newcommand{\opn}[1]{\operatorname{#1}}

\newcommand{\Expr}{\opn{EXPR}}
\newcommand{\Val}{\opn{VAL}}
\newcommand{\AVal}{\opn{AVAL}}
\newcommand{\vlam}{\mathsf{Lam}}
\newcommand{\vpi}{\mathsf{Pi}}
\newcommand{\cl}[2]{#1^{#2}}
\newcommand{\clp}[2]{\cl{(#1)}{#2}}
\newcommand{\Env}{\opn{ENV}}
\newcommand{\wh}{\searrow}
\newcommand{\ap}{\opn{app}}
\newcommand{\apf}{\opn{app_{fun}}}
\newcommand{\sinfty}{\opn{s_{\infty}}}
\newcommand{\lkup}{\opn{lkup}}
\newcommand{\ins}[3]{#1,#2=#3}
\newcommand{\forc}{\opn{force}}
\newcommand{\matchcls}{\opn{match_{cls}}}
\newcommand{\matchcl}{\opn{match_{cl}}}
\newcommand{\match}{\opn{match}}
\newcommand{\matchf}{\opn{match_f}}
\newcommand{\matchList}{\opn{match_{list}}}
\newcommand{\matchInfty}{\opn{match_{\infty}}}
\newcommand{\stuck}{\Uparrow}
\newcommand{\Clause}{\opn{CLAUSE}}
\newcommand{\Decl}{\opn{DECL}}
\newcommand{\Pat}{\opn{PAT}}

\newcommand{\Sig}{\Sigma}
\newcommand{\esig}{\Sigma_0}

\section{Preliminaries}
\newcommand{\pset}[1]{\mathcal{P}(#1)}

First we define some common mathematical sets:
\paragraph*{Natural numbers}
$ \cN = \{ 0 , 1 , ... \} $ is the set of natural numbers.
\paragraph*{Booleans}
$ \cB = \{ \top, \bot \} $ is the set of Boolean values.
\paragraph*{Power Set}
For some set $B$, $\pset{B}$ is the power set of $B$.
\paragraph*{Sequences}
For some set $B$, $\ves{B}$ is the set of finite sequences over $B$.
The empty sequence is written as $\diamond$.
Sequences are written as $\vec{b}$ or $(b_1 \ldots b_n)$.
The length of a sequence is defined as $\vert(b_1 \ldots b_n)\vert := n $. 
\section{Syntax of Mugda}

\subsubsection{Identifiers}
We assume the following disjoint sets of identifiers:
\[
\begin{array}{ll}
\ccc \ni \cc & \mbox{for constructors}\\
\cDD \ni \DD & \mbox{for data types}\\
\cff \ni \ff , \fg & \mbox{for functions}\\
\cdd \ni \dd & \mbox{for global lets}\\
\cV \ni x,y,i & \mbox{for variables}
\end{array}
\]
Elements of $\ccc, \cDD , \cff$ and $\cdd$ will be called \emph{constants}.


\subsubsection{Expressions}

\[
\begin{array}{lrll}
\Expr \ni e,A,B & ::= & \lam{x}{e} & \mbox{abstraction}\\
& \mid & \EPi{x}{A}{B} & \mbox{dependent function type}\\
& \mid & e \; e_1 \ldots e_n & \mbox{application}\\
& \mid & \ELet{x}{e_1}{A_1}{e_2} & \mbox{local let}\\ 
& \mid & \Set & \mbox{universe of small types}\\
& \mid & x & \mbox{variable}\\
& \mid & \cc & \mbox{constructor name}\\
& \mid & \DD & \mbox{data type name}\\
& \mid & \ff & \mbox{function name}\\
& \mid & \dd & \mbox{let name}
\end{array}
\]
Expressions are the unified syntax of types and terms.

\subsubsection{Patterns}
\[
\begin{array}{lrll}
\Pat \ni p & ::= & x & \mbox{variable pattern}\\
& \mid & \cc \: \vec{p} & \mbox{constructor pattern}\\
& \mid & \inacc{e} & \mbox{inaccessible pattern}\\
\end{array}
\]

\subsubsection{Telescopes}
\[
\begin{array}{lrll}
\tau & ::= & \diamond & \mbox{empty telescope}\\
& \mid & ( x : A ) \: \tau & \mbox{parameter}\\
& \mid & ( + \: x : A ) \: \tau & \mbox{strictly positive parameter}
\end{array}
\]

\subsubsection{Constructor definitions}
\[
\begin{array}{lrll}
\gamma &  ::= & \cc \: : \: A & 
\end{array}
\]

\subsubsection{Clause definitions}
\[
\begin{array}{lrll}
\Clause \ni \kappa & ::= & \ff \: \vec{p} = e & 
\end{array}
\]

\noindent For a clause $\ff \: \vec{p} = e$, $\vec{p}$ is called the right hand side and $e$ the left hand side. 

\subsubsection{Recursive function definitions}
\[
\begin{array}{lrll}
\mu & ::= & \fun \ff : A \: \vec{\kappa} & 
\end{array}
\]

\subsubsection{Corecursive function definitions}
\[
\begin{array}{lrll}
\nu & ::= & \cofun \ff : A \: \vec{\kappa} & 
\end{array}
\]

\subsubsection{Declarations}
\[
\begin{array}{lrll}
\Decl \ni \delta & ::= & \data \DD \: \tau : A \: \vec{\gamma} & \mbox{inductive data type}\\ 
& \mid & \codata \DD \: \tau : A \: \vec{\gamma} & \mbox{coinductive data type} \\
& \mid & \clet \dd : A \: = e & \mbox{global let}\\
& \mid & \mutual \vec{\mu} & \mbox{mutual fun }\\
& \mid & \mutual \vec{\nu} & \mbox{mutual cofun}
\end{array}
\]


\subsubsection{Mugda program}
A $\mugda$ program is a list of declarations $\vec{\delta}$.

\subsubsection{Syntactic sugar}
\begin{itemize}
\item
$A \ra B$ is short for $\EPi{x}{A}{B}$ where $x$ is some variable not occurring in $B$.
\item
In the examples, we write $\mutual \fun \ff : A \: \vec{\kappa} $ as $\fun \ff : A \: \vec{\kappa}\:$.
\item
If it is clear from the context, $\diamond$ is dropped. For example, $\cc$ is short for the pattern $\cc \: \diamond$ when $\cc$ is a nullary constructor.
\end{itemize}

\subsubsection{Prepend Telescope to expression}
Ignoring the $+$ annotations, for a telescope $\tau = (x_1 : A_1) \ldots (x_n : A_n)$ we write $\tau \ra B$ for the expression $\EPi{x_1}{A_1}{\ldots \ra \EPi{x_n}{A_n}{B}}$.
For an empty telescope $\tau = \diamond$, $\tau \ra B$ is meant to be the expression $B$.
Accordingly, we write $\Gamma \ra B$ for some expression $\EPi{x_1}{A_1}{\ldots \ra \EPi{x_n}{A_n}{B}}$.


\section{Semantics of Mugda expressions}
Computation is already needed during type-checking for dependent types.
Thus, we first need provide an evaluation that is used during type-checking.

\subsection{Values}

The semantics of $\mugda$ contain the main ingredients of the type-checking algorithm of \cite{coquand96algorithm}: Closures and generic values.
Closures, already introduced in \cite{Landin63}, are used for explicit substitution.
We simultaneously define values and environments:

\subsubsection{Values}
\[
\begin{array}{lrll}
\Val \ni v & ::= & v \: \vec{v} & \mbox{application}\\
& \mid & \vlam \: x \: \cl{e}{\rho} & \mbox{abstraction}\\
& \mid & \vpi \: x \: v \: \cl{e}{\rho} & \mbox{dependent function space}\\
& \mid & a & \mbox{atomic value}\\
\vs\\
\AVal \ni a & ::= & k & \mbox{generic value}\\
& \mid & \Set & \mbox{universe of small types} \\
& \mid & \cc & \mbox{constructor name}\\
& \mid & \ff & \mbox{function name}\\
& \mid & \DD & \mbox{data name}\\
\end{array}
\]
A generic value $k \in \cN$ represents the computed value of a variable during type-checking.

\subsubsection{Environments}
\[
\begin{array}{lrll}
\Env \ni \rho & ::= & \diamond & \mbox{empty environment}\\
& \mid & \rho,x = v & \mbox{extended with binding}\\
\end{array}
\]

A closure $\cl{e}{\rho}$ is a pair of an expression $e$ and an environment $\rho$.
The environment provides bindings for the free variables occurring in $e$.
Values can be seen as partially evaluated expressions that may contain closures.

\subsection{Signatures}

A signature carries information about all declared constants.
Each declaration adds newly defined constants to the signature, which is empty at the beginning.
\subsubsection{Signature}
A signature $\Sig$ is defined as a partial polymorphic mapping:
\begin{itemize}
\item
$ \Sig : \cff \ra \Val \times \: \ves{\Clause} \times \cB $\\
mapping a function constant to its type (as a value), the clauses and a flag to indicate whether the clauses have been type-checked.
\item
$ \Sig : \cdd \ra \Expr \times \Val $\\
mapping a global let constant to the expression and its type. 
\item
$ \Sig : \ccc \ra \Val $\\
mapping a constructor constant to its type.
\item
$ \Sig : \cDD \ra \Val \times \: \cN $\\
mapping a data type constant to its type and the number of parameters.
\end{itemize}

\noindent The empty signature $\esig$ is undefined on all arguments.



\subsection{Evaluation}

Now the evaluation of a closure $\cl{e}{\rho}$ can be defined. 
A closed expression can be evaluated in an empty environment.

We simultaneously define evaluation as a function $\wh$ along with some helper functions.    
Most of these functions are partial because only type-checked expressions should be evaluated.
A fixed signature $\Sig$ is assumed.

\subsubsection{Evaluation}

\[
\begin{array}{ll}
\multicolumn{2}{l}{\wh \: : \Expr \times \Env \ra \Val}\\
\wh \clp{\lam{x}{e}}{\rho} & = \vlam \: x \: \cl{e}{\rho} \\
\wh \clp{\EPi{x}{A}{B}}{\rho} &= \vpi \: x \: v_A \: \cl{B}{\rho} \mbox{ where } v_A = \wh \cl{A}{\rho} \\
\wh \clp{\ELet{x}{A}{e_1}{e_2}}{\rho} &= \wh \cl{e_2}{\ins{\rho}{x}{v_1}} \mbox { where } v_1 = \wh \cl{e_1}{\rho} \\
\wh \clp{e \: e_1 \ldots e_n}{\rho} &= \ap \: v \: v_1 \ldots v_n \mbox{ where } v = \wh \cl{e}{\rho} , v_i = \wh \cl{e_i}{\rho}\\
\wh \cl{\Set}{\rho} & = \Set \\ 
\wh \cl{\cc}{\rho} & = \cc  \\
\wh \cl{\ff}{\rho} & = \ff \\
\wh \cl{\dd}{\rho} & = \wh \cl{e}{\diamond} \mbox{ where } \: \Sig \: \dd = (e,v_t)\\
\wh \cl{x}{\rho} & = \lkup \: \rho \: x \\
\end{array}
\]
The closures in $\vlam \: x \: \cl{e}{\rho}$
and $\vpi \: x \: v_A \: \cl{b}{\rho}$ do not have a binding for $x$.
The missing binding has to be provided before these closures can be evaluated.
This might be a concrete value (for example during $\beta$-reduction) or a
fresh generic value $k$.

\subsubsection{Environment look-up}
\[
\begin{array}{ll}
\multicolumn{2}{l}{\lkup : \Env \times \cV \ra \Val }\\
\lkup \: (\ins{\rho}{x}{v}) \: x & = v \\
\lkup \: (\ins{\rho}{y}{v}) \: x & = \lkup \: \rho \: x \mbox{ if } y \neq x  
\end{array}
\]

\subsubsection{Application}
does $\beta$-reduction and inductive function application:
\[
\begin{array}{ll}
\multicolumn{2}{l}{\ap : \Val \times \ves{\Val} \ra \Val}\\
\ap \: u \: \diamond & = u \\
\ap \: (u \: \vec{c_1}) \: \vec{c_2} & = \ap \: u \: (\vec{c_1}\:\vec{c_2})\\
\ap \: (\vlam \: x \: \cl{e}{\rho}) \: (v \: \vec{v}) & = \ap \: v \: \vec{v} \mbox{ where } v = \wh \cl{e}{\ins{\rho}{x}{v}} \\ 
\ap \: \ff \: \vec{v} & = \apf \: \ff \: \vec{v} \mbox{ if } \ff \mbox{ is a } \fun \\ 
\ap \: v \: \vec{v} & = v \: \vec{v} \mbox{ otherwise } \\
\end{array}
\]

In the following, as pattern matching can fail, the object $\: \stuck \:$ is used to indicate this. 

\subsubsection{Pattern matching}
Pattern matching returns an environment that binds the variables in the patterns to values.
When matching against a coinductive constructor, the value is \emph{forced}:
\[
\begin{array}{ll}
\multicolumn{2}{l}{\match : \Env \times \Pat \times \Val \ra \Env \cup \: \{ \: \stuck \: \} }\\
\match \: \rho \: (\cc \: \vec{p}) \: v & = \matchf \: \rho \: p \: v' \spc \mbox{if  } \cc  \mbox{ is a coinductive constructor}\\
& \spcx \spcx \spcx \mbox{and } v' = \forc \: v \\
\match \: \rho \: p \: v & = \matchf \: \rho \: p \: v \mbox{ otherwise}\\
\vs\\
\multicolumn{2}{l}{\matchf : \Env \times \Pat \times \Val \ra \Env \cup \: \{ \: \stuck \: \} }\\
\matchf \: \rho \: \inacc{e} \: v & = \rho \\
\matchf \: \rho \: x \: v & = \ins{\rho}{x}{v} \\
\matchf \: \rho \: (\cc \: \diamond) \: \cc & = \rho \\
\matchf \: \rho \: (\cc \: \vec{p}) \: (\cc \: \vec{v}) & = \matchList \: \rho \: \vec{p} \: \vec{v} \\
\matchf \: \rho \: p \: v & = \: \stuck \mbox{ otherwise}\\
\vs\\
\multicolumn{2}{l}{\matchList : \Env \times \ves{\Pat} \times \ves{\Val} \ra \Env \cup \: \{ \: \stuck \: \} }\\
\matchList \: \rho \: \diamond \: \diamond & = \rho \\
\matchList \: \rho \: (p \: \vec{p}) \: (v \: \vec{v}) & = \matchList \: \rho_2 \: \vec{p} \: \vec{v} \mbox{ where } \rho_2 \: = \match \rho \: p \: v \neq \: \stuck \\
\matchList \: \rho \: p \: v & = \: \stuck \mbox { otherwise } \\
\end{array}
\]

\subsubsection{Matching of a clause}
For a single clause, if all patterns of a clause match against the argument values, then the right hand side can be evaluated: 
\[
\begin{array}{ll}
\multicolumn{2}{l}{\matchcl : \Env \times \ves{\Pat} \times \Expr \times \ves{\Val} \ra \Val \cup \: \{ \: \stuck \: \} }\\
\matchcl \: \rho \: \diamond \: e \: \vec{v} & = \ap \: v \: \vec{v} \mbox{ where } v = \wh \cl{e}{\rho} \\
\matchcl \: \rho \: (p \: \vec{p}) \: e \: (v \: \vec{v}) & = \matchcl \: \rho_2 \: \vec{p} \: \vec{v} \mbox{ if } \rho_2 = \match \: p \: v \neq \: \stuck \\  
\matchcl \: \rho \: \vec{p}\: e \: \vec{v} & = \: \stuck \mbox{ otherwise}\\  
\end{array}
\]

\subsubsection{Matching of multiple clauses} 
Now we define how a sequence of clauses matches against the argument values. 
Each clause is tried until one is matched or there are no clauses left. It is assumed that clauses are distinct and complete, so that at most one will match:

\[
\begin{array}{ll}
\multicolumn{2}{l}{\matchcls : \ves{\Clause} \times \ves{\Val} \ra \Val \cup \: \{ \: \stuck \: \} }\\
\matchcls \: \diamond \: \vec{v} & = \: \stuck \\
\matchcls \:  ((f \: \vec{p} = e)\:\vec{\gamma}) \: \vec{v} & = v \mbox{ if } v = \matchcl \: \diamond \: \vec{p} \: e \:  \vec{v} \neq \: \stuck \\
\matchcls \: ((f \: \vec{p} = e) \:\vec{\gamma}) \: \vec{v} & = \matchcls \: \vec{\gamma} \: \vec{v} \mbox{ otherwise}  
\end{array}
\]

Note that matching can fail, even when the coverage of all clauses is complete: there can be too few arguments. Furthermore, generic values do not match against a constructor pattern. 

Now the reduction behaviour of recursive and corecursive functions can be described.
Only functions whose clauses have been type-checked (and later termination-checked) will trigger reduction.

\subsubsection{Inductive function application}
We always reduce application of an inductive function if possible:\[
\begin{array}{ll}
\multicolumn{2}{l}{\apf : \cV \times \ves{\Val} \ra \Val}\\
\apf \: \ff \: \vec{v} & = v \mbox{ if } \Sig \: \ff = (t,\vec{\gamma},\top) \mbox { and } \matchcls \: \vec{\gamma} \: \vec{v} = v \neq \: \stuck \\ 
\apf \: \ff \: \vec{v} & = \ff \: \vec{v} \mbox { otherwise } 
\end{array}
\]

\subsubsection{Corecursive unrolling}
As said above, an application of a corecursive definition is lazily unrolled when needed.
$\forc$ tries to unroll a corecursive application until a constructor appears:
\[
\begin{array}{ll}
\multicolumn{2}{l}{\forc : \Val \ra \Val }\\
\forc \: \ff & = \forc \: \ff \: \diamond \\
\forc \: \ff \: \vec{v} & = \forc \; v \mbox{ where } v = \matchcls \: \vec{\gamma} \: \vec{v} \neq \: \stuck\\
& \spcx \mbox{ if } \Sig \: \ff = (t,\vec{\gamma},\top) \mbox { and } \ff \mbox { is a } \cofun\\ 
\forc \: v & = v \mbox{ otherwise}
\end{array}
\]

\section{Example programs}

\subsection{Identity function}
Without defining data types, we still can define some non-recursive functions.
One example is the identity mapping. 
Mugda is monomorphic, so the type has to be supplied as the first argument:
\begin{bsp}
$\clet \id : \EPi{A}{\Set}{A \ra A} = \lam{A}\lam{a}a$   
\end{bsp}

\subsection{Booleans}

The type of Booleans is introduced with:
\begin{bsp}
$\data \Bool : \Set$  \\
$\spc \ttt : \Bool $\\
$\spc \fff : \Bool $
\end{bsp}
\noindent An if-then-else construct can be defined by pattern matching:
\begin{bsp}
$\fun \ite : (A : \Set ) \ra \Bool \ra A \ra A \ra A$\\
$\spc \ite A\: \ttt a\: b\: = a$\\
$\spc \ite A\: \fff a\: b\: = b$
\end{bsp}
\subsection{Natural numbers}
The type of natural numbers:
\begin{bsp}
$\data \Nat : \Set$ \\
$\spc \zero : \Nat $\\
$\spc \suc : \Nat \ra \Nat$
\end{bsp}

\noindent The addition function can be defined recursively by pattern matching:
\begin{bsp}
$\fun \add : \Nat \ra \Nat \ra \Nat$\\
$\spc \add x \: \zero = x $\\
$\spc \add x \: (\suc y) = \suc (\add x \: y)  $
\end{bsp}
\subsection{Lists}
Lists are an example of a parameterized data type:
\begin{bsp}
$\data \List ( + \: A : \Set ) : \Set $ \\
$ \spc \nil : \List A  $\\
$ \spc \cons : A \ra \List A \ra \List A $
\end{bsp}
\subsection{Finitely branching trees}
\label{tre}
The following declaration introduces finitely branching trees where nodes and leafs carry elements of type $A$.
\begin{bsp}
$\data \Tree ( + \: A : \Set ) : \Set $ \\
$ \spc \node : A \ra \List (\Tree A ) \ra \Tree A  $
\end{bsp}
A leaf is a node with an empty list of successors. 
In $\node$, the recursive argument $\Tree A$ appears as a parameter to $\List$.
This will be allowed because the parameter of $\List$ is \emph{strictly positive}.
\subsection{Vectors}
Now things get more interesting. Vectors are an example of an inductive family of types.
\begin{bsp}
$\data \vVec ( + \: A : \Set ) : \Nat \ra \Set $ \\
$\spc \nil : \vVec A \: \zero  $\\
$ \spc \cons : \EPi{n}{\Nat}{ A \ra \vVec A \: n \ra \vVec A \: (\suc n)} $
\end{bsp}
Next let us define the $\heads$ function that returns the first element of a vector. This operation should only be allowed for a non-empty vector.
With dependent types, we can express this in the type signature:
\begin{bsp}
$ \fun \head : ( A : \Set ) \ra ( n : \Nat ) \ra \vVec A \: (\suc n) \ra A $
\end{bsp}
Now on to the clause definition, the usage of inaccessible patterns will become clear.
First, pattern matching against the vector argument is needed:
\begin{bsp}
$\head ? \: ? \: (\cons \: B \: m \: x \: xl) = x $
\end{bsp}
It is of note that a clause for the $\nil$ case is not needed, because $\nil$ does not belong to a type of the form $\vVec A \: (\suc n)$.
Now, what about those question marks above ? 
One might be inclined to use \emph{non-linear} pattern variables:
\begin{bsp}
$\head \: B \: m \: (\cons  B \: m \: x \: xl) = x $
\end{bsp}
But really, the system must not check at run-time that the values at the corresponding arguments are equal -- it is guaranteed for a well-typed program.

So to capture the notion that the first two arguments are automatically instantiated by matching against a constructor, the inaccessible pattern notation \cite{norell:thesis,GoguHMcBrCM2006} is used. The final definition is:
\begin{bsp}
$ \fun \head : ( A : \Set ) \ra ( n : \Nat ) \ra \vVec A \: (\suc n) \ra A $\\
$ \spc \head \inacc{B} \: \inacc{m} \: (\cons \: B \: m \: x \: xl) = x $
\end{bsp}
\subsection{Equality}
The following \emph{predicate} is called Martin-L\"of equality, another important example of a type that can be defined as an inductive family:
\begin{bsp}
$\data \Eq ( A : \Set ) : A \ra A \ra \Set $ \\
$\spc \refl : \EPi{a}{A}{\Eq A \: a \: a}$
\end{bsp}
where the single constructor $\refl$ states reflexivity of equality.

\noindent As a simple example, the program
\begin{bsp}
$ \clet \prof : \EPi{x}{\Nat}{\Eq \Nat (\add x \: \zero)\: x} $ \\ 
$ \spcx = \lam{y}\refl \Nat y$ 
\end{bsp}
can now be seen as a proof of the mathematical proposition $ \forall x \in \cN : x  + 0 = x $. Without having formally introduced type-checking, we want to provide some intuition on issues that arise while checking such declarations.

The above example passes the type-checker because $\add x \: \zero$ reduces to $\zero$, and which is -- for the type system -- equal to itself. 
Now, to prove $ \forall x \in \cN : 0 +  x = x $, more effort is needed.
The following will not be accepted:
\begin{bsp}
$ \clet \tprof : \EPi{x}{\Nat}{\Eq \Nat (\add \zero x) \: x} $ \\ 
$ \spcx \lam{y} \refl \Nat y$ 
\end{bsp}
The type-checker will only be able to ``see'' definitional equality.
As addition was defined by recursion on the second argument. $\add \zero x$ does not reduce to $x$ during type-checking.

More technically, the generic value introduced for $y$ -- that ``moves up'' to the type -- neither matches against the constructor pattern $\zero$ in the first clause
nor against the constructor pattern $\suc y$ in the second clause of $\add$.

So what is needed is a recursive proof, by manually doing case distinction:
\begin{bsp}
$\fun \eqsucc : \EPi{x}{\Nat}\EPi{y}{\Nat} \Eq \Nat x \: y \ra \Eq \Nat (\suc x) \: (\suc y)$ \\
$\spc \eqsucc \inacc{x} \: \inacc{x} \: (\refl \inacc{\Nat} \: x) = \refl \Nat (\suc x)$

$\fun \tprof : ( x : \Nat ) \ra \Eq \Nat (\add \zero x) \: x$\\
$\spc \tprof \zero = \refl \Nat \zero$\\
$\spc \tprof (\suc x) = \eqsucc (\add \zero x)\: x \: (\tprof x)$
\end{bsp}
Recursive proofs need to be total functions to be logically valid.

\subsection{Streams}
\label{lstream}
Next we introduce streams as an example of a coinductive type.
Coinductive types are not required to be well-founded, i.e. their inhabitants do not need to have finite height.
\begin{bsp}
$\codata \Stream : \Set $ \\
$\spc \cons : \Nat \ra \Stream \ra \Stream $
\end{bsp}
For less clutter in later examples, we focus here on streams of natural numbers, although they could be parameterized just like lists.
As $\Stream$ has only one constructor, there are no $\Stream$ objects of finite height.

A stream of zeroes can be declared by the following corecursive declaration:
\begin{bsp}
$\cofun \zeroes: \Stream $ \\
$\spc \zeroes = \cons \zero \zeroes$
\end{bsp}

\noindent The first element of a stream can be computed by the function $\head$:
\begin{bsp}
$\fun \head : \Stream \ra \Nat$\\
$\spc \head \: (\cons \: x \: xs) = x$
\end{bsp}

As expected $\wh\; {\clp{\head \zeroes}{\diamond}} = \zero $ because $\head$ triggers an unfolding. This evaluation is well-defined because $\zeroes$ is productive.

\chapter{Type-Checking}

\renewcommand\Check[5]{#1;#2;#3\vdash#4\leftleftarrows#5}
\newcommand\IsType[4]{#1;#2;#3\vdash#4\:\mathbf{Type}}
\newcommand\DataType[5]{#1;#2;#3;#4\vdash#5\:\mathbf{DataType}}
\newcommand\ConType[5]{#1;#2;#3;#4\vdash#5\:\mathbf{ConType}}
\newcommand\Infer[5]{#1;#2;#3\vdash#4\rightrightarrows#5}
\newcommand\EqVal[4]{#1;#2\vdash#3\Leftrightarrow#4}
\newcommand\FeqVal[4]{#1;#2\vdash#3{\:\leftrightarrow\:}#4}
\newcommand\DeqVal[2]{\vdash#1\leftrightarrow#2}
\newcommand\LeqVal[4]{#1;#2;\vdash#3\:\leqq\:#4}
\newcommand\FleqVal[4]{#1;#2\vdash#3{\:\ll\:}#4}
\newcommand\DLeqVal[2]{\vdash#1\leq#2}
\newcommand\SzLeq[2]{\vdash#1\sqsubseteq#2}

The starting point for this chapter was the simple type-checking algorithm for dependent types that is given in \cite{coquand96algorithm}.

In the following, we will introduce judgments for type-checking.
A rule of a judgment has the general form
\[
\nru{}
{A_1 \qquad \ldots \qquad A_n}
{B}
{\: c}
\]
\noindent where the $A_i$ are the premises and $B$ is the conclusion.
All these judgments given in this work are \emph{algorithmic}: a deterministic algorithm that builds a derivation \emph{bottom-up} is directly induced by the rules. If more than one rule is applicable, the first one (in reading direction -- left to right, top to bottom) must be chosen.

\section{Scope-Checking}
In the semantics of the previous section, looking up in an environment or in the signature was expected never to fail. This is only guaranteed for well-scoped programs, where identifiers are always \emph{in scope}.

We deal with this problem here by postulating that type-checking fails if lookup in an environment or signature is not defined.
In an actual implementation (see appendix), the use of identifiers can already be checked after parsing. 

\section{Bidirectional type-checking} 
A standard technique for dependent types called \emph{bidirectional type-checking} is used.
Intuitively, this means that the type-checker has two modes:
one for checking that an expression has a certain type and one for checking that an expression is correct while inferring its type.

\noindent The following three judgments are defined simultaneously:
\paragraph*{check type} checks that $A$ denotes a valid type (figure \ref{ftype}): 
\[\IsType{k}{\rho_1}{\rho_2}{A} \subseteq \cN \times \Env \times \Env \times \Expr \]

\paragraph*{check expression} checks that $e$ the type $v$ (figure \ref{fcheck}):

\[\Check{k}{\rho_1}{\rho_2}{e}{v} \subseteq \cN \times \Env \times \Env \times \Expr \times \Val\]

\paragraph*{infer type} checks that $e$ is correct and infers its type value $v$ (figure \ref{finf}):
\[\Infer{k}{\rho_1}{\rho_2}{e}{v} : \cN \times \Env \times \Env \times \Expr \ra \Val\]


The environment $\rho_1$ will be used to bind fresh generic values to variables, and $\rho_2$ will bind the type value corresponding to these generic values.

As noted, in checking mode, the type-checker might have to infer the type of the expression and then verify that the inferred type value is equal to the one that is being checked against : \emph{Equality} (also called \emph{convertibility}) between two values is needed:

\paragraph*{Equality of values}
A simple equality derived from \cite{coquand96algorithm} is presented. 
Regarding infinite objects, a corecursive definition should be equal to its unfolding.
We try to approximate this by allowing the type-checker to unroll ($\forc$) a corecursive definition. Unlimited unfolding cannot be allowed, thus we keep track of unfolding and only allow one side of two to be unrolled.

The \emph{force history} set $ \cF = \{\lF, \rF, \nF \} $ is introduced to keep track if the left or right side ($v_2$) has been forced, or no side yet.
We will return to the subject of equality for infinite objects in section \ref{heq}.

Two judgments (figure \ref{feq}) are defined simultaneously, the latter operating on already unrolled values:
\[\EqVal{f}{k}{v_1}{v_2} \subseteq \cF \times \cN \times \Val \times \Val \]
\[\FeqVal{f}{k}{v_1}{v_2} \subseteq \cF \times \cN \times\Val \times \Val \]

\begin{figure}[htp]

\begin{gather*}
%% Let
\ndru{\chklet}
{
\IsType{k}{\rho_1}{\rho_2}{A}
\qquad
\Check{k}{\rho_1}{\rho_2}{e1}{\wh \cl{A}{\rho_1}}
}
{
\Check{k}{\ins{\rho_1}{x}{\wh \cl{e_1}{\rho_1}}}{\ins{\rho_2}{x}{\wh \cl{A}{\rho_1}}}{e_2}{v}
}
{\Check{k}{\rho_1}{\rho_2}{\ELet{x}{A}{e_1}{e_2}}{v}}
\\[2ex]
%Lam
\nru{\chklam}
{\Check{k+1}{\ins{\rho_1}{x}{k}}{\ins{\rho_2}{x}{v_A}}{e}{\wh \cl{t}{\ins{\rho}{x}{k}}}}
{\Check{k}{\rho_1}{\rho_2}{\lam{x}{e}}{\vpi \: x \: v_A \: \cl{t}{\rho}}}
%% Pi
\\[2ex]
\nru{\chkpi}
{\Check{k}{\rho_1}{\rho_2}{A}{\Set}
\qquad
\Check{k+1}{\ins{\rho_1}{x}{k}}{\ins{\rho_2}{x}{\wh \cl{A}{\rho_1}}}{B}{\Set}
}
{\Check{k}{\rho_1}{\rho_2}{\EPi{x}{A}{B}}{\Set}}
\\[2ex]
%% Infer-mode
\nru{\chkinf}
{\Infer{k}{\rho_1}{\rho_2}{e}{v_2}
\qquad
{\EqVal{\nF}{k}{v_2}{v_1}}} 
{\Check{k}{\rho_1}{\rho_2}{e}{v_1}}
\end{gather*}
\caption{Checking expressions}
\label{fcheck}
\end{figure}

\begin{figure}[htp]
\begin{gather*}
%% App-1
\nru{\infappa}
{\Infer{k}{\rho_1}{\rho_2}{e_1}{\vpi \: x \: v_A \: \cl{b}{\rho}}
\qquad \Check{k}{\rho_1}{\rho_2}
{e_2}{v_A}}
{\Infer{k}{\rho_1}{\rho_2}{e_1 \: e_2}{\wh \cl{b}{\ins{\rho}{x}{\wh \cl{e_2}{\rho_1}}}}}
%% App-2
\\[2ex]
\nru{\infappb}
{\Infer{k}{\rho_1}{\rho_2}{(e_1 \: e_2)\:\vec{e}}{v}}
{\Infer{k}{\rho_1}{\rho_2}{e_1 \: (e_2 \: \vec{e})}{v}}
\qquad
%% Var
\nru{\infvar}{v = \lkup \: \rho_2 \: x }{\Infer{k}{\rho_1}{\rho_2}{x}{v}}
\\[2ex]
%% Def	
\nru{\infdef}
{\Sig \: \ff \: = (v,\vec{\gamma},b) }
{\Infer{k}{\rho_1}{\rho_2}{\ff}{v}}
\qquad
%% Con 
\nru{\infcon}
{\Sig \: \cc \: = v}
{\Infer{k}{\rho_1}{\rho_2}{\cc}{v}}
\\[2ex] 
%% Data
\nru{\infdata}
{\Sig \: \DD \: = (v,n)}
{\Infer{k}{\rho_1}{\rho_2}{\DD}{v}} 
%% Const
\qquad
\nru{\inflet}
{\Sig \: \dd \: = (e,v)}
{\Infer {k}{\rho_1}{\rho_2}{\dd}{v}}
\end{gather*}
\caption{Inferring type of expressions}
\label{finf}
\end{figure}

\begin{figure}[htp]
\begin{gather*}
\nru{}
{\IsType{k}{\rho_1}{\rho_2}{A}
\qquad
\IsType{k+1}{\ins{\rho_1}{x}{k}}{\ins{\rho_2}{x}{\wh \cl{A}{\rho}}}{B}}
{\IsType{k}{\rho_1}{\rho_2}{\EPi{x}{A}{B}}}
\\[2ex]
\nru{}{}{\IsType{k}{\rho_1}{\rho_2}{\Set}}
\qquad
\nru{}
{\Check{k}{\rho_1}{\rho_2}{t}{\Set}}
{\IsType{k}{\rho_1}{\rho_2}{t}}
\end{gather*}
\caption{Type judgment}
\label{ftype}
\end{figure}

\begin{figure}[htp]
\begin{gather*}
%% Force
\nxru{\eqforce}
{\FeqVal{\lF}{k}{\forc \; v_1}{v_2} \qquad \mbox{if } \forc \: v_1 \neq v_1 , \forc \: v_2 = v_2  , f \neq \rF
}
{\FeqVal{\rF}{k}{v_1}{\forc \; v_2} \qquad \mbox{if } \forc \: v_1 = v_1 , \forc \: v_2 \neq v_2 , f \neq \lF }
{\FeqVal{f}{k}{v_1}{v_2} \qquad \qquad \qquad \qquad \qquad \mbox{ otherwise}}
{\EqVal{f}{k}{v_1}{v_2}}
\\[2ex]
%% App
\nru{\eqapp}
{\EqVal{f}{k}{v}{w} \qquad
\EqVal{f}{k}{v_i}{w_i} \mbox{ for all } i \in \{ 1 \ldots n\}}
{\FeqVal{f}{k}{v \: v_1 \ldots v_n}{w \: w_1 \ldots w_n}}
%% Pi
\\[2ex]
\nru{\eqpi}
{\EqVal{f}{k}{v_1}{v_2}
\qquad
\EqVal{f}{k+1}{\wh \cl{b_1}{\ins{\rho_1}{x_1}{k}}}{\wh \cl{b_2}{\ins{\rho_2}{x_2}{k}}}}
{\FeqVal{f}{k}{\vpi \: x_1 \: v_1 \: \cl{b_1}{\rho_1}}{\vpi \: x_2 \: v_2 \: \cl{b_2}{\rho_2}}}
%% Lam
\\[2ex]
\nru{\eqlam}
{\EqVal{f}{k+1}{\wh \cl{e_1}{\ins{\rho_1}{x_1}{k}}}{\wh \cl{e_2}{\ins{\rho_2}{x_2}{k}}}}
{\FeqVal{f}{k}{\vlam \: x_1 \: \cl{e_1}{\rho_1}}{\vlam \: x_2 \: \cl{e_2}{\rho_2}}}
%% Atom
\qquad
\nru{\eqatom}
{}
{\FeqVal{f}{k}{a}{a}}
\end{gather*}
\caption{equality checking}
\label{feq}
\end{figure}

After having defined the basic building blocks, type-checking individual declarations is next.
For readability, we formalize these as imperative algorithms rather than judgments. 
When a declaration passes the type-check, the signature can be expanded. 
\subsubsection{Type-correctness of empty signature:}
The empty signature $\Sig_0$ is \emph{type-correct}.

\clearpage

\section{Let declarations}
Checking a global let declaration is pretty straightforward:

\subsubsection{Algorithm}
\begin{quote} 
For $ \delta = \clet \dd : A = e $ and a type-correct signature $\Sigma$, if 
\begin{enumerate}
\item
$\IsType{1}{\diamond}{\diamond}{A}$ 
\item
$ v_A := \wh \cl{A}{\diamond}$
\item
$\Check{1}{\diamond}{\diamond}{e}{v_A}$ 
\item
$\Sig' := \Sig \cup \{\dd \mapsto (e,v_A)\}$
\end{enumerate}
then $\Sig'$ is a type-correct signature.
\end{quote}
\section{Data type declarations}
We have to check that data type declarations follow the scheme of inductive families \cite{dybjer94inductive}.
Actually, we here stay closer to the current version of Agda, namely the strict positivity test is more lenient. In addition, \mugda supports coinductive types \cite{coquand-infinite,gimenez98tutorial} with the $\codata$ construct.


\subsection{Checking data and constructor types}

First, a valid data type declarations needs to have the syntactic form outlined in figure \ref{syn},
so that the parameters $p_i \ldots p_n$ in the result type of every constructor exactly match those of the telescope. 

\begin{figure}[htp]
\centering{
\begin{minipage}{0.5\textwidth}
$\data \DD \: ( p_1 : P_1) \ldots ( p_n : P_n) : \Gamma \ra \Set $\\
$\spc \cc_1 : \Delta_1 \ra \DD \: p_1 \ldots p_n \: t_1 \ldots t_m$\\
$\spc \ldots$\\   
$\spc \cc_k : \Delta_k \ra \DD \: p_1 \ldots p_n \: t_1 \ldots t_m$\\
\end{minipage}
}
\caption{Syntactic valid form of data declaration}
\label{syn}
\end{figure}


Figure \ref{fdata} shows the check for the data type, and figure \ref{fcon} shows the check for each constructor type.
As the types will depend on the parameters, we will have to prepend the telescope when checking them.

The parameter types $P_i$ can be arbitrary types, while the types in indices $\Gamma$  and the constructor arguments in $\Delta_i$  need to be small types, i.e. of type $\Set$. 
Otherwise the data type $\Vd$ in figure \ref{fv}, adapted from \cite{coquand92pattern}, would be accepted.
Then $\lop$ would pass the termination-check of the next chapter. In a language with a \emph{predicative} hierarchy of universes like Agda \cite{norell:thesis}, the type $\Vd$ is not accepted as a small type, but as an inhabitant of a higher level ($\Set_1$). Then $\lop$ would not be type-correct.



\begin{figure}[htp]
\centering{
\begin{minipage}{0.5\textwidth}
\begin{bsp}
$\data \Vd : \Set $\\
$\spc \vc : (( A : \Set ) \ra (A \ra A )) \ra \Vd$

$\fun \lop : \Vd \ra \Set$\\
$\spc \lop (\vc f) = \lop (f \: \Vd (\vc f)) $

$\clet \diverge : \Set = \lop (\vc \id)$
\end{bsp}
\end{minipage}
}
\caption{Invalid data type V}
\label{fv}
\end{figure}


\begin{figure}[htp]
\begin{gather*}
\nxru{}
{\mbox {if } k < n \mbox{ then } \IsType{k}{\rho_1}{\rho_2}{A}}
{\mbox {if } k \geq n \mbox { then } \Check{k}{\rho_1}{\rho_2}{A}{\Set}}
{\DataType{k+1}{\ins{\rho_1}{x}{k}}{\ins{\rho_2}{x}{\wh \cl{A}{\rho}}}{n}{B}}
{\DataType{k}{\rho_1}{\rho_2}{p}{\EPi{x}{A}{B}}}
\\[2ex]
\nru{}
{}
{\DataType{k}{\rho_1}{\rho_2}{n}{\Set}}
\end{gather*}
\caption{Data-type judgment}
\label{fdata}
\end{figure}

\begin{figure}[htp]
\begin{gather*}
\ndru{}
{\mbox {if } k \geq n \mbox { then } \Check{k}{\rho_1}{\rho_2}{A}{\Set}}
{\ConType{k+1}{\ins{\rho_1}{x}{k}}{\ins{\rho_2}{x}{\wh \cl{A}{\rho}}}{n}{B}}
{\ConType{k}{\rho_1}{\rho_2}{n}{\EPi{x}{A}{B}}}
\\[2ex]
\nru{}
{\Check{k}{\rho_1}{\rho_2}{v}{\Set}}
{\ConType{k}{\rho_1}{\rho_2}{n}{v}}
\end{gather*}
\caption{Constructor type judgment}
\label{fcon}
\end{figure}

\subsection{Strict positivity}

\newcommand{\pos}{\mbox{pos}}

In a telescope, each $(p_i : P_1) $ can be written as $( + \: p_i : P_1) $ to denote strict positivity.

\subsubsection{Positive parameters}
For $\DD$, we define the set of positive parameter indices $\pos(\DD)$:

\[\pos(\DD) \subseteq \cN := \{ \: i \mid p_i \mbox{ is declared strictly positive} \}\]

We require that data type declarations are strictly positive.
Otherwise there are inconsistencies \cite{paulinmohring93inductive}, i.e. non-terminating terms could be constructed that would fail the check of the next chapter.
Roughly, its meaning is: In every constructor argument. the data type to be defined is not allowed to occur in a function domain or in an application.

An exception to this that is allowed in \mugda is that parameters that are for themselves strictly positive do preserve strict positivity. 
This allows the definition of the $\Tree$ type in section \ref{tre} to pass the strict positivity test. In that regard, \mugda 's data types are closer to the \emph{strictly positive families} described in \cite{alti:cats07}.
For the strict positivity test of a constructor (figure \ref{sposcf}) the non-occurrence (figure \ref{fnocc}) and the strict positive occurrence (figure \ref{fspos}) of an atomic value is needed.

\newcommand{\sposc}[3]{#1\vdash#2\mbox{ \textbf{sposc} }#3}
\newcommand{\spos}[3]{#1\vdash#2\mbox{ \textbf{spos} }#3}


\newcommand{\nocc}[3]{#1\vdash#2\mbox{ \textbf{nocc} }#3}



\begin{figure}[htp]
\begin{gather*}
\nru{}
{\nocc{k}{a}{v_A} \qquad \nocc{k+1}{a}{\wh \cl{b}{\ins{\rho}{x}{k}}}}
{\nocc{k}{a}{\vpi \: x \: v_A \: \cl{b}{\rho}}  }
\qquad
\nru{}
{\nocc{k+1}{a}{\wh \cl{e}{\ins{\rho}{x}{k}}}}
{\nocc{k}{a}{\vlam \: x \: \cl{e}{\rho}}  }
\\[2ex]
\nru{}
{\nocc{k}{a}{v} \qquad \nocc{k}{a}{v_j} \mbox{ for all } j \in \{1 \ldots n \}}
{\nocc{k}{a}{v \: v_1 \ldots v_n}}
\qquad
\nru{}
{a \neq a'}
{\nocc{k}{a}{a'}}
\end{gather*}
\caption{Non-occurrence of atomic value a}
\label{fnocc}
\end{figure}

\begin{figure}[htp]
\begin{gather*}
\nru{}
{ \nocc{a}{a}{v_A} \qquad \spos{k+1}{a}{\wh \cl{b}{\ins{\rho}{x}{k}}}}
{\spos{a}{a}{\vpi \: x \: v_A \: \cl{b}{\rho}}  }
\qquad
\nru{}
{\spos{k+1}{a}{\wh \cl{e}{\ins{\rho}{x}{k}}}}
{\spos{k}{a}{\vlam \: x \: \: \cl{e}{\rho}}  }
\\[2ex]
\nru{}
{\nocc{k}{a}{v_j} \mbox{ for all } j \in \{1 \ldots n \}} 
{\spos{k}{a}{a \: v_1 \ldots v_n}}
\\[2ex]
\ndru{}
{\nocc{k}{a}{v_j} \mbox{ for all } j \in \{ 1 \ldots m\} , j \notin \pos(\DD)}
{\spos{k}{a}{v_k} \mbox{ for all } k \in \pos(\DD) \qquad}
{\spos{k}{a}{\DD \: v_1 \ldots v_m}}
\\[2ex]
\nru{}
{\spos{k}{a}{v} \qquad \nocc{k}{a}{v_j} \mbox{ for all } j \in \{1 \ldots n \}}
{\spos{k}{a}{v \: v_1 \ldots v_n}}
\qquad
\nru{}
{}
{\spos{k}{a}{v}}
\end{gather*}
\caption{Strictly positive occurrence of atomic value a}
\label{fspos}
\end{figure}


\subsubsection{Example : Untyped lambda calculus}
Figures \ref{abs1} and \ref{abs2} show two encodings of untyped lambda calculus that are rejected. 
The data type $\Term$ is rejected in figure \ref{abs1}, and in figure \ref{abs2}, $\funt$ is rejected because its parameter $A$ is declared as strictly positive, but is not.
In both example, $\app$ is non-recursive and would pass the termination check of the next chapter.


\begin{figure}[htp]
\begin{gather*}
\nru{}
{ \spos{k}{a}{v_A} \qquad \sposc{k+1}{a}{\wh \cl{b}{\ins{\rho}{x}{k}}}}
{\sposc{k}{a}{\vpi \: x \: v_A \: \cl{b}{\rho}}  }
\nru{}
{}
{\sposc{k}{a}{v}}
\end{gather*}
\caption{strict positivity test for constructor}
\label{sposcf}
\end{figure}

\begin{figure}[htp]
\centering{
\begin{minipage}{0.6\textwidth}
$\data \Term : \Set$\\
$\spc \abs : (\Term \ra \Term) \ra \Term$

$\fun \app : \Term \ra \Term \ra \Term$\\
$\spc \app (\abs f) \: y = f \: y$

$\clet   \omegad : \Term = \abs (\lam{x} \app x \: x)$\\
$\clet \diverge : \Term = \app \omegad \omegad$
\end{minipage}
}
\caption{Encoding of untyped lambda calculus}
\label{abs1}
\end{figure}

\begin{figure}[htp]
\centering{
\begin{minipage}{0.6\textwidth}
$\data \funt (+ A : \Set ) : \Set$\\
$\spc \func : (A \ra A) \ra \funt A$

$\data \Term : \Set$\\
$\spc \abs : \funt \Term \ra \Term$

$\fun \app : \Term \ra \Term \ra \Term$\\
$\spc \app (\abs (\func \inacc{\Terms} \: f)) \: y = f \: y$

$\clet \omegad : \Term = \abs (\func \Term (\lam{x} \app x \: x))$\\
$\clet \diverge : \Term = \app \omegad \omegad$
\end{minipage}
}
\caption{Another encoding of untyped lambda calculus}
\label{abs2}
\end{figure}

\newpage

\subsection{Checking the whole declaration}

\subsubsection{Algorithm}
\begin{quote}
For declaration $\delta = \data \DD \: \tau : A \: \vec{\gamma}$ and type-correct signature $\Sig$, if 
\begin{enumerate}
\item 
$\delta$ follows form of figure \ref{syn}
\item
$ n := \vert\tau\vert $
\item
$\DataType{1}{\diamond}{\diamond}{n}{ \tau \ra A}$ 
\item
$v_D := \wh \clp{\tau \ra A}{\diamond}$
\item
$\Sig' := \Sig \cup \{\DD \mapsto (v_D,n)\}$
\item
Using $\Sig'$, for every constructor declaration $\cc_i : B_i  \in \vec{\gamma}$:
\begin{enumerate} 
\item
$\ConType{1}{\diamond}{\diamond}{n}{\tau \ra B_i}$ 
\item
$v_i := \wh \clp{\tau \ra B_i}{\diamond}$
\item
$\sposc{1}{v_i}{j}$ for every $j \in \pos(\DD) $ 
\item
$\sposc{1}{v_i}{\DD}$ 
\end{enumerate}
\item
$\Sig'' := \Sig' \cup \displaystyle\bigcup_{i} \{\cc_i \mapsto v_i\}$
\end{enumerate}
then $\Sig''$ is a type-correct signature.
\end{quote}
A $\codata$ declaration is checked just the same.


\section{Function declarations}

\newcommand\checkcl[2]{#1\:\mathbf{checkcl}\:#2}
\newcommand\checkp[5]{(#1)\vdash#2\:\mathbf{chkP}\:#3\Rightarrow (#4),#5}
\newcommand\checkps[5]{(#1)\vdash#2\:\mathbf{chkPs}\:#3\Rightarrow (#4),#5}
\newcommand\checkinacc[4]{(#1)\vdash#2\:\mathbf{checkinacc}\:#3 : #4}
\newcommand\unify[5]{#1,#2\vdash#3\:\mathbf{inst}\:#4\Rightarrow #5}
\newcommand\unifylist[5]{#1,#2\vdash#3\:\mathbf{inst_{list}}\:#4\Rightarrow #5}


\subsubsection{Overview}
We need to check each clause $\ff \: \vec{p} \: e$ of a function separately against the declared type.
So for each clause, we have to check the patterns $\vec{p}$ and then the right hand side $e$.
Checking the patterns $\vec{p}$ will yield an environment for the free variables in $e$.

Because of the inaccessible patterns, checking $\vec{p}$ will need two phases.
In the first phase, we will skip inaccessible patterns, and only check the \emph{accessible part} of the patterns.
Thee inaccessible patterns will be represented by fresh \emph{flexible} generic values.
These flexible generic values will be \emph{instantiated} to concrete values when checking constructor patterns.
Then in the second phase, it is verified that the expressions of the inaccessible patterns are equal to those instantiated in the first phase, and finally the right hand side $e$ can be checked.

\newcommand{\subst}[2]{#1\{#2\}}
\newcommand{\substEnv}[2]{#1\{#2\}}
\newcommand{\compS}[2]{\opn{comp}\:#1\:#2}
\newcommand\ptv[2]{#1\Searrow#2}
\newcommand\ptvx{\opn{p2v}}
\newcommand\pstvs{\opn{p_{s}2v_{s}}}

\subsection{Syntactic checks for patterns}

We first outline some syntactic requirements on the patterns:
\begin{itemize}
\item
all patterns of a clause are linear - all variables occur only once in the accessible parts.
\item
variable bindings in the right hand side do not \emph{shadow} the variable patterns.
\item
all clauses should have the same number of patterns.
\end{itemize}
This can already be checked during scope-checking.
The last two items are important for the syntactic termination-checker described in the following chapter.

\subsection{Coverage of pattern matching}
Also not covered here is very important for logical consistency:
It is necessary to check that the clauses cover all possible cases.
This is quite complicated \cite{norell:thesis,schurmann03coverage} for inductive families and will not be described here.
In addition, due to our ``try all clauses'' semantics, clauses should not overlap.

\subsection{Preliminaries}


\subsubsection{Converting pattern to value}
During type-checking of the accessible patterns, patterns will be converted to values because the type may depend on them:
\[
\begin{array}{ll}
\multicolumn{2}{l}{\ptv{\_}{\_}: \cN \times \Pat \ra \Val}\\
\ptv{k}{p} & = v \mbox{ where } \ptvx \: k \: p = (v,k')
\vs\\ 
\multicolumn{2}{l}{\ptvx : \cN \times \Pat \ra (\Val,\cN)}\\
\ptvx \: k \:x & = (k,k+1) \\
\ptvx \: k \: (\cc \: \diamond) & = (\cc,k) \\
\ptvx \: k \: (\cc \: \vec{p}) & = (\cc \: \vec{v},k') \mbox{ where } (\vec{e},k') = \pstvs \: k \: \vec{p}\\
\ptvx \: k \: \inacc{e} & = (k,k+1) 
\vs\\
\multicolumn{2}{l}{\pstvs : \cN \times \ves{\Pat} \ra (\ves{\Val},\cN)}\\
\pstvs \: k \: \diamond & = (\diamond,k) \\
\pstvs \: k \: (p \: \vec{p}) & = (k'', v \: \vec{v}) \mbox{ where } \\
& \spcx (v,k') = \ptvx \: k \: p \\
& \spcx (\vec{v},k'') = \pstvs \: k' \: \vec{p}  
\end{array}
\]
Both pattern variables and inaccessible patterns are converted to ascending fresh generic values, just as later during checking of the pattern.

Instantiating by successful \emph{unification} (figure \ref{finst}) will yield a substitution for flexible generic values:
\subsubsection{Substitutions}
A substitution is a list $\sigma : \ves{(\cN \times \Val)} $ and denotes a partial mapping of generic values to values.
$\diamond$ is the empty substitution.
$\sigma[k \rightarrow v] $ is the substitution where $k$ is mapped to $v$ and others according to $\sigma$.

\subsubsection{Application of a substitution}
Now we define simultaneously the application of a substitution on values and environments:
\[
\begin{array}{ll}
\multicolumn{2}{l}{\subst{\_}{\_} : \ves{(\cN \times \Val)} \times \Val \ra \Val} \\
\subst{\sigma}{k} & = v \mbox{ if } (k,v) \in \sigma \\
\subst{\sigma}{v \: v_1 \ldots v_n} & = \subst{\sigma}{v} \: \subst{\sigma}{v_1} \ldots \subst{\sigma}{v_n}\\
\subst{\sigma}{\vpi \: x \: v_A \: \cl{B}{\rho}} & = \vpi \: x \: \subst{\sigma}{a_v} \: \cl{B}{\subst{\sigma}{\rho}} \\
\subst{\sigma}{\vlam \: x \: \cl{e}{\rho}} & = \vlam \: x \: \cl{e}{\subst{\sigma}{\rho}} \\
\subst{\sigma}{v} & = v \mbox{ otherwise }
\vs\\
\multicolumn{2}{l}{\substEnv{\_}{\_} : \ves{(\cN \times \Val)} \times \Env \ra \Val} \\
\substEnv{\sigma}{\diamond} & = \diamond \\
\substEnv{\sigma}{(x,v)\rho} & = (x,\subst{\sigma}{v}) \substEnv{\sigma}{\rho} \\
\end{array}
\]

\subsubsection{Composition of substitutions}
Next is the composition $\compS{\sigma_1}{\sigma_2}$ of two substitutions $\sigma_1$ and $\sigma_2$:
\[
\begin{array}{ll}
\multicolumn{2}{l}{\compS : \ves{(\cN \times \Val)} \times \ves{(\cN \times \Val)} \ra \ves{(\cN \times \Val)}} \\
\compS{((k_1,v_1)\ldots(k_n,v_n))}{\sigma_2} & = ((k_1,\subst{\sigma_2}{v_1})\ldots(k_n,\subst{\sigma_2}{v_n})\: \sigma_2) 
\end{array}
\]
It is assumed that the domains of the substitutions are disjoint and that there are no occurrences of the generic values of $\sigma_1$ in the values of the codomain of $\sigma_2$.  

\begin{figure}[htp]
\begin{gather*}
\nru{\unifyl}
{\nocc{k}{k'}{v}}
{\unify{k}{\xi}{k'}{v}{(k',v)}}
{\; k \in \xi }
\\[2ex]
\nru{\unifyr}
{\nocc{k}{k'}{v}}
{\unify{k}{\xi}{v}{k'}{(k',v)}}
{\; k \in \xi}
\\[2ex]
\nru{\unifyc}
{{\unifylist{k}{\xi}{\vec{v}}{\vec{w}}{\sigma}}}
{\unify{k}{\xi}{\cc \: \vec{v}}{\cc \: \vec{w}}{\sigma}
}
\qquad
\nru{\unifyd}
{{\unifylist{k}{\xi}{\vec{v}}{\vec{w}}{\sigma}}}
{\unify{k}{\xi}{\DD \: \vec{v}}{\DD \: \vec{w}}{\sigma}
}
\\[2ex]
\nru{\unifye}
{\EqVal{k}{N}{v_1}{v_2}}
{\unify{k}{\xi}{v_1}{v_2}{\diamond}
}
\\[2ex]
\nru{}
{}
{\unifylist{k}{\xi}{\diamond}{\diamond}{\diamond}}
\qquad
\ndru{}
{\unify{k}{\xi}{v_1}{w_1}{\sigma}}
{\unifylist{k}{\xi}{\subst{\sigma}{v_2} \ldots \subst{\sigma}{v_n}}{\subst{\sigma}{w_2} \ldots \subst{\sigma}{w_n}}{\sigma'}}
{\unifylist{k}{\xi}{v_1 v_2 \ldots v_n}{w_1 w_2 \ldots w_n}{\compS \sigma \sigma'}}
\end{gather*}
\caption{Instantiation of flexible values by unification}
\label{finst}
\end{figure}

\subsection{Checking accessible part of patterns}

The main work of the first phase is the judgment 
\[
\checkp{k,\xi,\sigma,\rho_1,\rho_2}{p}{v}{k',\xi',\sigma',\rho'_1,\rho'_2}{v'}
\]

where
\begin{itemize}
\item
$k$ denotes the next fresh generic value
\item
$\rho_1$ and $\rho_2$ are environments, again providing for a variable a value and its type.
\item
$\xi \subseteq (\cN \times (\Expr \times \Val)$ is a set of flexible generic values, together with the corresponding inaccessible expression and its supposed type.
\item
$\sigma$ is a substitution for the flexible generic values
\item
$p$ is the pattern to check
\item
$v$ is the remaining type of the function.
\item
$(k',\xi',\sigma',\rho'_1,\rho'_2),v'$ as ``output'', are updated versions of the above.
\end{itemize}
This is simultaneously defined with checking a list of patterns in figure \ref{fphase1}. 
Unification is used in the rule $\chkpcon$ for checking a constructor pattern.

\subsection{Checking inaccessible patterns}

After the accessible part of the patterns has been checked, now the inaccessible patterns have to be checked.
Checking a inaccessible pattern $\inacc{e}$ is described in figure \ref{fphase2}: 
The expression may contain any pattern variable of the clause. 
It is checked that the value inferred during the first phase is equal to the expression $e$ written down by the user.

\begin{figure}[htp]
\begin{gather*}
% VAR
\nru{\chkpvar}
{v_B = \wh \cl{b}{\ins{\rho}{x}{k}} \qquad \rho'_1 =\ins{\rho_1}{y}{k} \qquad \rho'_2 = \ins{\rho_2}{y}{v_A}}
{
\checkp
{k,\xi,\sigma,\rho_1,\rho_2}
{y}
{\vpi \: x \: v_A \: \cl{b}{\rho}}
{k+1,\xi,\sigma,\rho'_1,\rho'_2}
{v_B}}
% CON
\\[2ex]
\nxru{\chkpcon}
{\Sig \: {\cc}=v_c \qquad
\checkps{k,\xi,\sigma,\rho_1,\rho_2}
{\vec{p}}
{v_c}
{k',\xi',\sigma',\rho_1',\rho_2'}
{v'_c}
}
{\unify{k}{\xi}{v'_c}{v_A}{\sigma_2}
\qquad
v_p = \ptv{k}{p}    
\qquad
\sigma'' = \compS{\sigma'}{\sigma_2}
}
{
\rho''_2 = \substEnv{\sigma''}{\rho'_2} \qquad
v_B = \subst{\sigma''}{\wh \cl{b}{\ins{\rho}{x}{v_p}}}
}
{\checkp
{k,\xi,\sigma,\rho_1,\rho_2}
{\cc \: \vec{p}}
{\vpi \: x \: v_A \: \cl{b}{\rho}}
{k',\xi',\sigma'',\rho'_1,\rho''_2}
{v_B}
}
% DOT
\\[2ex]
\nru{\chkpinacc}
{v_B = \wh \cl{b}{\ins{\rho}{x}{k}} , \xi' = \xi \cup \{ (k,(e,v_A)) \} } 
{
\checkp
{k,\xi,\sigma,\rho_1,\rho_2}
{\inacc{e}}
{\vpi \: x \: v_A \: \cl{b}{\rho}}
{k+1,\xi',\sigma,\rho_1,\rho_2}
{v_B}}
\\[2ex]
\nru{}
{}
{
\checkps
{k,\xi,\sigma,\rho_1\rho_2}
{\diamond}
{v}
{k,\xi,\sigma,\rho_1,\rho_2}
{v}}
\\[2ex]
\ndru{}
{\checkp
{k,\xi,\sigma,\rho_1,\rho_2}
{p}
{v}
{k',\xi',\sigma',\rho'_1,\rho'_2}
{v'}
}
{\checkps
{k',\xi',\sigma',\rho'_1,\rho'_2}
{\vec{p}}
{v'}
{k'',\xi'',\sigma'',\rho''_1,\rho''_2}
{v''}
}
{
\checkps
{k,\xi,\sigma,\rho_1\rho_2}
{p \: \vec{p}}
{v}
{k'',\xi'',\sigma'',\rho''_1,\rho''_2}
{v''}}
\end{gather*}
\caption{checking accessible patterns}
\label{fphase1}
\end{figure}

\begin{figure}[htp]
\begin{gather*}
\ndru{}
{
\Check{k}{\rho_1}{\rho_2}{e}{\subst{\sigma}{v_t}}
}
{
(i,v_i) \in \sigma \qquad \EqVal{k}{\nF}{\wh \cl{e}{\rho_1}}{v_i}
}
{
\checkinacc
{k,\sigma,\rho_1,\rho_2}
{i}{e}{v_t}
}
\end{gather*}
\caption{Checking inaccessible pattern}
\label{fphase2}
\end{figure}

%\newpage

\subsection{Checking the whole declaration}

Now, only the right hand side is left to be checked. It is type-checked against the type remaining from the first phase. 
The complete algorithm for type-checking a mutual function declaration follows: 
\subsubsection{Algorithm}
\begin{quote}
For a mutual declaration $\delta = $ 
\begin{bsp}
$\mutual$\\
$\spcx \fun f_1 : A_1 $\\
$\spcx \spc \vec{\gamma}_1$\\
$\spcx \ldots$\\
$\spcx \fun f_n : A_n $\\
$\spcx \spc \vec{\gamma}_n$
\end{bsp}
and a type-correct signature $\Sig$, if 
\begin{enumerate}
\item
for every $i \in \{1 \ldots n \} $
\begin{enumerate}
\item
$\IsType{1}{\diamond}{\diamond}{t_i}$
\item
$v_{i} := \wh \cl{A_i}{\diamond}$
\end{enumerate}
\item
$\Sig' := \Sig \cup \displaystyle\bigcup_{i \in \{1 \ldots n\}}\{f_i \mapsto (v_{i},\vec{\gamma}_i,\bot)\}$
\item
Using $\Sig'$, for every $i \in \{ 1 \ldots n \} $ and every clause  $ f \vec{p} \: e \: \in \vec{\gamma}_i$:
\begin{enumerate}
\item
$\checkps{1,\diamond,\diamond,\diamond,\diamond}{\vec{p}}{v_i}{k,\xi,\sigma,\rho_1,\rho_2}{v}$
\item
$\checkinacc{k,\sigma,\rho_1,\rho_2}{i}{e}{v_t}\: $ for all $(i,(e,v_t)) \in \xi$ 
\item 
$\Check{k}{\rho_1}{\rho_2}{e}{v}$
\end{enumerate}
\item
$\Sig'' := \Sig \cup \displaystyle\bigcup_{i \in \{1 \ldots n\}} \{f_i \mapsto (v_{_i},\vec{\gamma_i},\top)\}$
\end{enumerate}
then $\Sig''$ is a type-correct signature.
\end{quote}
A mutual $\cofun$ declaration is type-checked just the same.

%\clearpage

\section{Mugda programs}
\label{tca}
The previous sections detailed algorithms for all declarations of $\mugda$. 
We can apply those to a list of declarations:
\paragraph*{Algorithm}
\begin{quote}
For a $\mugda$ program $ \vec{\delta} = \delta_1 \ldots \delta_n$, if
\begin{itemize}
\item
for every $i \in \{ 1 \ldots n \}$: $\Sig_i$ is the result of type-checking $\delta_i$ using $\Sig_{i-1}$
\end{itemize}
then $\vec{\delta}$ is a type-correct $\mugda$ program.
\end{quote}
So for a whole $\mugda$ program, the declarations are checked one by one, starting with the empty signature $\Sig_0$.

\chapter{Termination-Checking}

\section{Motivation}
The type-checking algorithm presented in the previous chapter has one drawback:
The function declaration
\begin{bsp}
$\fun \foo : \Nat \ra \Nat$\\
$\spc \foo x = \foo x$
\end{bsp}
passes the type checker, although the function denoted by $\foo$ is not well-defined.
Also, now the declaration
\begin{bsp}
$\clet v : \vVec \Nat (\foo \zero) = \nil \Nat \zero$
\end{bsp}
is neither rejected nor accepted, because the evaluation of $\foo \zero$ is undefined.
For an implementation, this means that the type checker itself never finishes, or perhaps crashes with a stack overflow.
Type-checking is not \emph{decidable}, because it is not enforced that evaluation is well-defined.


\subsubsection{Terminating closure}
A closure $\cl{e}{\rho}$ is \emph{terminating} if its evaluation is defined, i.e. there exists a value $v \in \Val$ such that $\wh \cl{e}{\rho}= v$. 

\subsubsection{Terminating signature}
A type-correct signature $\Sig$ is \emph{terminating}, if every closure that can be evaluated during type-checking with $\Sig$ is terminating. 

\subsubsection{Proposition: decidable type-checking}
Type-checking of a $\mugda$ program $\delta_i \ldots \delta_n$ is decidable if all signatures $\Sig_i$ that arise during type-checking (see section {\ref{tca}}) are terminating. 

\subsubsection{Proposition}
\begin{itemize}
\item
The empty signature $\Sig_0$ is terminating.
\item
Given
\begin{itemize}
\item
$\delta$ either a $\clet$ , $\data $ or $\codata$ declaration
\item
$\Sigma$ a terminating signature 
\item
$\Sigma'$ the signature resulting from type-checking $\delta$ with $\Sigma$. 
\end{itemize}
then $\Sigma'$ is terminating. 
\end{itemize}
\noindent Only mutual (co)recursive declarations can possibly cause evaluation to be undefined.
What is left to do in this chapter is to give a criterion that rejects all those declarations that would result in a signature that is not terminating. Of course, any such criterion will also reject valid ones, as the \emph{halteproblem} is undecidable.

\section{Matrix notation}

\newcommand{\diag}{\opn{diag}}

For a set $R$ we write $R^{m,n}$ for the set of matrices with $m$ rows and $n$ columns and elements in $R$.
We write $\alpha_{i,j}$ for the element in the $i$th row and $j$th column.
For a matrix $\alpha \in R^{m,n}$, the dimension is $\vert\alpha\vert := (m,n)$.
For a square matrix $\beta \in R^{m,m}$, the set of diagonal elements is $\diag \beta := \{ \beta_{i,i} \mid i \leq m\}$.

\newcommand{\order}{\opn{ORDER}}

\section{Relating pattern and expressions}


The size-change principle \cite{lee01sizechange} states:
\begin{quote}
a program terminates on all inputs if every infinite call sequence would cause an infinite descent in some data values.
\end{quote}

So to see that an evaluation in our semantics is terminating, we need to know how the size of the \emph{semantical} values change during evaluation of recursion.
This is done by analyzing the \emph{syntax} of the program: how do the expression of the arguments from a recursive call relate to the patterns on the right hand side ?

This section is mostly based on \cite{abelAltenkirch:predStRec}, but where their work constructs a lexicographic order to show termination, we will use a simpler and yet more powerful criterion based on the size change principle, which is presented in the following section.

\paragraph*{Order}
\[ \order = \{ \: ? \: , \: \leq \: , \: < \: \} \]
\noindent This set denotes the possible results when comparing a expression $e$ to a pattern $p$, with the intended meaning:
\begin{itemize}
\item
$e < p$ : the value represented by $e$ is \emph{smaller} than that of $p$.
\item
$e \leq p $: we do not have $e_1 < e_2$, but the value represented by $p$ is not smaller than that of $e$.
\item
$e \: ? \: p $ : the value represented by $p$ is smaller than $e$ or their relation is not known.
\end{itemize}
Our order will be based on two rules:
\begin{itemize}
\item
\textbf{Axiom 1}: $x \: < \: \cc \: \vec{p_1} \: x \: \vec{p_n} $ when $\cc$ is a inductive constructor
\item
\textbf{Axiom 2}: $f \: \vec{e} \: \leq \: f $
\end{itemize}
The first rule expresses that $x$ is a structural part of $\cc \: \vec{p_1} \: x \: \vec{p_n}$.
The second rule is essential for higher order data types like $\Ord$.


\subsubsection{Order multiplication}
can be seen as serial composition:
\[
\begin{array}{l | c |c | c}
 *   &  <  &  \leq & ? \\
\hline
  <   & <    & <    &   ? \\
 \leq & <    & \leq &   ?  \\
 ?    & ?    & ?    &   ?
\end{array}
\]

\subsubsection{Order Addition}
can be seen as parallel composition, or as the maximum:
\[
\begin{array}{l | c |c | c }

 +    &  <   &  \leq   &  ? \\
 \hline 
 <    &  <   & <       &  <   \\
 \leq &  <   & \leq    &  \leq \\
 ?    &  <   & \leq    &  ?
\end{array}
\]

The triple $(\order,+,*)$ forms a commutative semiring, where $?$  is the neutral element of $+$ and $\leq$ is the neutral element of $*$.

\newcommand{\info}{\opn{min_o}}
\newcommand{\supo}{\opn{max_o}}

\subsubsection{Order minimum}
\[
\begin{array}{l | c |c | c}
\wedge &  <      &  \leq    & ? \\
\hline
  <    & <       &  \leq    &   ? \\
 \leq  & \leq    &  \leq    &   ?  \\
 ?     & ?       &  ?       &   ?
\end{array}
\]



\subsubsection{Maximum and minimum for list}
for a non-empty list $(o_1 \ldots o_n)$ we define as abbreviations
\[\supo \: (o_1 \ldots o_n) := o_1 \: + \: \ldots \: + \: o_n \]
and
\[\info \: (o_1 \ldots o_n) := o_1 \: \wedge \: \ldots \: \wedge \: o_n \]


\newcommand{\cmp}{\opn{cmp}}
\newcommand{\cmpv}{\opn{cmp_v}}
\newcommand{\cmps}{\opn{cmps}}
\newcommand{\cmpsv}{\opn{cmpsv}}
\newcommand{\etp}{\opn{etp}}
\newcommand{\extr}{\opn{extr}}

\newcommand{\ar}{\opn{ar}}

\subsubsection{Arity}
All clauses of a function declaration should have the same number of patterns.\\
The \emph{arity} of a function $\ff$  is $\ar(\ff) := \vert \vec{p} \vert $ where $\ff \: \vec{p} = e $ is some clause of $\ff$.   

\subsubsection{Call}
A \emph{call} is an expression of the form $\ff \: \vec{e}$.


\subsubsection{Expression to pattern}

As said, we want to compare an expression to a pattern, yielding an element of $\order$.
$\mugda$ mainly differs from simply typed languages by having the inaccessible patterns $\inacc{e}$.
We handle this kind of pattern by trying to convert the expression $e$ into a ``normal'' pattern:

\[
\begin{array}[t]{ll}
\multicolumn{2}{l}{\etp : \Expr \ra \Pat \cup \: \{ \: \stuck \: \} }\\ 
\etp \: x & = x \\
\etp \: (\cc \: e_1 \ldots e_n ) & = \cc \: (\etp e_1) \ldots (\etp e_n) \\
\etp \: \cc & = \cc \: \diamond \\
\etp \: e & = \: \stuck \mbox{ otherwise}
\end{array}
\]

\noindent Now the comparison of an expression to a pattern follows: 

\subsubsection{Comparing expression to pattern}
\[
\begin{array}[t]{ll}
\multicolumn{2}{l}{\cmp : \Expr \times \Pat \ra \order}\\ 
\cmp \: e_1 \: \inacc{e_2} & = \left\{ 
\begin{array}{ll}
\cmp \: e_1 \: p & \mbox{ if } \etp \: e_2 = p \neq \: \stuck \\
\: \leq \: & \mbox{ if } e_1 = e_2 \\
\: ? \:    & \mbox{ otherwise}
\end{array}\right.\\ 
\cmp \: x \: p & = \cmpv \: x \:p  \\
\cmp \: (x \: \vec{e}) \: p & = \cmpv \: x \:p  \\
\cmp \: \cc   \: (\cc \: \diamond) & = \: \leq \\
\cmp \: (\cc \: e_1 \ldots e_n) \: (\cc \: p_1 \ldots p_n) & = \info \: (\cmp \: e_1 \: p_n) \ldots (\cmp \: e_n \: p_n)\\

\cmp \: e \: p & = \: ? \mbox{ otherwise}
\end{array}
\]

where $\cmpv$ compares a variable to a pattern:

\subsubsection{Comparing variable to pattern}
\[
\begin{array}[t]{ll}
\multicolumn{2}{l}{\cmpv : \cV \times \Pat \ra \order}\\ 
\cmpv \: x \: x & = \: \leq \\
\cmpv \: x \: (\cc \: p_1 \ldots p_n) & = \: < \: * \: \supo \: (\cmpv \: x \: p_1) \ldots (\cmpv \: x \: p_n) \\
& \spc \: \mbox{ if } \cc \mbox{ is inductive constructor}\\
\cmpv \: x \: \inacc{e} & = \cmpv \: x \: p \mbox{ if } \etp \: e = p \neq \: \stuck \\
\cmpv \: x \: p & = \: ? \mbox{ otherwise}
\end{array}
\]


\noindent Note that coinductive objects are not well-founded, so we have 
\[\cmpv x \: (\cc \: \vec{p}) = \: ? \]
when $\cc$ is a coinductive constructor. 

\newcommand{\call}{\opn{CALL}}
\newcommand{\cm}{\opn{cm}}
\newcommand{\cs}{\opn{cs}}
\newcommand{\complete}{\opn{complete}}

\subsubsection{Call matrices}
A \emph{call matrix} is a triple $(\ff,\alpha,\fg)$ where $\ff,\fg \in \cF$ and $\alpha \in \order^{\ar(f),\ar(g)}$.
The set of all call matrices is $\call$.

The size-change principle is often presented with bipartite graphs instead of call matrices, but these are
equivalent representations.


\subsubsection{Call matrix of call}
Let $\ff \: \vec{p} = e$ be a function clause and $\fg \: \vec{e}\:$ be a call in $e$,
where $\vec{p} = p_1 \ldots p_n$ and $\vec{e} = e_1 \ldots e_m$.
The associated call matrix $\cm(\ff,\vec{p},\fg,\vec{e}) \in \call $ is given by 

\[\cm(\ff,\vec{p},\fg,\vec{e}) := (\ff,\alpha,\fg) \]


where $\alpha \in \order^{\ar(\ff),\ar(\fg)} $ with the elements:
\[ 
\alpha_{i,j} = \left\{ \begin{array}{ll}
    \cmp \: e_j \: p_i & \mbox{ if } j \leq m \\
    \: ?  & \mbox{ if } j > m
 \end{array}\right.
\]

    
In the definition of $\alpha$, it was taken into consideration that there can be calls $f e_1 \ldots e_n$ where $n < ar f$, because a function might be used as a higher-order argument. The missing arguments are filled up with $?$. 

\subsubsection{Call set}
A finite set of call matrices is called a \emph{call set}.
For a mutual declaration 
\[ \delta = \mutual (f_1 \: A_1 \: \gamma_1) \ldots (f_n \: A_n \gamma_n) \] 
we define the set $rec$ of recursive function identifiers
\[ rec := \{ f_1 , \ldots , f_n \} \]
and the \emph{initial call set} $\cs(\delta)$ as
\[ \cs(\delta) := \bigcup_{i \in \{1 \ldots n\}} \{ \extr \: rec \: f_i \: \vec{p} \: e \mid f_i \: \vec{p} \: e \in \gamma_i  \} \] 
The initial call set includes the call matrices of all recursive calls, where the extraction of call matrices from a right hand side is defined as follows:
\subsubsection{Extraction of call matrices from expression}
\[
\begin{array}[t]{ll}
\multicolumn{2}{l}{\extr : \pset{\cF} \times \cF \times \ves{\Pat} \times \Expr \ra \pset{\call}}\\ 
\extr \: rec \: \ff \: \vec{p} \quad (\fg \: e_1 \ldots e_n) & = \{ \cm (\ff,\vec{p},\fg,\vec{e}) \} \cup \bigcup_{i \in \{ 1 \ldots n \}} \extr \: rec \: \ff \: \: \vec{p} \: e_i \\
\extr \: rec \: \ff \: \vec{p} \quad \fg & =  \{ \cm (\ff,\vec{p},\fg,\diamond) \} \\
\extr \: rec \: \ff \: \vec{p} \quad (e \: e_1 \ldots e_n) & =  \extr \: rec \: \ff \: \vec{p} \: e \cup \bigcup_{i \in \{ 1 \ldots n \}} \extr \: rec \: \ff \: \vec{p} \: e_i\\ 
\extr \: rec \: \ff \: \vec{p} \quad \lam{x}e & = \extr \: rec \: \ff \: \vec{p} \: e \\
\extr \: rec \: \ff \: \vec{p} \quad \ELet{x}{A}{e_1}{e_2} & = \extr \: rec \: \ff \: \vec{p} \: e_1 \cup \extr \: rec \: \ff \: \vec{p} \: e_2 \\
\extr \: rec \: \ff \: \vec{p} \quad \EPi{x}{A}{B} & =  \extr \: rec \: \ff \: \vec{p} \: A \cup \extr \: rec \: \ff \: \vec{p} \: B \\ 
\extr \: rec \: \ff \: \vec{p} \quad e & = \emptyset \mbox{ otherwise}   
\end{array}
\]


\section{Applying the size-change principle}

Starting from the initial call set, the following allows reasoning about the possibility of infinite call sequences:

\subsubsection{Order matrix multiplication}
\[ 
\begin{array}[t]{l}
\times : O^{m,n} \times O^{n,p} \ra O^{m,p} \\
   (\alpha \times \beta)_{i,j} = \displaystyle\sum^{m}_{k=1}{\alpha_{i,k} * \beta_{k,j}}
\end{array}
\]

\subsubsection{Call Matrix composition}
\[(f,\alpha,g) \star (g,\beta,h) := (f,\alpha \times \beta,h)\]


\subsubsection{Call set completion}
\emph{Completing} a call set is closing it under composition.
This can be achieved with a fixed point algorithm:
\[
\begin{array}[t]{ll}
\multicolumn{2}{l}{\complete : \pset{\call} \ra \pset{\call}}\\ 
\complete \: cs  & = \left\{ \begin{array}{ll} 
    \complete \: cs' & \mbox{if } cs \neq cs'\\
     cs' & \mbox{if } cs = cs'  
     \end{array}\right.\\
\multicolumn{2}{l}{\spc \mbox { where } cs' = cs \cup \{ (f,\alpha,g) \star (g',\beta,h) \mid g = g' \mbox{ and } (f,\alpha,g) , (g',\beta,h) \in cs \}} \\
\end{array}
\]
\noindent For a declaration $\delta$,  $\complete \cs (\delta)$ is the \emph{completed call set} of $\delta$.

\subsubsection{Idempotent call matrix}
A call-matrix $(f,\alpha,g)$ is \emph{idempotent} if $ f = g $ and $ \alpha \times \alpha = \alpha $ 

\subsubsection{Decreasing call matrix}
The element $< \: \in O$ is \emph{decreasing}.\\
A call matrix $(f,\alpha,g)$ is \emph{decreasing} if $\diag(\alpha)$ contains a decreasing element.


\subsubsection{Proposition: Size-change principle for Mugda}
We can now give a criterion based on the size-change principle \cite{lee01sizechange}:
\begin{quote}
Given
\begin{enumerate}
\item
a terminating signature $\Sig$
\item
a mutual recursive function declaration $\delta$
\item
$\Sig'$ is the signature resulting from type-checking $\delta$ in $\Sig$
\item
every idempotent call-matrix  $ \alpha \in \complete \: \cs(\delta)$ is decreasing.
\end{enumerate}
then $\Sigma'$ is terminating.
\end{quote}
\noindent The size-change principle is proven to be correct in \cite{wahlstedt:phd:2007} for a dependently typed language 
with simpler \emph{first order data types}.
The work in \cite{abelAltenkirch:predStRec} contains a proof of correctness for the structural order defined by axioms 1 and 2 in a simply typed language. 

Productivity of corecursive declarations will follow in the next chapter, when the Size type is added to $\mugda$.

\section{Examples}

\subsection{Addition}

Recall the addition function on natural numbers:

\begin{bsp}
$\fun \add : \Nat \ra \Nat \ra \Nat$\\
$\spc \add x \: \zero = x $\\
$\spc \add x \: (\suc y) = \suc (\add x \: y)  $
\end{bsp}
The single recursive call is $\add x \: y$ in the second clause.
To build the call matrix we need to calculate:
\begin{quote}
$\cmp \: x \: x = \: \leq $\\
$\cmp \: x \: (\suc y) = \: ? $\\
$\cmp \: y \: x = \: ? $\\
$\cmp \: y \: (\suc y) = \: < $ 
\end{quote}

\noindent The call set is a singleton set $\{ \alpha \}$ where
\[
\alpha = 
\add\begin{pmatrix}
\leq & ? \\
?    & < \\
\end{pmatrix}\add
\]
\noindent To complete the call set, $\alpha$ is composed with itself:

\[
\add\begin{pmatrix}
\leq & ? \\
?    & < \\
\end{pmatrix}\add
\star
\add\begin{pmatrix}
\leq & ? \\
?    & < \\
\end{pmatrix}\add
=
\add\begin{pmatrix}
\leq & ? \\
?    & < \\
\end{pmatrix}\add
\]

\noindent which yields again $\alpha$. 
Thus, $\{ \alpha \}$ is the completed call set.
$\alpha$ is an idempotent matrix, and because it is decreasing, the declaration for $\add$ is accepted.

\noindent Now suppose the arguments in the recursive call are \emph{permuted}:
\begin{quote}
$\fun \addt : \Nat \ra \Nat \ra \Nat$\\
$\spc \addt x \: \zero = x $\\
$\spc \addt x \: (\suc y) = \suc (\addt y \: x)  $
\end{quote}
which still computes the addition of two natural numbers.
\noindent The initial call set is  
\[
\beta_1 = \addt\begin{pmatrix}
? & \leq \\
< & ?   \\
\end{pmatrix}\addt
\]
The completed call set is $\{\beta_1,\beta_2,\beta_3\}$ where 
\[
\beta_2 = \addt \begin{pmatrix}
< & ?    \\
? & <    \\
\end{pmatrix} \addt
\:,\:
\beta_3 = \addt \begin{pmatrix}
? & <    \\
< & ?    \\
\end{pmatrix} \addt
\] 
The only idempotent matrix $\beta_2$ is decreasing, so the definition of $\addt$ is also accepted.
Permuted arguments are a strength of the size-change principle.

\subsection{Mutual even and odd}

The following mutual declaration 
\begin{bsp}
$\mutual$\\
$\spc \fun \even : \Nat \ra \Bool$\\
$\spc \spc \even \zero = \ttt $\\
$\spc \spc \even (\suc x) = \odd x $\\
$\spc \fun \odd : \Nat \ra \Bool$\\
$\spc \spc \odd \zero = \fff$\\
$\spc \spc \odd (\suc x) = \even x$
\end{bsp}
yields the initial call set  
\[
\left\{
\even \begin{pmatrix}
< \\
\end{pmatrix}\odd
, 
\odd \begin{pmatrix}
< \\
\end{pmatrix}\even
\right\}
\] 
\noindent The completed call set
\[
\left\{
\even \begin{pmatrix}
< \\
\end{pmatrix}\odd
, 
\odd \begin{pmatrix}
< \\
\end{pmatrix}\even
,
\even \begin{pmatrix}
< \\
\end{pmatrix}\even
, 
\odd \begin{pmatrix}
< \\
\end{pmatrix}\odd
\right\}
\] 
includes two additional idempotent matrices that are both decreasing.
Thus this mutual definition is also accepted.
\subsection{Brouwer ordinals}
The so-called Brouwer ordinals can be defined with
\begin{bsp}
$\data \Ord : \Set$ \\
$\spc \ozero : \Ord $\\
$\spc \olim : (\Nat \ra \Ord ) \ra \Ord $
\end{bsp}
This is an example of a higher-order data type, because the argument to $\olim$ contains a function space.
Now ordinal addition can be defined:
\begin{bsp}
$\fun \addOrd : \Ord \ra \Ord \ra \Ord$\\
$\spc \addOrd x \: \ozero = x $\\
$\spc \addOrd x \: (\olim f) = \olim ( \lam{y} \addOrd  x \: (f \: y))  $
\end{bsp}
The interesting comparison is $(f \: y)$ against $\olim f$.
We should have $f \: y \leq f$ (Axiom 2) and $\olim f < f $ (Axiom 1), thus with transitivity $ f\: y < \olim f$.
And indeed we get:
\begin{bsp}
$\cmp \: x \: x = \: \leq $\\
$\cmp \: x \: (\olim f) = \: ? $\\
$\cmp \: (f \: y) \: x = \: ? $\\
$\cmp \: (f \: y) \: (\olim f) = \cmpv \: f \: (\olim f) = \: < \: * \: \supo \: (\cmpv \: f \: f) = \: < \: * \: \leq \: = \: < $ 
\end{bsp}
The corresponding call set is
\[
\addOrd\begin{pmatrix}
\leq & ? \\
?    & < \\
\end{pmatrix}\addOrd
\]
and is then accepted just like $\add$.
\section{Extending the order}

\subsubsection{Motivation}

The following type
\begin{bsp}
$\data \NatP : \Set$\\
$\spc \np : \Nat \ra \Nat \ra \NatP $
\end{bsp}
denotes pairs of natural numbers. Now if one defines addition on pairs:
\begin{bsp}
$\fun \addp : \NatP \ra \Nat$\\
$\spc \addp (\np x \: \zero) = x$\\
$\spc \addp (\np x \: (\suc \: y)) = \suc (\addp (\np x \: y))$
\end{bsp}
\noindent but this does no longer pass the termination-check.
Too much information is lost when creating the call matrices for this single nested pattern.
For $\addp$, the call matrix is 
\[
\addp \begin{pmatrix}
\leq \\
\end{pmatrix}\addp
\]
which is idempotent but not decreasing.
\newcommand{\coll}{\mathrm{collapse}\:}

\newcommand{\opl}{\operatorname{ORDER_{+}}}

The plan is to do a finer comparison for nested patterns, where the result is not a single element, but a whole matrix of elements:
\subsubsection{Extended Order}
The set $\opl$ is defined recursively by   
\begin{itemize}
\item
$ \order \subseteq \opl$
\item
if $ M \subseteq \opl $ and $n \in \cN $ then $M^{ n,n} \subseteq \opl$
\end{itemize}

\subsubsection{Extended Order operations}
The operations $*$, $+$ and $\wedge$ given in the previous section are extended to $\opl$ by adding the following additional cases together with a new operation, $\coll$:

\subsubsection{Additional composition cases}
\[
\begin{array}{llll}
\alpha  & * &\beta & = \left\{ \begin{array}{ll}
                          \alpha \times \beta & \mbox{ if } \vert \alpha \vert = \vert \beta \vert \\
                          (\coll \alpha) * (\coll \beta) & \mbox{ otherwise}  
                          \end{array}\right.\\ 
\alpha  & * & ?     & = \: ? \\
\alpha  & * & \leq  & = \alpha \\
\alpha  & * & <     & = (\coll \alpha) \: * \: < \\
?       & * & \beta & = \: ? \\
\leq    & * & \beta & = \beta \\
<       & * & \beta & = \: < * \: (\coll \beta)
\end{array}
\]
\subsubsection{Additional addition cases}

\[
\begin{array}{llll}
\alpha  & + &\beta & = \left\{ \begin{array}{ll}
                          \gamma \mbox{ with } \gamma_{i,j} = \alpha_{i,j} \: + \: \beta_{i,j} & \mbox{ if } \vert \alpha \vert = \vert \beta \vert \\
                          (\coll \alpha) + (\coll \beta) & \mbox{ otherwise}  
                          \end{array}\right.\\ 
\alpha  & + & ?     & = \alpha \\
\alpha  & + & \leq  & = (\coll \alpha) \: + \: \leq \\
\alpha  & + & <     & = \: < \\
?       & + & \beta & = \beta \\
\leq    & + & \beta & = \: \leq + \: (\coll \beta)\\
<       & + & \beta & = \: < \\
\end{array}
\]

\subsubsection{Additional minimum cases}
\[
\begin{array}{llll}
\alpha  & \wedge &\beta & = \left\{ \begin{array}{ll}
                          \gamma \mbox{ with } \gamma_{i,j} = \alpha_{i,j} \: \wedge \: \beta_{i,j} & \mbox{ if } \vert \alpha \vert = \vert \beta \vert \\
                          (\coll \alpha) \wedge (\coll \beta) & \mbox{ otherwise}  
                          \end{array}\right.\\ 
\alpha  & \wedge & ?     & = \: ? \\
\alpha  & \wedge & \leq  & = (\coll \alpha) \: \wedge \: \leq \\
\alpha  & \wedge & <     & = \beta \\
?       & \wedge & \beta & = \: ? \\
\leq    & \wedge & \beta & = \: \leq \wedge \: (\coll \beta)\\
<       & \wedge & \beta & = \beta \\
\end{array}
\]

\subsubsection{Collapsing of matrix}
\[
\begin{array}{ll}
\multicolumn{2}{l}{\coll : M(\opl) \ra \opl}\\
\coll \alpha = \info \: (\diag \alpha) 
\end{array}
\]

\subsubsection{Comparison with extended order}
\noindent Now, the following clause of $\cmp$   
\[
\begin{array}[t]{ll}
\cmp \: (\cc \: e_1 \ldots e_n) \: (\cc \: p_1 \ldots p_n ) & = \: \info \: (\cmp \: e_1 \: p_1) \ldots (\cmp \: e_n \: p_n)\end{array}
\]
is changed to

\[
\begin{array}[t]{ll}
\cmp \: (\cc \: e_1 \ldots e_n) \: (\cc \: p_1 \ldots p_n ) & = 
\left\{\begin{array}{l} \gamma \in \opl^{n,n} \mbox{ with } \gamma_{i,j} = \cmp \: e_j \: p_i \\
\: \mbox{ if } n \geq 2 \\
\\
\cmp \: e_1 \: p_1 \mbox { otherwise}
\end{array}
\right.
\end{array}
\]
so more order information about the components is used, for as long as possible before the matrix might have to be collapsed.
\subsubsection{Decreasing element}
In the extended order, a decreasing element is recursively defined by
\begin{itemize}
\item
$<$ is decreasing
\item
$\alpha$ is decreasing if $\diag(\alpha)$ has at least one decreasing element.
\end{itemize}

The remaining operations from the previous section carry over directly to $\opl$.
\noindent Before giving examples, we conjecture the following:
\subsubsection{Proposition: Size-change principle for extended order}
The size change principle with $\opl$ for $\mugda$ holds.

\subsection{Examples}

\subsubsection{Pair addition}
Back to the example from the motivation, the initial call set calculated with the extended order is:
\[
\add_p\begin{pmatrix}
\begin{pmatrix}
\leq & ? \\
?    & < \\
\end{pmatrix}
\end{pmatrix}\add_p
\]

\noindent This is also the completed call set and the matrix is decreasing.

\subsubsection{List flattening}
Another example that now is recognized as terminating is this version of list flattening:
\begin{bsp}
$\fun \flatt : (A : \Set ) \ra \List (\List A) \ra \List A$\\
$\spc \flatt A \: (\nil \inacc{\List A}) = \nil A $\\
$\spc \flatt A \: (\cons \inacc{\List A} \: (\nil \inacc{A}) \: yl) = \flatt A \: yl $\\
$\spc \flatt A \: (\cons \inacc{\List A} \: (\cons \inacc{A} \: x \: xl) \: yl)  = \cons A \: x \: (\flatt A \: (\cons (\List A) \: xl \: yl))$
\end{bsp}
\noindent The initial set of call matrices is:
\[
\flatt \begin{pmatrix}
\leq & < \\
?    & < 
\end{pmatrix}\flatt
, 
\flatt \begin{pmatrix}
\leq & < \\
?    & 
\begin{pmatrix}
\leq &  ?  & ? \\
?    &  <  & ? \\
?    &  ?  & \leq 
\end{pmatrix}
\end{pmatrix}\flatt
\] 
Again, this is already the completed call set.
Both matrices are idempotent and decreasing.

\section{List reversion: Vectors to the rescue}
First, we present a peculiar list reversion algorithm found in \cite{blanqui04typebased}.
It is shown in figure \ref{lrev}. This mutual declaration is not accepted as terminating. 
For example the call 
\[\rev A \: (\revb A \: x \: xs)\]
in the third clause of $\revb$ induces the following call matrix:
\[
\revb \begin{pmatrix}
\leq & ? \\
\leq & ?  
\end{pmatrix}\rev
\] 
Indirect calls are not handled well by a purely syntactic test.
A termination-checker needs information about the behaviour of $\revb$, more precisely 
about the size of the returned list.
But the user can help by using vectors instead of lists: the same algorithm with vectors is shown in figure \ref{vref}. 
This declaration is now accepted, because in enough recursive calls the first argument is getting structurally smaller.

\begin{figure}[htp]
$\mutual $  \\
$ \spc \fun \rev : \EPi{A}{\Set}{\List A \ra  \List A} $ \\
$ \spc \spc \rev \inacc{A} \: (\nil A ) = \nil A$\\
$ \spc \spc \rev \inacc{A} \: (\cons A \:x  \: xs) = \cons A \: (\reva A \: x \: xs) \: (\revb A \: x \: xs)$\\
$\ \spc \fun \reva : \EPi{A }{\Set}{A  \ra \List A \ra A}$\\
$ \spc \spc \reva \inacc{A} \: a \: (\nil A ) = a$ \\
$\spc \spc \reva \inacc{A} \: a \: (\cons A \: x \: xs) = \reva A \: x \: xs$\\
$\spc \fun \revb : \EPi{A}{\Set}{ A \ra \List A \ra \List A} $\\
$\spc \spc \revb \inacc{A} \: a \: (\nil A ) = \nil A $\\
$\spc \spc \revb \inacc{A} \: a \: (\cons A\:  x \: xs) = \rev A \: (\cons A \: a \: (\rev A \: (\revb A \: x \: xs)))$
\caption{not accepted : reversion on Lists}  
\label{lrev}
\end{figure}

\begin{figure}[htp] 
$\mutual $ \\
$\spc \fun \rev : \EPi{n}{\Nat }{\EPi{A}{\Set }{\vVec A \: n \ra \vVec A \: n}}$\\
$\spc \spc \rev \inacc{\zero} \: \inacc{A}  \: (\nil A) = \nil A $\\
$\spc \spc \rev \inacc{\suc n} \: \inacc{A} \: (\cons A \: n \: x \: xs) = $\\
$\spcx \cons A \: n \: (\reva n \: A \: x \: xs) \: (\revb n \: A \: x \: xs) $ \\
$\spc \fun \reva : \EPi{n}{\Nat}\EPi{A}{\Set}{A \ra Vec A \: n \ra A}$\\
$\spc \spc \reva \inacc{\zero} \: \inacc{A} \: a \: (\nil A) = a $\\
$\spc \spc \reva \inacc{\suc n} \: \inacc{A} \: a \: (\cons A \: n \: x \: xs) = \reva n \: A \: x \: xs $\\
$\spc \fun \revb : \EPi{n}{\Nat}{\EPi{A}{\Set}{A \ra \vVec A \: n \ra \vVec A \: n}}$\\
$ \spc \spc \revb \inacc{\zero} \: \inacc{A}  \: a \: (\nil A) = \nil A $\\
$ \spc \spc \revb \inacc{\suc n} \: \inacc{A} \: a \: (\cons A \: n \: x \: xs) = $\\
$ \spcx \rev (\suc n) \: A \: (\cons A \: n \: a \: (\rev \: n \: A \: (\revb \: n \: A \: x \: xs))$
\caption{accepted: reversion on vectors}  
\label{vref}
\end{figure}  

\chapter{Sized data types}

As the $\revs$ example demonstrated, richer types allow more declarations to be accepted by the termination-checker. 
But natural numbers are not sufficient as the height for inhabitants of coinductive types like $\Streams$ or types with infinite branching like $\Ords$.
This is why a type $\Size$ will be added to $\mugda$ that has an element $\infty$ that is ``big enough'' for all objects in $\mugda$.
Inductive families that use this new type as an index will be called \emph{sized types}. 

In addition, using vectors instead of lists is not always easily possible.
Consider a $\filter$ function that removes some elements from a list.
For vectors, we would need to resort to existential types\footnote{for the use of existential types for termination, see \cite{xi99dependent}.}, which we will sketch as
\[ \begin{array}{l} 
\fun \filter : (A : \Set ) \ra (p : A \ra \Bool ) \ra (n : \Nat ) \\
\spcx \ra \vVec A \: n \ra (m : \Nat . (\: \tLeq m \: n \:, \: \vVec A \: m))
\end{array}\]   
where  $\filter$ returns a natural number $m$, a proof that $m$ is smaller than $n$, and a vector of length $m$.

The $\Size$ type will provide an easy way out. For inductive types, the size will be an upper bound 
on the height. Filtering on sized lists then has the type 
\[
\begin{array}{l} 
\fun \filter : (A : \Set ) \ra (p : A \ra \Bool ) \ra (i : \Size ) \ra \List A \: i \ra \List i 
\end{array}   
\]
and $\mugda$ will offer subtyping that is helpful when implementing $\filter$.

\section{Adding a Size type}

In most works on sized types, sizes annotations are merely ``tacked on'' to an existing language. An exception is \cite{abel:PhD}, where a \emph{size kind} is added to a polymorphic lambda calculus. 
As $\mugda$ already supports inductive families, adding a primitive type of sizes was the natural choice. 

\subsection{Syntax and semantics}

We extend expressions and patterns:
\[
\begin{array}{lrll}
\Expr \ni e,A,B & ::= & \ldots & \\
& \mid & \Size & \mbox{Size type} \\
& \mid & \s e  & \mbox{size successor}\\ 
& \mid & \infty & \mbox{limit size}\\
\end{array}
\]

\[
\begin{array}{lrll}
\Pat \ni p & ::= & \ldots & \\
& \mid & \s p & \mbox{size successor pattern}\\
\end{array}
\]

As $\Size$ is a type, in particular we also have variables of type $\Size$.
Thus, we can form size expressions of the form that was first developed in \cite{bgp:lpar06}.


\subsubsection{Semantics}

Values also are extended accordingly:
\[
\begin{array}{lrll}
\Val \ni v & ::= & \ldots & \\
& \mid & \Size & \mbox{Size type}\\
& \mid & \s v & \mbox{size successor}\\
& \mid & \infty & \mbox{size limit}\\
\end{array}
\]

We will now expand semantics for the size type. For the semantics, the size type can be imagined as a coinductive type with one constructor $\s$ and a distinguished inhabitant $\infty$.

\subsubsection{Size successor}
The semantics should take into account that $\infty$ is the limit size, i.e. the equation $\s \infty = \infty$ holds:
\[
\begin{array}[t]{ll}
\multicolumn{2}{l}{\sinfty : \Val \ra \Val}\\ 
\sinfty \: \infty & = \infty\\
\sinfty \:  v & = \s v  \mbox{ otherwise}\\ 
\end{array}
\]

\subsubsection{Evaluation}

\[
\begin{array}{ll}
\wh \cl{\Size}{\rho} & = \Size \\
\wh \cl{\infty}{\rho} & = \infty  \\
\wh \clp{\s e}{\rho} & = \sinfty \: v \mbox{ where } \: v = \wh \cl{e}{\rho} \\ 
\end{array}
\]



\subsubsection{Pattern matching}
\[
\begin{array}{ll}
\matchf \: \rho \: (\s p) \: \infty & = \match \: \rho \: p \: \infty \\
\matchf \: \rho \: (\s p) \: (\s v) & = \match \: \rho \: p \: v \\
\end{array}
\]

\subsection{Type-Checking}

First the following clauses are added:
\subsubsection{Application of substitution}
\[
\begin{array}{ll}
\subst{\sigma}{\s v} & = \sinfty \subst{\sigma}{v} \\
\end{array}
\]

\subsubsection{Converting pattern to value}
\[
\begin{array}{ll}
\ptvx \: k \: (\s p) & = (\s v,k') \mbox{ where } (v,k') = \ptvx \: k \: p\\
\end{array}
\]

\paragraph*{}
\noindent All additional typing rules are shown in figure \ref{fadd}.
Of note is that $\Size$ is not a small type.

\begin{figure}[htp]
\begin{gather*}
\nru{}{}{\IsType{k}{\rho_1}{\rho_2}{\Size}}
\qquad
%% Succ
\nru{\chksucc}
{\Check{k}{\rho_1}{\rho_2}{e}{\Size}}
{\Check{k}{\rho_1}{\rho_2}{\s e}{\Size}} 
%% Infty
\\[2ex]
\nru{\infinfty}{}{\Infer{k}{\rho_1}{\rho_2}{\infty}{\Size}} 
%% Succ
\\[2ex]
\nru{\eqsuc}
{\EqVal{f}{k}{v_1}{v_2}}
{\FeqVal{f}{k}{\s v_1}{\s v_2}}
% SUCC
\\[2ex]
\nru{}
{\nocc{k}{a}{v}}
{\nocc{k}{a}{\s v}}
\qquad
\nru{}
{\spos{k}{a}{v}}
{\spos{k}{a}{\s v}}
\\[2ex]
\ndru{\chkpsucc}
{
\checkp{k,\xi,\sigma,\rho_1,\rho_2}{p}{\Size \ra \Size}
{k',\xi',\sigma',\rho'_1,\rho'_2}{v'}}
{v = \ptv{k}{p} }
{
\checkp
{k,\xi,\sigma,\rho_1,\rho_2}
{\s p}
{\vpi \: x \: \Size \: \cl{b}{\rho}}
{k',\xi',\sigma',\rho'_1,\rho'_2}{\wh \cl{b}{\ins{\rho}{x}{v}}}}
\\[2ex]
\nru{\unifys}
{\unify{k}{\xi}{v_1}{v_2}{\sigma}}
{\unify{k}{\xi}{\s v_1}{\s v_2}{\sigma}
}
\\[2ex]
\nru{\unifyi}
{\unify{k}{\xi}{v_1}{\infty}{\sigma}}
{\unify{k}{\xi}{\s v_1}{\infty}{\sigma}
}
\end{gather*}
\caption{Additional typing rules for Size type}
\label{fadd}
\end{figure}

\subsection{Termination-Checking}

We now detail the additions to chapter 4:

\subsubsection{Expression to pattern}
\[
\begin{array}[t]{ll}
\etp \: (\s e) = \s (\etp e) \\
\end{array}
\]

\subsubsection{Comparing expression to pattern}
\[
\begin{array}[t]{ll}
\cmp \: (\s e_1) \: (\s e_2) & = \cmp \: e_1 \: e_2 \\
\end{array}
\]

\subsubsection{Comparing variable to pattern}
the size successor is treated like an \emph{inductive} constructor:
\[
\begin{array}[t]{ll}
\cmpv \: x \: (\s p) & = \: < * \: \cmpv \: x \: p\\
\end{array}
\]

This is quite dangerous, as the $\Size$ type is not well-founded.
Later the use of $\Size$ must be controlled so that termination-checking remains valid.

\section{Sized data type declarations}

\newcommand{\sizecon}[4]{#1\vdash#2\mbox{ \textbf{sizeCon} }#3\:#4}
\newcommand{\sizeuse}[4]{#1\vdash#2\mbox{ \textbf{sizeUse} }#3\:#4}

Two new declarations are added to the language:
\subsubsection{Declarations}
\[
\begin{array}{lrll}
\Decl \ni \delta & ::= & \ldots & \\
& \mid & \sized \data \DD \: \tau : A \: \vec{\gamma} & \mbox{sized inductive data type}\\ 
& \mid & \sized \codata \DD \: \tau : A \: \vec{\gamma} & \mbox{sized co inductive data type} \\
\end{array}
\]

\subsection{Examples}


\subsubsection{Sized natural numbers}

\begin{bsp}
$\sized \data \SNat : \Size \ra \Set $ \\
$\spc \zero : \EPi{i}{\Size}{ \SNat \: (\s i)} $\\
$\spc \suc : \EPi{i}{\Size}{ \SNat \: i \ra \SNat \: (\s i)} $
\end{bsp}

\noindent The following declaration
\begin{bsp}
$\clet 1 : \SNat \infty = \suc \: \infty \: (\zero \infty)$ 
\end{bsp}
\noindent defines the natural number $1$. So ``at run-time'' all objects of a sized type get the height $\infty$,
which is the only closed expression of type $\Size$. We also introduce the type of sized streams:
\subsubsection{Sized Streams}
\begin{bsp}
$\sized \codata \SStream : \Size \ra \Set $ \\
$\spc \cons : \EPi{i}{\Size}{\Nat \ra \SStream i \: \ra \SStream (\s i) } $
\end{bsp}
Next is type-checking of the new sized declarations. 
\subsection{Checking sized data type declarations}

A sized data declaration need to be of the syntactic scheme outlined in figure \ref{ssyn}:

\begin{figure}[htp]
\centering{
\begin{minipage}{0.7\textwidth}
\begin{bsp}
$\sized \data \DD \: ( p_1 : P_1) \ldots ( p_n : P_n) : \Size \ra \Gamma \ra \Set $\\
$\spc \cc_1 : (i_1 : \Size) \ra \Delta_1 \ra \DD \: p_1 \ldots p_n \: (\s i_1) \: t_2 \ldots t_m$\\
$\spc \ldots$\\   
$\spc \cc_k : (i_k : \Size) \ra \Delta_k \ra \DD \: p_1 \ldots p_n \: (\s i_k) \: t_2 \ldots t_m$\\
\end{bsp}
\end{minipage}
}
\caption{Syntactic valid form of sized data declaration}
\label{ssyn}
\end{figure}

\noindent For non-sized data declarations, a size is not allowed in the indices $\Gamma$ because $\Size$ is not a small type.
But, for a sized declarations, it is allowed as the first index and, for every constructor, as the first argument.
We will check that every recursive argument in every $\Delta_j$ has the form \[\DD \: \ldots \: i \: \ldots \]
and that $i$ does not appear anywhere else (figure \ref{fsizecon}). The size argument should denote the height of an object when viewed as a tree, and every constructor increases this height.

\noindent In summary, the type-checking algorithm for sized data types follows: 
\subsubsection{Algorithm}
\begin{quote}
For the declaration $\delta = \sized \data \DD \: \tau : A \: \vec{\gamma} $ 
and a type-correct signature $\Sig$, if
\begin{enumerate}
\item
$ n := \vert\tau\vert$
\item
$\delta$ follows the form of figure \ref{ssyn}
\item
$\DataType{1}{\diamond}{\diamond}{n+1}{\tau \ra A}$
\item
$v_D := \wh \clp{\tau \ra A}{\diamond}$
\item
$\Sig' := \Sig \cup \{\DD \mapsto (v_D,n)\}$
\item
Given $\Sig'$, for every constructor declaration $\cc : B  \in \vec{\gamma}$:
\begin{enumerate} 
\item
$\ConType{1}{\diamond}{\diamond}{n+1}{\tau \ra B}$
\item
$v_i := \wh \clp{\tau \ra B}{\diamond}$
\item
$\sposc{1}{v_i}{j}$ for every $j \in \pos(\DD) $ 
\item
$\sposc{1}{v_i}{\DD}$
\item
$\sizecon{1}{v_i}{(n+1)}{\DD}$
\end{enumerate}
\item
$\Sig'' := \Sig' \cup \displaystyle\bigcup_{i}\{\cc_i \mapsto v_i\}$
\end{enumerate}
then $\Sig''$ is a type-correct signature.
\end{quote}

\begin{figure}[htp]
\begin{gather*}
\nru{}
{\nocc{k}{t_k}{i} \mbox { for all } k \in \{ 2 \ldots m\}}
{\sizeuse{k}{\DD \: p_1 \ldots p_n \: i \: t_2 \ldots t_m}{i}{\DD}}
{\quad \Sig \: \DD = (v_D,n)}
\\[2ex]
\ndru{}
{\sizeuse{k}{v}{i}{\DD}}
{\sizeuse{k}{v_1}{i}{\DD} \mbox { for all } k \in \{ 1 \ldots n\}}
{\sizeuse{k}{v \: v_1 \ldots v_n}{i}{\DD}}
\\[2ex]
\nru{}
{\sizeuse{k}{v}{i}{\DD}}
{\sizeuse{k}{\s v}{i}{\DD}}
\qquad
\ndru{}
{\sizeuse{k}{v_A}{i}{\DD}}
{\sizeuse{k+1}{\wh \cl{b}{\ins{\rho}{x}{k}}}{i}{\DD}}
{\sizeuse{k}{\vpi \: x \: v_A \: \cl{b}{\rho}}{i}{\DD}}
\\[2ex]
\nru{}
{\sizeuse{k+1}{\wh \cl{e}{\ins{\rho}{x}{k}}}{i}{\DD}}
{\sizeuse{k}{\vlam \: x \: \cl{e}{\rho}}{i}{\DD}}
\qquad
\nru{}
{a \neq i}
{\sizeuse{k}{a}{i}{\DD}}
\\[2ex]
\nxru{}
{\sizeuse{k}{v_A}{i}{\DD} \mbox{ if } k > i}
{v_A = \Size \mbox { if } k = i }
{\sizecon{k+1}{\wh \cl{b}{\ins{\rho}{x}{k}}}{i}{\DD}}
{\sizecon{k}{\vpi \: x \: v_A \: \cl{b}{\rho}}{i}{\DD}}
\\[2ex]
\ndru{}
{\nocc{k}{p_j}{i} \mbox { for all } j \in \{ 1 \ldots n\}}
{\nocc{k}{t_k}{i} \mbox { for all } k \in \{ 2 \ldots m\}}
{\sizecon{k}{\DD \:p_1 \ldots p_n \: (\s i) \:t_2 \ldots t_m}{i}{\DD}}
{\quad \Sig \: \DD = (v_D,n)}
\end{gather*}
\caption{constructor size check}
\label{fsizecon}
\end{figure}


\section{Sub-typing for size}

For an inductive sized type, the size index can be interpreted as an \emph{upper bound} on the height of its inhabitants.
An inhabitant of $\SNat i$ is also an inhabitant of $\SNat (\s i)$. For an inductive sized types like $\SNat$ the following should hold:
\begin{bsp}
$\SNat A\: i$ is a subtype of $\SNat A \: (\s i)$  
\end{bsp}
and
\begin{bsp}
$\SNat A\: i$ is a subtype of $\SNat A \: \infty$.
\end{bsp}
for every size $i$.

For $\SStream$ as a sized coinductive type, the type $\SStream \infty$ contains all fully defined streams.
Thus, the size can be interpreted as a \emph{lower bound} for the definedness of its inhabitants:
\begin{bsp}
$\SStream (\s i) $ is a subtype of $\SStream i$
\end{bsp}
and
\begin{bsp}
$\SStream \infty$ is a subtype of $\SStream i$
\end{bsp}
for every size $i$.



A partial order on values of type $\Size$ is given in figure \ref{substage}. \begin{figure}[tp]
\begin{gather*}
\nru{\sleqinfty}
{}
{\SzLeq{v}{\infty}}
\qquad
\nru{\sleqgen}
{}
{\SzLeq{k}{k}}
\\[2ex]
\nru{\sleqsucci}
{\SzLeq{v_1}{v_2}}
{\SzLeq{\s v_1}{\s v_2}}
\qquad
\nru{\sleqsuccii}
{\SzLeq{v_1}{v_2}}
{\SzLeq{v_1}{\s v_2}}
\end{gather*}
\caption{Size value comparison}
\label{substage}
\end{figure} As strict positivity implies positivity (monotonicity), the information about strictly positive parameters can be used for subtyping. The subtype relation in figure \ref{fsub} is defined, just like the equality relation, by two simultaneous judgments:
\[\LeqVal{f}{k}{v_1}{v_2} \subseteq \cF \times \cN \times \Val \times \Val \]
\[\FleqVal{f}{k}{v_1}{v_2} \subseteq \cF \times \cN \times\Val \times \Val \]
Of special interest are the rules:
\begin{itemize}
\item
$\leqappind$: subtyping for sized inductive type. 
\item
$\leqappco$: subtyping for sized coinductive type.
\item
$\leqappd$: subtyping for other data types.
\item
$\leqpi$: subtyping is \emph{contravariant} for the function domain. 
\end{itemize}
As the values ``switch places'' for the domain in the rule $\leqpi$, the force history $f$ is switched accordingly.
We define $\overline{\rF}=\lF$,  $\overline{\lF}=\rF$ and $\overline{\nF}=\nF$.

Subtyping is now replacing type equality in the rule $\chkinf$. 
The updated rule is shown in figure \ref{fnews}.

\begin{figure}[htp]
\begin{gather*}
%% Force
\nxru{\leqforce}
{\FleqVal{\lF}{k}{\forc \: v_1}{v_2} \qquad \mbox{if } \forc \: v_1 \neq v_1 , \forc \: v_2 = v_2  , f \neq \rF
}
{\FleqVal{\rF}{k}{v_1}{\forc \: v_2} \qquad \mbox{if } \forc \: v_1 = v_1 , \forc \: v_2 \neq v_2 , f \neq \lF }
{\FleqVal{f}{k}{v_1}{v_2} \qquad \qquad \qquad \qquad \qquad \mbox{ otherwise}}
{\LeqVal{f}{k}{v_1}{v_2}}
\\[2ex]
%% Ind Data
\nxru{\leqappind}
{\mbox{for all } i \in \pos(\DD): \LeqVal{f}{k}{p_i}{q_i} 
}
{
\mbox{for all } j \notin \pos(\DD): \EqVal{f}{k}{p_j}{q_j} 
}
{\mbox{for all } k \in \{1,..m\}: \EqVal{f}{k}{v_k}{w_k}
\qquad
\SzLeq s t }
{\FleqVal{f}{k}{\DD \: p_1 \ldots p_n \: s \: v_1 \ldots v_m}{\DD \: q_1 \ldots q_n \: t \: w_1 \ldots w_m}}
{\: \Sig \DD = (v_D,n)}
%% Co Data
\\[2ex]
\nxru{\leqappco}
{\mbox{for all } i \in \pos(\DD): \LeqVal{f}{k}{p_i}{q_i} 
}
{
\mbox{for all } j \notin \pos(\DD): \EqVal{f}{k}{p_j}{q_j}
}
{\mbox{for all } k \in \{1,..m\}:\EqVal{f}{k}{v_k}{w_k}
\qquad
\SzLeq t s }
{\FleqVal{f}{k}{\DD \: p_1 \ldots p_n \: s \: v_1 \ldots v_m}{\DD \: q_1 \ldots q_n \: t \: w_1 \ldots w_m}}
{\: \Sig \DD = (v_D,n)} 
%% Data
\\[2ex]
\nxru{\leqappd}
{\mbox{for all } i \in \pos(\DD): \LeqVal{f}{k}{p_i}{q_i} 
}
{
\mbox{for all } j \notin \pos(\DD): \EqVal{f}{k}{p_j}{q_j}}
{\mbox{for all } k \in \{1,..m\}:\EqVal{f}{k}{v_k}{w_k}}
{\FleqVal{f}{k}{\DD \: p_1 \ldots p_n \: v_1 \ldots v_m}{\DD \: q_1 \ldots q_n \: w_1 \ldots w_m}}
{\:\Sig \DD = (v_D,n) }
%%App
\\[2ex]
\nru{\leqapp}
{\LeqVal{f}{k}{v}{w} \qquad
\mbox{for all } k \in \{1,..m\}:\EqVal{f}{k}{v_k}{w_k}}
{\FleqVal{f}{k}{v \: v_1 \ldots v_n}{w \: w_1 \ldots w_n}}
%% Pi
\\[2ex]
\nru{\leqpi}
{\LeqVal{\overline{f}}{k}{v_2}{v_1}
\qquad
\LeqVal{f}{k+1}{\wh \cl{b_1}{\ins{\rho_1}{x_1}{k}}}{\wh \cl{b_2}{\ins{\rho_2}{x_2}{k}}}}
{\FleqVal{f}{k}{\vpi \: x_1 \: v_1 \: \cl{b_1}{\rho_1}}{\vpi \: x_2 \: v_2 \: \cl{b_2}{\rho_2}}}
%% Lam
\\[2ex]
\nru{\leqlam}
{\LeqVal{f}{k+1}{\wh \cl{e_1}{\ins{\rho_1}{x_1}{k}}}{\wh \cl{e_2}{\ins{\rho_2}{x_2}{k}}}}
{\FleqVal{f}{k}{\vlam \: x_1 \: \cl{e_1}{\rho_1}}{\vlam \: x_2 \: \cl{e_2}{\rho_2}}}
\\[2ex]
\nru{\leqsuc}
{\LeqVal{f}{k}{v_1}{v_2}}
{\FleqVal{f}{k}{\s v_1}{\s v_2}}
%% Atom
\qquad
\nru{\leqatom}
{}
{\FleqVal{f}{k}{a}{a}}
\end{gather*}
\caption{subtype checking}
\label{fsub}
\end{figure}

\begin{figure}[htp]

\begin{gather*}
%% Infer-mode
\nru{\chkinf}
{\Infer{k}{\rho_1}{\rho_2}{e}{v_2}
\qquad
{\LeqVal{\nF}{k}{v_2}{v_1}}} 
{\Check{k}{\rho_1}{\rho_2}{e}{v_1}}
\end{gather*}
\caption{updated type-checking rule for subtyping}
\label{fnews}
\end{figure}

\section{Examples: sized inductive types }

\subsection{Natural number division}

The minus function can be defined by:
\begin{bsp}
$\fun \minus : (i : \Size ) \ra \SNat i \ra \SNat \infty \ra \SNat i$\\
$\spc \minus \inacc{\s i} \: (\zero i) \:  y = \zero i $ \\
$\spc \minus i \: x \: (\zero \inacc{\infty})  = x $\\
$\spc \minus \inacc{\s i} \: (\suc \: i \: x) \: (\suc \inacc{\infty} \: y) = \minus i \: x \: y $
\end{bsp}
Subtyping is used in the right hande side of the last clause.
While $\minus$ is structurally recursive even without a size, 
there is more information in the type: the size of the result list now has an upper bound $i$.
This is essential for the following definition of division on natural numbers:
\begin{bsp}
$\fun \div : ( i : \Size )  \ra  \SNat i \ra \SNat \infty \ra \SNat i$\\
$\spc \div \inacc{\s i} \: (\zero i) \: y = \zero i$\\
$\spc \div \inacc{\s i} \: (\suc i \: x) \: (\zero \inacc{\infty})  = \zero i$\\
$\spc \div \inacc{\s i} \: (\suc i \: x) \: (\suc \inacc{\infty} \: y) = $\\
$\spc \spcx \suc i \: (\div i \: (\minus i \: x \:y) \: (\suc \infty \: y))$
\end{bsp}
$\div$ is structural recursive on the size argument.
The initial call set is the singleton matrix
\[
\div
\begin{pmatrix}
<    &  <  & ? \\
?    &  ?  &  ? \\
?    &  ?  & \leq 
\end{pmatrix}\div
\] 
which is idempotent and decreasing.

\clearpage

\subsection{Sized Lists}
\begin{bsp}
$\sized \data \SList ( A : \Set ) : \Size \ra \Set $ \\
$\spcx \nil : \EPi{i}{\Size}{ \SList A \: (\s i)} $\\
$\spcx \cons : \EPi{i}{\Size}{ A \ra \SList A \: i \ra \SList A \: (\s i)} $
\end{bsp}

\subsubsection{Quicksort}
This version of the well-known sorting algorithm was adapted from \cite{abel:rairo04} is shown in figure \ref{fsort}.

\begin{figure}[htp]
\centering{
\begin{minipage}{0.8\textwidth}
\begin{bsp}
$\data \ProdT ( + A : \Set) : \Set$\\
$\spc  \prd : A \ra A \ra \ProdT A$\\
$\fun \pra : ( A : \Set ) \ra \ProdT A\: \ra A$\\
$\spc \pra \inacc{A}\: (\prd A\: a \: b) = a$\\
$\fun \prb : ( A : \Set ) \ra \ProdT A\: \ra A$\\
$\spc \prb \inacc{A}\: (\prd A\: a \: b) = b$

$\fun \pivot : (i : \Size ) \ra (A : \Set ) \ra ( leq : A \ra A \ra \Bool ) $\\
$\spcx	\spcx \ra A \ra \SList A \: i \ra \ProdT (\SList A \: i)$\\
$\spc \pivot \inacc{\s i} \: \inacc{A} \: leq \: a \: (\nil A \: i) = \prd (\SList \: A \: (\s i)) \: (\nil A \: i) \: (\nil A \: i)$\\
$\spc \pivot \inacc{\s i} \: \inacc{A} \: leq \: a \: (\cons A \: i \: x \:xs) = $\\ 
$\spc \spc \ELet{rec}{\ProdT (\SList A \:i)}{\pivot i \: A \: leq \: a \: xs}$\\
$\spc \spc \ELet{l_1}{\SList A \:i}{\pra (\SList A \: i) \: rec}$\\
$\spc \spc \ELet{l_2}{\SList A \:i}{\prb (\SList A \: i) \: rec}$\\  
$\spc \spcx  \ite (\ProdT (\SList A \: (\s i))) \: (leq \: a \: x)  $\\
$\spc \spcx  \spc \prd (\SList A \: (\s i)) \: l_1 \: (\cons A \: i \: x \: l_2)$\\
$\spc \spcx  \spc \prd (\SList A \: (\s i)) \: (\cons A \: i \: x \: l_1) \: l_2$

$\fun \qsapp : (i : \Size ) \ra ( A : \Set ) \ra ( leq : A \ra A \ra \Bool) $\\
$\spcx \spcx \ra \SList A \: i \ra \SList A \: \infty \ra \SList A \: \infty$\\
$\spc \qsapp \inacc{\s i} \: \inacc{A} \: leq \: (\nil A \: i) \:             ys = ys$\\
$\spc \qsapp \inacc{\s i} \: \inacc{A} \: leq \: (\cons A \: i \: x \: xs) \: ys = $\\
$\spc \spc  \ELet{sl}{\ProdT (\SList A \:i)}{\pivot i \: A \: leq \: x \: xs}$\\
$\spc \spc  \ELet{l_1}{\SList A \: i}{\pra (\SList A \: i) \: sl}$\\
$\spc \spc  \ELet{l_2}{\SList A \: i}{\prb (\SList A \: i) \: sl}$\\
$\spc \spcx \qsapp i \: A \: leq \: l_1 \: (\cons A \: \infty \: x \: (\qsapp i \: A \: leq \: l_2 \: ys))$

$\clet \quicksort : (i : \Size ) \ra (A : \Set ) \ra (\lleq : A \ra A \ra \Bool)$\\ 
$\spcx \spcx \ra \SList A\: i \ra \SList A\: \infty$\\
$\spcx = \lam{i}\lam{A}\lam{leq}\lam{l} \qsapp i\: A\: leq \: l \: (\nil A\: \infty)$ 
\end{bsp}
\end{minipage}
}
\caption{Quicksort}
\label{fsort}
\end{figure}

The main routine $\quicksort$ is parameterized over a linear order $leq : A \ra A \ra \Bool$ on $A$.
The $\pivot$ function splits a list $l$ into a pair of list $l_1$ and $l_2$ where all smaller elements are in $l_1$ and the others in $l_2$.
The type of this function  
\[ (i : \Size ) \ra \ldots \ra \SList A \: i \ra \ProdT (\SList A \: i) \]
\noindent allows a rough upper bound on the sizes of both result lists.
This example shows how little effort sized types require to make some definitions accepted
to the termination check, but the price to pay is that the type of $\quicksort$ itself is 
\[ (i : \Size ) \ra \ldots \ra \SList A\: i \ra \SList A\: \infty \]
\noindent so the information that $\quicksort$ is size preserving is lost.

\subsection{Sized Brouwer ordinals}
Now sized Brouwer ordinals are introduced:
\begin{bsp}
$\sized \data \SOrd : \Size \Set$ \\
$\spcx \ozero : (i : \Size ) \ra \SOrd (\s i) $\\
$\spcx \olim : ( i : \Size ) \ra (\Nat \ra \SOrd i) \ra \SOrd (\s i) $
\end{bsp}
With sized ordinals, Axiom 2 ($f \: \vec{e} \leq f$) is not needed to accept ordinal addition:
\begin{bsp}
$\fun \addOrd : \SOrd \infty \ra \EPi{i}{\Size}{\SOrd i \ra \SOrd \infty }$\\
$\spc \addOrd x\: \inacc{\s i}\: (\ozero i) = \ozero i $\\
$\spc \addOrd x\: \inacc{\s i}\: (\olim f)  = \olim \infty \: ( \lam{y} \addOrd  x \: i \: (f \: y))  $
\end{bsp}
$\addOrd$ is structurally recursive in the size argument.
\subsection{A higher-order function}
While somewhat artificial, the following example is interesting for two reasons:
\begin{bsp}
$\fun \addWith : ((k : \Size ) \ra \SNat k \ra \SNat k ) \ra (i : \Size ) \ra (j : \Size ) $\\
$\spcx \spcx \ra  \SNat i \ra \SNat j \ra \SNat \infty $\\
$\spc \addWith f \: \inacc{\s i} \: j \: (\zero i) \: y = y $ \\
$\spc \addWith f \: \inacc{\s i} \: j \: (\suc i \: x) \: y = \suc \infty \: (\addWith f \: j \: i \: y \: (f \: i \: x))$
\end{bsp}
$\addWith$ resembles the permuting function $\addp$ that was presented in the previous chapter.
But it has one additional parameter that is declared to be a \emph{size-preserving function}. 
Any function of type 
\[ (k : \Size ) \ra \SNat k \ra \SNat k \]
can now be passed along to $\addWith$.
The second reason is that the function has two size arguments that are permuting in the recursive call. 
So this is really a example that plays to the strength of our system: both the $\Size$ type and the size-change principle are needed to show termination.

\section{Examples: Sized coinductive types}

The \emph{guardedness condition} \cite{coquand-infinite} could be employed for checking productivity of coinductive definitions. But in the following, we will use the $\Size$ type to prove productivity.

\subsection{Sized Streams}

Recall the declaration of $\zeroes$ from section \ref{lstream}: 
\begin{bsp}
$\cofun \zeroes: \Stream $ \\
$\spc \zeroes = \cons \zero \zeroes$
\end{bsp}
\noindent An example of an unproductive stream is $\unp$:
\begin{bsp}
$\cofun \unp : \Stream $ \\
$\spc \unp = \unp$
\end{bsp}
and $\wh\; {\clp{\head \unp}{\diamond}}$ is not defined.
Now let us transfer these declarations to the sized version of $\Stream$.
For the productive stream $\zeroes$, we turn it into
\begin{bsp}
$\cofun \zeroes: \EPi{i}{\Size}{\SStream i} $ \\
$\spc \zeroes (\s i) = \cons i \: \zero (\zeroes i)$
\end{bsp}
This passes the termination checker, because 
the call matrix
\[
\zeroes
\begin{pmatrix}
<    
\end{pmatrix}\zeroes
\] 
is idempotent and decreasing. With sized types, productive corecursive definitions are now structurally recursive on the size argument. This would also pass the guardedness check: the call to $\zeroes$ is \emph{guarded} by $\cons$.

For the unproductive stream, we have two bad choices:
The first one
\begin{bsp}
$\cofun \unp : \EPi{i}{\Size}{\SStream i}$ \\
$\spc \unp i =  \unp i$
\end{bsp}
is type-correct, but does not pass the termination-check.
The second option
\begin{bsp}
$\cofun \unp : \EPi{i}{\Size}{\SStream i}$ \\
$\spc \unp (\s i) = \unp i $
\end{bsp}
would be accepted by the termination-checker, but is not type-correct.

Now, for a productive stream, we can look at the first element of a stream with $\head$, or remove the first element of a stream with $\tail$:
\begin{bsp}
$\fun \head : \SStream \infty \ra \Nat$\\
$\spc \head \: (\cons \: \inacc{\infty} \: x \: xs) = x$

$\fun \tail :  \SStream \infty \ra \SStream \infty $\\
$\spc \tail \: (\cons \: \inacc{\infty} \: x \: xs) = xs$\\
\end{bsp}

\noindent We can now define the $n$th element of a stream:
\begin{bsp}
$\fun \nth : \Nat \ra \SStream \infty \ra \Nat$\\
$\spc \nth \zero xs = \head xs$\\
$\spc \nth (\suc n) \: xs = \nth n \: (\tail xs)$\\
\end{bsp}
\subsection{Fibonacci stream}

The obligatory example is to define the stream of Fibonacci numbers:
\begin{bsp}
$\cofun \fibs : \Nat \ra \Nat \ra \EPi{i}{\Size}{\SStream i}$\\
$\spc  \fibs x \: y \: (\s i) = \cons i \: x \: (\fibs y \: (\add x \: y) \: i)$

$\clet \fib : \SStream \infty = \fibs (\suc \zero) \: (\suc \zero) \: \infty$
\end{bsp}
and get the fourth Fibonacci number by:
\begin{bsp}
$\clet \fibf : \Nat = \nth (\suc (\suc (\suc (\suc \zero )))) \: \fib$
\end{bsp}

\subsection{Equality of streams}
\label{heq}
The following is a type-correct declaration:
\begin{bsp}
$\clet \eqo : \Eq (\SStream \infty) \: (\zeroes \infty) \: (\cons \infty \: \zero (\zeroes \infty))$\\
$\spc = \refl (\SStream \infty) \: (\zeroes \infty)$
\end{bsp}
The inferred type of $\refl (\SStream \infty) \: (\zeroes \infty)$ is 
\[\Eq (\SStream \infty) \: (\zeroes \infty) \: (\zeroes \infty)\]
The type-checker has to unroll $(\zeroes \infty)$ once to see that the type is equal to the declared type of $\eqo$.
Now let's define another stream:
\begin{bsp}
$\cofun \zeroest: \EPi{i}{Size}{\SStream i} $ \\
$\spc \zeroest (\s \s i) = \cons (\s i) \: \zero (\cons i \: \zero (\zeroest i))$
\end{bsp}
We cannot prove \[\Eq (\SStream \infty) \: (\zeroes \infty) \: (\zeroest \infty)\]
because the type-checker can't unroll both $\zeroes \infty$ and $\zeroest \infty$ as
this would lead to no progress.
But we can define bisimilarity on streams \cite{coquand-infinite}, as a sized coinductive predicate:
\begin{bsp}
$\codata \beqd : \Size \ra \SStream \infty \ra \SStream \infty \ra \Set$\\
$ \spcx \beq : (i : \Size ) \ra ( n : \Nat ) \ra (s_1 : \SStream \infty ) \ra (s_2 : \SStream \infty )$\\
$ \spcx \spc  \ra \beqd i \: s_1 \: s_2 \ra \beqd (\cons n \: s_1) \: (\cons n \: s_2)$
\end{bsp}
\noindent And now the following is a valid infinite proof:
\begin{bsp}
$\cofun \eqt : (i : \Size ) \ra \beqd i \: (\zeroes \infty) \: (\zeroest \infty)$\\
$\spc \eqt (\s \s i) = $\\
$\spc \spcx \beq (\s i) \: \zero (\cons \infty \: \zero (\zeroes \infty)) \: (\cons \infty \: \zero (\zeroest \infty))$\\ 
$\spc \spcx \spc (\beq i \: \zero (\zeroes \infty) \: (\zeroest \infty) \: (\eqt i))$
\end{bsp}
Observational equality \cite{conf/plpv/AltenkirchMS07}, which is a recent attempt to strengthen decidable equality for dependent types, would automatically entail this notion of bisimilarity for coinductive types.

\subsection{Stream processors}

A stream processor \cite{journals/entcs/GhaniHP06} transforms an input streams into an output stream.
It can $\getd$ an element from the input stream or $\putd$ an element into the output stream.
The generated output stream is productive if the stream processor does not stop putting elements into the output stream.

Productive stream processors can be nicely modeled with mixed in\-ductive/coinductive sized types \cite{abel:aplas07}. 
As $\mugda$ does not support such mixed declarations directly, we need to resort to a \emph{continuation-passing style}:
\begin{bsp}
$\data \ispd ( + \: K : \Set) : \Set$\\
$\spc \putd : \Nat \ra K \ra \ispd K$ \\
$\spc \getd : (\Nat \ra \ispd K) \ra \ispd K $ \\

$\sized \codata \spd : \Size \ra \Set$\\
$\spc \isp : (i : \Size) \ra \ispd (\spd i) \ra \spd (\s i)$
\end{bsp}

As an example, the stream processor $\adder$ continuously gets a natural number $n$, 
then puts the sum of the following $n$ input elements into the output stream:
\begin{bsp}
$\fun \iadder : \Nat \ra \Nat \ra (K : \Set ) \ra K \ra \ispd K$\\
$\spc \iadder \zero acc \: K \: k = \putd K \: acc \: k$\\
$\spc \iadder (\suc n) \: acc \: K \: k = \getd \: K \: (\lam{m} (\iadder n \: (\add \: m \: acc) \: K \: k))$

$\cofun \cadder : (i : \Size ) \ra \spd i$\\
$\spc \cadder (\s i) = \isp i \: (\getd (\spd i) \: (\lam{n} \iadder n \: \zero (\spd i) \: (\cadder i)))$

$\clet \adder : \spd \infty = \cadder \infty$
\end{bsp}
\noindent The execution of a stream processor, called \emph{eating}, follows:
\begin{bsp}
$\fun \ieat : (K : \Set) \ra (C : \Set) \ra $\\
$ \spcx \ispd K \ra \SStream \infty \ra (\Nat \ra K \ra \SStream \infty \ra C) \ra C$ \\
$ \ieat \inacc{K} \: C \: (\getd K \: f) \: (\cons \inacc{\infty} \: a \: as) \: h = \ieat K \: C \: (f a) \: as \: h $ \\
$ \ieat \inacc{K} \: C \: (\putd K \: b \:k) \: as  \: h = h \: b \: k \: as $

$\cofun \eat : (i : \Size) \ra \spd \infty \ra \SStream \infty \ra \SStream i$\\
$\spc \eat (\s i) \: (\isp \inacc{\infty} \: ip) \: as = \ieat (\spd \infty) \: (\SStream (\s i))$\\
$\spcx \spcx \spcx  \spcx ip \: as \: (\lam{b} \: \lam{k} \: \lam{as'} (\cons i \: b \: (\eat i \: k \: as'))$   
\end{bsp}
\noindent $\ieat$ is shown terminating due to Axiom 2, and $\eat$ is structurally recursive on the size argument, thus productive.

But $\eat$ would not be accepted by the guardedness check: Although the recursive call to $\eat$ is guarded by the constructor $\cons$, this constructor is again surrounded by $\ieat$, which is not allowed by this syntactic criterion.

\section{Admissible recursive function declarations}


As we will see, the use of $\Size$ needs to be constraint.
Otherwise, there are declarations that are type-correct and pass the termination-checker, but lead to non-termination.

We will give a criterion for when a mutual declaration $\delta$ is \emph{admissible}.
Admissibility is a necessary concept in works on sized types.
Admissibility based on \emph{monotonicity} is for example used in \cite{bgp:lpar06} and \cite{blanqui04typebased}.
The admissibility of \cite{abel:PhD} and \cite{hughes96proving} is based on the more advanced concept of \emph{continuity}.

The criterion for $\mugda$ is based on monotonicity and has to deal with dependent pattern matching. 
We look at the types of a mutual declaration, and also -- because matching on constructors has an influence on the type -- at the patterns. Consider for example the following:
\begin{bsp}
$\fun \bad : \EPi{i}{\Size}{\Bool}$\\
$\spc \bad (\s i) = \bad i $ 
\end{bsp}
It is type-correct, and the termination-checker will happily tell you that $\bad$ is terminating.
But $\wh \clp{\bad \infty}{\rho}$ is not defined.

It can be argued that the pattern $\s i$ does not cover all cases: A hypothetical bottom size element would not match against $\s i$. Case dinstinction on a size should not be accepted.
For the same reason, the following function definition is also not admissible:
\begin{bsp}
$\fun \badb : \EPi{i}{\Size}{\SNat i \ra \Bool}$\\
$\spc \badb \inacc{\s i} \: (\zero (\s i)) = \badb i \: (\zero i)$\\
$\spc \badb \inacc{\s i}\: (\suc i \: x) = \badb i \: x;$
\end{bsp}
For example, $\wh \clp{\badb \infty \: (\zero \infty)}{\rho}$ is not defined.
Such examples are forbidden by the \emph{bottom-check} \cite{hughes96proving} in other systems with sized types.

Both examples will be rejected because of their incomplete size pattern.
But there are examples where the size patterns are complete, but the type needs to be rejected.
The program in figure \ref{loop} is adopted from \cite{abel:PhD}. The type of the function $\lop$ is   
\[\EPi{i}{\Size}{\SNat i \ra \Maybe (\SNat (\s i)) \ra (\SNat \infty \ra \Maybe (\SNat i))} \]
and needs to be rejected by an admissibility check. Otherwise $\diverge$ could be constructed, which leads to non-termination
when evaluated.
\begin{figure}[p]
\centering{
\begin{minipage}{0.8\textwidth}
\begin{bsp}
$\data \Maybe ( + \: A : \Set ) : \Set$\\
$\spc  \nothing : \Maybe A$\\
$\spc  \just : A \ra\Maybe A$

$\fun \shiftcase : (i : \Size) \ra\Maybe (\SNat (\s i)) \ra\Maybe (\SNat i)$\\
$\spc \shiftcase i \: (\nothing \inacc{\SNat \s i}) = \nothing (\SNat i)$\\
$\spc \shiftcase \inacc{i} \: (\just \inacc{\SNat \s i} \: (\zero i)) = \nothing (\SNat i)$\\
$\spc \shiftcase \inacc{i} \: (\just \inacc{\SNat \s i} \: (\suc i \: x)) = \just (\SNat i) \: x$\\

$\clet \shift : (i : \Size) \ra(\SNat \infty \ra \Maybe (\SNat (\s i))) $\\
$\spcx \spcx \ra\SNat \infty \ra \Maybe (\SNat i) $\\ 
$\spc = \lam{i}{\lam{f}{\lam{n}{\shiftcase i \: (f \: (\suc \infty \: n))}}}$

$\clet \inc : \SNat \infty \ra \Maybe \SNat \infty = \lam{n}{\just \SNat \infty \: (\suc \infty \: n)}$

$\mutual$\\
$\spc \fun \lop : (i : \Size ) \ra \SNat i $\\
$\spcx \spcx \ra (\SNat \infty \ra\Maybe (\SNat i)) \ra\Bool$\\
$\spc \spc \lop \inacc{\s i} \: (\zero i) f = \lopcase (\s i) \: (\zero i) \: f \: (f \: (\zero i))$\\
$\spc \spc \lop \inacc{\s i} \: (\suc i \: n) f = \lop i \: n \: (\shift i \: f)$

$\spc \fun \lopcase : (i : \Size ) \ra (\SNat \infty \ra \Maybe (\SNat i)) $\\
$\spcx \spcx \ra \Maybe (\SNat i) \ra\Bool$\\
$\spc \spc \lopcase i \: f \: (\nothing \inacc{\SNat i}) = \ttt $\\
$\spc \spc \lopcase \inacc{\s i} \: f \: (\just \inacc{\SNat (\s i)} \: (\zero i)) = \ttt $\\
$\spc \spc \lopcase \inacc{\s i} \: f \: (\just \inacc{\SNat (\s i)} \: (\suc i \: y)) = \lop i \: y \: (\shift i \:f) $

$\clet \diverge : \Bool = \lop \infty\: (\zero \infty) \: \inc$
\end{bsp}
\end{minipage}
}
\caption{Loop example}
\label{loop}
\end{figure}


\newcommand{\mon}[3]{#1\vdash#2\mbox{ \textbf{mon} }#3}
\newcommand{\indu}[3]{#1\vdash#2\mbox{ \textbf{ind} }#3}
\newcommand{\coind}[3]{#1\vdash#2\mbox{ \textbf{coind} }#3}
\newcommand{\admIndSize}[3]{#1\vdash#2\mbox{ \textbf{admIndSize} }#3}
\newcommand{\admCoSize}[3]{#1\vdash#2\mbox{ \textbf{admCoSize} }#3}
\newcommand{\admIndType}[2]{#1\vdash#2\mbox{ \textbf{admIndType}}}
\newcommand{\admCoType}[2]{#1\vdash#2\mbox{ \textbf{admCoType}}}

\newcommand{\sizePat}[1]{\vdash#1\mbox{ \textbf{sizePat}}}
\newcommand{\sizePats}[1]{\vdash#1\mbox{ \textbf{sizePats}}}

What follows is admissibility for the case of a mutual recursive declarations.
Admissibility for corecursive declarations will be covered in the next section.
\subsection{Admissible type}

First, all types of a mutual function declaration have to be \emph{admissible}.
This can be described informally for a type \emph{expression} $t$:
\begin{itemize}
\item
For an inductive function with type \[t = (a_1 : A_1) \ra \ldots \ra (a_n : A_n) \ra R\]
it is required that for every \emph{argument type} $(a_j : A_j)$ of the form $(i : \Size)$  
\begin{itemize}
\item
For $k>j$, either $i$ is not occurring in $A_k$ or $A_k$ is \emph{inductive in $i$}, i.e. $A_k$ is a sized inductive type of size $i$.
\item
the \emph{result type} $R$ is \emph{monotone} in $i$. 
\end{itemize}
\end{itemize}
The formal judgment in figure \ref{admi} is defined on evaluated types, not expressions.

\begin{figure}[htp]
\begin{gather*}
\nru{}
{\LeqVal{k}{\nF}{v}{\subst{(i,\s i)}{v}}}
{\mon{k}{v}{i}}
\qquad
\nru{}
{\DD \mbox{ sized inductive data type}}
{\indu{k}{\DD \: p_1 \ldots p_n \: i \: v_1 \ldots v_m}{i}}
{\: \Sig \: \DD = (v_D,n)}
\\[2ex]
\ndru{}
{\indu{k}{v_A}{i} \quad \mbox{or} \quad \nocc{k}{v_A}{i}}
{\admIndSize{k+1}{\wh \cl{b}{\ins{\rho}{x}{k}}}{i}}
{\admIndSize{k}{\vpi \: x \: v_A \: \cl{b}{\rho}}{i}}
\nru{}
{\mon{k}{v}{i}}
{\admIndSize{k}{v}{i}}
\\[2ex]
\ndru{}
{\admIndSize{k+1}{\wh \cl{b}{\ins{\rho}{x}{k}}}{k} \mbox{ if } v_A = \Size}
{\admIndType{k+1}{\wh \cl{b}{\ins{\rho}{x}{k}}}}
{\admIndType{k}{\vpi \: x \: v_A \: \cl{b}{\rho}}}
\qquad
\nru{}
{}
{\admIndType{k}{v}}
\end{gather*}
\caption{Admissibility type for inductive function}
\label{admi}
\end{figure}

\subsection{Size pattern coverage}

As was said before, the check for complete coverage of pattern matching will not be detailed in this work. 
But a special case is coverage for the built-in $\Size$ type: without this check even non-terminating function declarations would be accepted. 
For a mutual recursive declaration $\delta$, the patterns $\vec{p}$ are \emph{size complete}, if no pattern of the form $(\s p)$ occurs. So only a variable $i$ will be a complete pattern. This still allows $\s$ to appear inside of inaccessible patterns.
Although straightforward, this is formalized with the judgment in figure \ref{spj}.

\begin{figure}[htp]
\begin{gather*}
\nru{}
{\sizePats{\vec{p}}}
{\sizePat{\cc \: \vec{p}}}
\qquad
\nru{}
{}
{\sizePat{x}}
\qquad
\nru{}
{}
{\sizePat{\inacc{e}}}
\\[2ex]
\nru{}
{\sizePat{p} \qquad \sizePats{\vec{p}}}
{\sizePats{p \: \vec{p}}}
\qquad
\nru{}
{}
{\sizePats{\diamond}}
\end{gather*}
\caption{Size pattern completeness for recursive function}
\label{spj}
\end{figure}


\subsection{Admissibility criterion}
Putting the previous two sections together,
the final admissibility criterion follows:
\begin{quote}
Fo the mutual recursive function declaration $\delta =$
\begin{bsp}
$\mutual$\\
$\spcx \fun f_1 : A_1 $\\
$\spcx \spc \vec{\gamma}_1$\\
$\spcx \ldots$\\
$\spcx \fun f_n : A_n $\\
$\spcx \spc \vec{\gamma}_n$
\end{bsp}
if  for every $i \in \{1 \ldots n \}$:
\begin{enumerate}
\item
$\admIndType{1}{\wh \cl{A_i}{\diamond}}$
\item
$\sizePats{\vec{p}}$ for every clause $f \: \vec{p} \: e \in \gamma_i$
\end{enumerate}
then $\delta$ is \emph{admissible}.
\end{quote}

\subsubsection{Examples}

For the rejected example $\lop$, $i$ is occurring in the argument type \[\Maybe (\SNat (\s i))\] but this argument type is not inductive in $i$, so the function $\lop$ is not admissible because of its type.
Both $\bad$ and $\badb$ are not admissible because their size patterns. 
As can be seen, all examples given in sections 5.4 and 5.5 are admissible.

\section{Admissible corecursive declarations}

For corecursive functions, the type admissibility criterion is even more restrict.
Again, it is first outlined for a type expression:
\begin{itemize}
\item
For a corecursive function with type \[t = (a_1 : A_1) \ra \ldots \ra (a_n : A_n) \ra R\] it is required that 
for every argument type $(a_j : A_j)$ of the form $(i : \Size)$  
\begin{itemize}
\item
$i$ is not occurring in argument types $A_k$ (where $k > j$). 
\item
the result type $R$ is \emph{coinductive in i}, i.e. $R$ is a sized coinductive type of size $i$. 
\end{itemize}
\end{itemize}
This is formalized with the judgment given in figure \ref{coadm}.

We do not check the completeness of the size pattern for corecursive declarations because the size argument is determined by the right hand side -- the object that is being defined -- and thus can be seen as inaccessible. 


\begin{figure}[htp]
\begin{gather*}
\nru{}
{\DD \mbox{ sized codata}}
{\coind{k}{\DD \: p_1 \ldots p_n \: i \: v_1 \ldots v_m}{i}}
{\: \Sig \: \DD = (v_D,n)}
\\[2ex]
\nru{}
{\nocc{k}{v_A}{i}
\quad
\admCoSize{k+1}{\wh \cl{b}{\ins{\rho}{x}{k}}}{i}}
{\admCoSize{k}{\vpi \: x \: v_A \: \cl{b}{\rho}}{i}}
\quad
\nru{}
{\coind{k}{v}{i}}
{\admCoSize{k}{v}{i}}
\\[2ex]
\ndru{}
{\admCoSize{k+1}{\wh \cl{b}{\ins{\rho}{x}{k}}}{k} \mbox{ if } v_A = \Size}
{\admCoType{k+1}{\wh \cl{b}{\ins{\rho}{x}{k}}}}
{\admCoType{k}{\vpi \: x \: v_A \: \cl{b}{\rho}}}
\qquad
\nru{}
{}
{\admCoType{k}{v}}
\end{gather*}
\caption{Admissibility type for corecursive function}
\label{coadm}
\end{figure}


\subsection{Admissibility criterion}
\begin{quote}
For the mutual corecursive function declaration $\delta =$
\begin{bsp}
$\mutual$\\
$\spcx \cofun f_1 : A_1 $\\
$\spcx \spc \vec{\gamma}_1$\\
$\spcx \ldots$\\
$\spcx \cofun f_n : A_n $\\
$\spcx \spc \vec{\gamma}_n$
\end{bsp}
if  for every $i \in \{1 \ldots n \}$:
\begin{itemize}
\item
$\admCoType{1}{\wh \cl{A_i}{\diamond}}$
\end{itemize}
then $\delta$ is \emph{admissible}.
\end{quote}


\subsection{Fibonacci \`a la Haskell}


Both admissibility criteria do not allow an argument of the form $\SStream i$.
Functions can only have arguments of the form $\SStream \infty$, so pattern matching can only 
happen on fully constructed infinite objects.

In the lazy functional programming language Haskell \cite{haskell}, which supports partiality, the following is a valid definition of the stream of Fibonacci numbers:
\begin{quote}
\begin{verbatim}
fib :: [Int]
fib = (1 : ( 1 : zipWith (+) fib (tail fib)))
\end{verbatim}
\end{quote}
Translating this definition into $\mugda$ as a sized stream is not possible, and indeed is not productive with the given evaluation semantics.

In Haskell, the helper functions \verb+zipWith+ and \verb+tail+ can operate on any list, even on \verb+fib+ which is just being defined.
In the $\mugda$ setting, $\tail$ would need to have the type
\[ ( i : \Size ) \ra \SStream (\s i) \ra \SStream i \]
\noindent This type is not admissible, because $i$ is occurring in the argument type $\SStream (\s i)$.

\section{On the necessity of subtyping}
Subtyping for inductive types is actually just for convenience to the user. 
For example
\begin{bsp}
$\fun \wkSNat : (i : \Size ) \ra \SNat i \ra \SNat (\s i)$\\
$\spc \wkSNat \inacc{\s i} \: (\zero i) = \zero (\s i)$\\
$\spc \wkSNat \inacc{\s i} \: (\suc i \: x) = \suc (\s i) \: (\wkSNat i \: x) $
\end{bsp}
could be used to manually weaken an object into a greater type.
But subtyping does make the system more comfortable, and also this weakening would have
a significant impact on runtime performance.
Its bad brother 
\begin{bsp}
$\fun \wkSNatb : (i : \Size ) \ra \SNat (\s i) \ra \SNat i$\\
$\spc \wkSNatb \inacc{\s i} \: (\zero (\s i) = \zero i$\\
$\spc \wkSNatb \inacc{\s i} \: (\suc (\s i) \: x) = \suc i \: (\wkSNatb i \: x) $
\end{bsp}
fails to pass the admissibility test. 
For $\Stream$, the weakening function would be
\begin{bsp}
$\cofun \wkStream : ( A : \Set ) \ra (i : \Size ) \ra \SStream A \: (\s i) \ra \SStream A \: i$\\
$\spc \wkStream \inacc{A} \: \inacc{\s i} \: (\cons A \: ((\s i)) \: x \: xs) = \cons A \: i \: x \: (\wkStream A \: i \: xs) $
\end{bsp}
but its type is not admissible. This seems to indicate that a less restrictive admissibility criterion should be achievable.

\section{Putting it all together}
Now that we have limited the use of the $\Size$ type, we can formulate our termination criterion for 
$\mugda$ with sized types. We conjecture:
\subsubsection{Proposition : Size-change principle for Mugda with sized types}
\begin{quote}
Given
\begin{enumerate}
\item
$\Sig$ a terminating signature
\item
a mutual (co)recursive declaration $\delta$
\item
$\Sigma'$ is resulting from type-checking $\delta$ in $\Sigma$
\item
$\delta$ is admissible
\item
every idempotent call-matrix $ \alpha \in \complete \cs(\delta)$ is decreasing
\end{enumerate}
then $\Sigma'$ is terminating.
\end{quote}
\noindent The integration of the size type in $\mugda$ went through a lot of changes.
At first, $\Size$ was a small type. Through this, a ``bad user'' could do a lot of things
to fool the system. Taking $\Size$ out of $\Set$ defused the situation quite a bit.
Admissibility was at times much more lenient, until counter-examples such as ``Fibonacci \`a la Haskell'' were found.

%%% Local Variables: 
%%% mode: latex
%%% TeX-master: "da"
%%% End: 
