\chapter{Basic MiniAgda}

\newcommand{\dd}{\mathbf{d}}
\newcommand{\DD}{\mathbf{D}}
\newcommand{\cc}{\mathbf{c}}
\newcommand{\ff}{\mathbf{f}}

\begin{figure}
\caption{MiniAgda syntax}
Expressions : \[e,t :  \lam x . e \vsp \EPi{x}{t}{t} \vsp e \; \vec{e} \vsp  \Set \vsp x \vsp \cc \vsp \DD \vsp \ff \vsp \dd\]
Patterns: \[ x  \vsp \cc \; \vec{p} \]
\end{figure}

MiniAgda is a small experimental version of Agda.
The current version of Agda is described by Ulf Norell in his thesis \cite{norell:thesis}.
\section{Syntax}
Let $\vec{a}$ denote a (possible empty) list of elements of category $a$
and $\diamond$ denote the empty list.
Int the following, these identifiers are used:
\begin{itemize}
\item for constructors : $\cc$ 
\item for data types : $\DD$
\item for functions : $\ff$
\item for definied constants : $\dd$
\item for variables : $x$
\end{itemize}
Now the various syntactical categories can be definied.
\\
Expressions (terms and types) $e,t$:  
\begin{itemize}
\item
abstraction: $ \lam x . e $ 
\item
dependent function space: $\EPi{x}{t}{t}$ 
\item 
application: $ e \; \vec{e} $ 
\item
type of all types: $ \Set $  $ \Set $ 

variables: $x$
\item
defined constructors, data types, functions and constants:
$ \cc \vsp \DD \vsp \ff \vsp \dd$
\end{itemize}
Patterns $p$:
\begin{itemize}
\item
variable pattern: $x$ 
\item
constructor pattern:$ \cc \; \vec{p} $ 
\item
dot pattern: $ . \; e $ 
\item
absurd pattern: $()$
\end{itemize}
Declarations $\delta$:
\begin{itemize}
\item
inductive datatype: 
\[\data \; \DD \; \tau : t \; \vec{\gamma}\]  
\item
inductive function:
\[\fun \; \ff \; : \; t \; \vec{\kappa}\]
\item
constant:
\[\const \;  \dd \; : t \; = \; e \]
\item
mutual declaration:
\[\mutual \; \vec{\delta}\]
\end{itemize}
Constructor definitions $\gamma$:
\begin{itemize}
\item
$ \cc \; : \; t $ 
\end{itemize}
Function clause definitions: $ \kappa $
\begin{itemize}
\item
normal clauses: $ \ff \; \vec{p} = e $
\item
absurd clauses : $ \ff \; \vec{p} $
\end{itemize}
Telescopes:
\[ \tau := \diamond \vsp ( x : t ) \; \tau \]
A MiniAgda program consists of a list of declarations.
\section{Examples}
\subsection{Identity function}
as this functon is not recursive, it can be defined as a const.
this is not really a polymorphic identity function, as you have to supply the
type as the first argument.
$\const id : \EPi{A}{\Set}{A \ra A} = \lam a \lam x . ; x$   
\subsection{Booleans}
$\data \Bool : \Set$ 
\\
$\ttt : \Bool $
\\
$\fff : \Bool $
\subsection{Natural Numbers}
First we declare the type of natural numbers:
\begin{quote}
$\data \Nat : \Set$ 
\\
$\zero : \Nat $
\\
$\suc : \Nat \ra \Nat $
\end{quote}
then we declare the addition function by pattern matching:
\begin{quote}
$\fun \add : \Nat \ra \Nat \ra \Nat$
\\
$\add x \; \zero = x $
\\
$\add x (\suc y) = \suc ( \add x \; y)  $
\end{quote}
\subsection{Lists}
\begin{quote}
$\data \List ( A : \Set ) : \Set $ 
\\
$\nil : \List A  $
\\
$\cons : A \ra \List A \ra \List A $
\end{quote}


\section{Universes}
To keep MiniAgda simple, $\Set$ is the type of all types.
The judgement is
\[\Set : \Set \]
This is known to be inconsistent.
\section{Missing features}
Due to its use for interactivly building proofs, the full Agda has meta-variables that represent parts of tge proof that are still misssing. These 
Also, many bookkeeping arguments can be easily infered by the type checker,easying tedious pain for the user.
The user can mark these arguments as \emph{implicit}. The type checker will automatically infer them.
To keep MiniAgda simple, these features were not implemented.



\section{Type Checking}

\renewcommand\Check[4]{#1,#2\;\vdash\;#3\uparrow#4}
\newcommand\Infer[4]{#1,#2\;\vdash\;#3\downarrow#4}
\newcommand\IsType[3]{#1,#2\;\vdash\;#3\uparrow\Set}
\newcommand\EqVal[2]{\vdash\;#1\leftrightarrow#2}
\newcommand\Eval[3]{\mathrm{eval}\;#1\;#2\leadsto#3}
\newcommand\App[3]{\mathrm{app}\;#1,#2\leadsto#3}
\newcommand\AppFun[3]{\mathrm{appFun}\;#1,#2,\leadsto#3}
\newcommand\Fapp[6]{\mathrm{fapp}\;#1,#2,#3,#4,#5\leadsto#6}
\newcommand\Whnf[2]{\mathrm{whnf}\;#1\leadsto#2}
\newcommand\Lookup[3]{#1(#2) = #3}

\subsection{Values}
Values are the semantical interpretation of expressions.
For the interpretation of abstraction and the depentend function space, closures are used.
Values:
\[ v := \Set \vsp \Size \vsp vs \vsp \infty \vsp v \; \vec{v} \vsp \cc \vsp \ff \vsp \ e \; \rho vsp k \]
Environments:
\[ \rho := \diamond \vsp (x,v) \rho \]

The generic values $k$ are needed during type checking.
\subsection{Computation during type checking}
First of all, as types can depent on terms, it becomes necessary to perfom computation on terms during type checking.
The mutually recursive pair of functions 
\textbf{eval} and \textbf{app} do the job.
eval computes the weak head normal form of a expression.
\[ evaluation : \Eval{\rho}{e}{v}\]

application is the equivalent of beta reduction on values
\[ application : \App{v}{\vec{v}}{v'}\]

weak head normal form: 
\[ whnf : \Whnf{v}{v'} \]
\begin{figure}

\[\begin{array}{c}
	  \infer{\Eval{\rho}{\Set}{\Set}} 
          {} 
\end{array}
\]

\[
\begin{array}{c}
	  \infer{\Eval{\rho}{\cc}{\cc}} 
          {}

\end{array}
\]

\[\begin{array}{c}
	  \infer{\Eval{\rho}{\ff}{\ff}} 
          {} 
\end{array}
\]

\[\begin{array}{c}
	  \infer{\Eval{\rho}{\dd}{v}} 
          {\Lookup{\Sigma}{\dd}{e} & \Eval{\rho}{e}{v}} 
\end{array}
\]

\[\begin{array}{c}
	  \infer{\Eval{\rho}{x}{v}} 
          {\Lookup{\rho}{\dd}{v}} 
\end{array}
\]

\[\begin{array}{c}
	  \infer{\Eval{\rho}{\EPi{x}{t}{t'}{\VPi{x}{v}{t'}{\rho}}}}
          {\Eval{\rho}{t}{v}}
\end{array}
\]

\[\begin{array}{c}
	  \infer{\Eval{\rho}{\lam x . e}{\VLam{x}{e}{\rho}}}
		{} 
\end{array}
\]


\[
\begin{array}{c}
          
	  \infer{\Eval{\rho}{e \vec{e}}{v \vec{v}}}
          {\Eval{\rho}{e}{v} & \Eval{\rho}{\vec{e}}{\vec{v}}}
\end{array}
\]
\caption{Evaluation of expressions}
\end{figure}

\begin{figure}

\[
\begin{array}{c}
          
	  \infer{\App{\VLam{x}{e}{\rho}}{v \; \vec{v}}{v''}}
          {\Eval{(x,v)\rho}{e}{v'} & \App{v'}{\vec{v}}{v''}}
\end{array}
\]

\[
\begin{array}{c}
          
	  \infer{\App{\VPi{x}{w}{e}{\rho}}{v \; \vec{v}}{v''}}
          {\Eval{(x,v)\rho}{e}{v'} & \App{v'}{\vec{v}}{v''}}

\end{array}
\]

\[
\begin{array}{c}
          
	  \infer{\App{\ff}{\vec{v}}{v'}}
          {\AppFun{\ff}{\vec{v}}{v'}}
\end{array}
\]

\caption{application of values}
\end{figure}


\begin{figure}
\[
\begin{array}{c}
          
	  \infer{\Whnf{e \; \rho}{v'}}{\Eval{\rho}{e}{v'}}

\end{array}
\]
\caption{weak head normal form}
\end{figure}

\subsection{Bidirectional type checking} 
For depentend types, a practice called \emph{bidirectional type checking} is used.
This means that the type checker has two modes:
one for checking that an expression has a certain type and one for infering the type of an expression.
These modes are again implement as functions
\[ \mathrm{check expression} :  \Check{\rho_1}{\rho_2}{e}{v}\]

and

\[ \mathrm{infer expression} : \Infer{\rho_1}{\rho_2}{e}{v}\]

in checking mode, the type checker might have to infer the type of the expression and then check that the infered
type is equal (convertible) against the proposed type.

Also, as a special case of check expression, the judgment 

\[ \mathrm{is type} : \IsType{\rho_1}{\rho_2}{e}\]

which checks that an expression is a type, i.e has type $\Set$.

so one needs equality between values, which is handled by the function
\[ \mathrm{eqVal} : \EqVal{v_1}{v_2} \]


\begin{figure}
\[\begin{array}{c}
	  \infer{\Check{\rho_1}{\rho_2}{\lam x . e}{\VPi{y}{v}{e'}{\rho}}}
          { \Eval{(x,k)\;\rho}{e}{v'} & \Check{(x,k)\;\rho_1}{(x,v)\;\rho_2}{e'}{v'}} 
\end{array}
\]

\[\begin{array}{c}
	  \infer{\Check{\rho_1}{\rho_2}{\EPi{x}{t}{t'}}{\Set}}
          {\IsType{\rho_1}{\rho_2}{t}  
	    & \Eval{(x,k)\;\rho_1}{t}{v}
	    & \IsType{(x,k)\rho}{{x,v}\rho_2}{t'}}
\end{array}
\]

\[\begin{array}{c}
	  \infer{\Check{\rho_1}{\rho_2}{e}{v}} 
          { \Infer{\rho_1}{\rho_2}{e}{v'} & \EqVal{\rho_1}{\rho_2}{e}{v'}} 
\end{array}
\]
\caption{Checking expressions}
\end{figure}

\begin{figure}
\[\begin{array}{c}
	  \infer{\Infer{\rho_1}{\rho_2}{x}{v}} 
          {\Lookup{\rho_2}{x}{v}}
\end{array}
\]

\[\begin{array}{c}
	  \infer{\Infer{\rho_1}{\rho_2}{\Set}{\Set}} 
          {}
\end{array}
\]

\[\begin{array}{c}
	  \infer{\Infer{\rho_1}{\rho_2}{\ff}{v}} 
          {\Lookup{\Sigma}{\ff}{v}}
\end{array}
\]

\[\begin{array}{c}
	  \infer{\Infer{\rho_1}{\rho_2}{\cc}{v}} 
          {\Lookup{\Sigma}{\cc}{v}}
\end{array}
\]

\[\begin{array}{c}
	  \infer{\Infer{\rho_1}{\rho_2}{\DD}{v}} 
          {\Lookup{\Sigma}{\dd}{v}}
\end{array}
\]

\[\begin{array}{c}
	  \infer{\Infer{\rho_1}{\rho_2}{e_1 \; e_2}{v''}}
{\Infer{\rho_1}{\rho_2}{e1}{\VPi{x}{v'}{e_3}{\rho}}
& \Check{\rho_1}{\rho_2}{e_2}{w}
& \Eval{\rho_1}{e_2}{v'}
& \Eval{(x,v')\rho}{e_3}{v''}}

\end{array}
\]

\[\begin{array}{c}
	  \infer{\Infer{\rho_1}{\rho_2}{e_1 \; e_2 \vec{e}}{v}
	  } 
          {\Infer{\rho_1}{\rho_2}{(e_1 \; e_2) \vec{e}}{v}}
\end{array}
\]



\caption{Infering type of expressions}
\end{figure}

\begin{figure}
\caption{Untypes value equality checking}

\[\begin{array}{c}
	  \infer{\EqVal{\Set}{\Set}}
          {}
\end{array}\]

\[\begin{array}{c}
	  \infer{\EqVal{k_n}{k_n}}
          {}
\end{array}\]

\[\begin{array}{c}
	  \infer{\EqVal{\VPi{x}{v_1}{e_1}{\rho_1}}{\VPi{y}{v_2}{e_2}{\rho_2}}}
          {   \EqVal{v_1}{v} 
& \Eval{(x,k)\rho_1}{e_2}{v'_1}
& \Eval{(x,k)\rho_2}{e_2}{v'_2}
& \EqVal{v'_1}{v'_2}}
\end{array}\]

\[\begin{array}{c}
	  \infer{\EqVal{\VPi{x}{v_1}{e_1}{\rho_1}}{\VPi{y}{v_2}{e_2}{\rho_2}}}
          {
\Eval{(x,k)\rho_1}{e_2}{v'_1}
& \Eval{(x,k)\rho_2}{e_2}{v'_2}
& \EqVal{v'_1}{v'_2}}
\end{array}\]

\[\begin{array}{c}
	  \infer{\EqVal{\cc}{\cc}}
          {}
\end{array}\]

\[\begin{array}{c}
	  \infer{\EqVal{\ff}{\ff}}
          {}
\end{array}\]

\[\begin{array}{c}
	  \infer{\EqVal{\DD}{\DD}}
          {}
\end{array}\]


\[\begin{array}{c}
	  \infer{\EqVal{v_1 \; \vec{v_1}}{v_2 \; \vec{v_2}}}
          {\EqVal{v_1}{v_2} & \EqVal{\vec{v_1}}{\vec{v_2}}}
\end{array}\]


\end{figure}

\subsection{Checking declarations}
\subsubsection{Datatype declarations}
\subsubsection{Function declarations}

\subsubsection{Constant declaration}

\section{Pattern matching}
Dependend pattern matching was introduced by Thierry Coquand in \cite{coquand92pattern}.
The patterns for all the clauses of a function are assumed to be complete. This means that during runtime
it is guaranteed that matching always succeeds.


\subsection{Absurd Pattern}
There arises the need to specify that a certain case cannot happen at runtime.
So you can use the absurd pattenr $ () $ to indicate that the type of the argument expected here is the empty type.
This needs to be checked during type checking. Emptiness checking of types is unddeciable.
Consider defining the $ < $ relation on natural numbers.

\begin{verbatim}
data Lt : Nat -> Nat -> Set
{
  ltzero : (y : Nat) -> Lt zero (succ y) ;
  ltsucc : (x : Nat) -> (y : Nat) -> Lt x y -> Lt (succ x) (succ y)
}

\end{verbatim}

As we know,no natural number is less than zero.
Now, for some $x$, if we have a proof that $x$ is smaller smaller zero, we can proof everything (ex falso quodlibet)
This can be formalized as:

\begin{verbatim}
fun notLt0 : (x : Nat) -> Lt x zero -> (C : Set) -> C
{
notLt0 x  () C 
}
\end{verbatim}
So it has to be checked that the type Lt x zero is really empty.

\subsection{Dot Pattern}
Also you need to be able to tell the type checker that you know exactly the argument value for this pattern during runtime,if the application is type correct.
In the dot pattern $ . e $ the expression $ e $ may contain variables used in the remaming patterns of the clause.

For example, consider the isZero function for sized natural numbers

\begin{verbatim}

fun isZero : (i : Size ) -> Nat i -> Bool
{

isZero i (zero .i) = tt;
isZero (s i) (succ .i x) = ff
}
\end{verbatim}
We just know that the size index that the sized natural numbers carry around equals their size.


\section{Implementation}

MiniAgda was implemented in the function language Haskell.
The alex and happy tools where used for generating lexer and parser.

\subsection{Scope checking}
The first step after parsing is scope checking. This checks that names are used correctly:
all declared constants should have a uniquie name and variables bound by 
The use of linear patterns is also restricted at scope checking.
Thus, all used identifiers can be categorized as refering to a constructor, a variable or a definied data type or
function.
After scope checking, it is guaranteed that looking up in signature or environments is guaranteed to succeed.
\subsection{Type Checking}
Type checking was implemented as explained above.
\subsection{Addmisibility Checking}

\chapter{Termination Checking}

\section{Structural recursion}

\begin{definition}[Order]
\[ O := \{ < , \leq , ? \} \]
\end{definition}

\begin{definition}[Order Addition]
\[
\begin{array}{l | c |c | c }

 +    &  <   &  \leq   &  ? \\
 \hline 
 <    &  <   & <       &  < \\
 \leq &  <   & \leq    &  <  \\
 ?    &  <   & \leq    &  ?
\end{array}
\]

\end{definition}


\begin{definition}[Order multiplication]
\[
\begin{array}{l | c |c | c}
 *   &  <  &  \leq & ? \\
\hline
  <   & <    & <    &   ? \\
 \leq & <    & \leq &   ?  \\
 ?    & ?    & ?    &   ?
\end{array}
\]
\end{definition}

\begin{definition}{Relating patterns and arguments}
\end{definition}

The following definitons are taken from \cite{abelAltenkirch:predStRec}

\begin{definition}
CallMatrix
A call matrix collects the comparisons of arguments and patterns of single call from function $f$ to function $g$
\end{definition}

\begin{definition}[Call Matrix multiplication]

\[(f,M_1,g) \times (g,M_2,h) := (f,M_1 \times M_2,h)\]
\end{definition}

\begin{definition}[Call Graph]
A finite set of Call matrices is called a \emph{call graph}
\end{definition}

\begin{definition}[Call Graph completion]
Call graph completion is closing the set under multiplication.
Completion of a finite set is also finite. 
\end{definition}

\section{Size change principle}

The termination criterion in \cite{abelAltenkirch:predStRec} can be replaced with the
simpler and more powerful criterion from \cite{lee01sizechang}
Their formalism is based on bipartit graphs instead of call matrixes, but these are
equivalent representations.

\begin{definition}
A callmatrix (f,M,f) is \emph{idempotent} if $ M \times M = M $ 
\end{definition}

\begin{definition}[Size change principle]
A function f is terminating, if every idempotent callmatrix  $ \in \mathrm{complete g} $ from f to f has a decreasing element $<$ on
the diagonal.
\end{definition}

\begin{figure}
\[
\mathrm{add}\begin{pmatrix}

\leq & ? \\
?    & < \\

\end{pmatrix}\mathrm{add}
\]
\caption{add call matrix}
\end{figure}

\begin{figure}

\[
\mathrm{add'} \begin{pmatrix}

? & \leq \\
< & ?   \\

\end{pmatrix} \mathrm{add'}
\]

\caption{initial add' call matrix}

\end{figure}


\begin{figure}

\[
\mathrm{add'} \begin{pmatrix}

< & ?    \\
? & <    \\

\end{pmatrix} \mathrm{add'}
\] 
\caption{idempotent add' call matrix}

\end{figure}


\section{Extending the Order}

Definition of pair data type.
Add and ackerman no longer termination check.
Idea: extend order

\begin{definition}[MOrder]
\[ MO := \{ < , \leq , ? , MO^{r,n}\} \]
\end{definition}

\begin{definition}[Order Addition]
\[
\begin{array}{l | c |c | c | c}

 +    &  <   &  \leq   &  ? & M \\
 \hline 
 <    &  <   & <       &  < &  \\
 \leq &  <   & \leq    &  < &  \\
 ?    &  <   & \leq    &  ? &  \\
 M    &      &         &  M &  
\end{array}
\]

\end{definition}


\begin{definition}[Order multiplication]
\[
\begin{array}{l | c |c | c| c }
 *    &  <  &  \leq &   ? & M \\
\hline
  <   & <    & <    &   ? &   \\
 \leq & <    & \leq &   ? &   \\
 ?    & ?    & ?    &   ? &   \\
 M    &      &      &   ? &
\end{array}
\]
\end{definition}

\begin{definition}[Completion with extended order]
is also finite because matrices are only getting smaller or a collapsed to a single value.
\end{definition}

\section{Examples}

\subsection{Huet list rev}
fails the current termination checker
\subsection{natural number division}
fails the current termination checker
\chapter{Adding a Size Type}

\section{Syntax}

\begin{itemize}
\item
size type: $ \Size $ 
\item
size succesor: $\s e $ 
\item
infinity: $\infty$ 
\item
\end{itemize}

\begin{itemize}
\item
size succesor pattern:$ \s p $ 
\end{itemize}

\section{Matching against infinity}
The equation 

\[ s \infty = \infty \]

needs to be taken into account when evaluating a function application.
So whhen the pattern $ s \; x $ is matches against $ \infty $ , $x$ will be bound to $ \infty $  

\section{Interpretation of the Size type}
One may see the size type as a coniductive data type - the dual to natural numbers, the conatural numbers.


\section{Examples}
\subsection{Sized Lists}
\begin{quote}
$\data \List ( A : \Set ) : \Size \ra \Set $ 
\\
$\nil : \EPi{i}{\Size}{ \List A \; (\s i)} $
\\
$\cons : \EPi{i}{\Size}{ A \ra \List A \; i \ra \List A \; (\s i)} $
\end{quote}

\section{Admissibility}
\subsection{Motivation}
The use of sizes needs to be constraint.
Consire for example the function definition

\[ \mathrm{fun} \mathrm{bad} : Size -> Size \]
\[ \mathrm{bad} (s i) = i \]

The termination checker will hapilly tell you that bad is terminating.
The issue here is that the size argument is not really bound to any sized data type.
But there are more subtile difficulties with using the size type.
See the example in the appendix, due to A. Abel,
the type of the function shift

\[ (i : Size) \rightarrow (Nat \rightarrow Maybe (SNat (s i))) \rightarrow Nat \rightarrow Maybe (SNat i) \] 

needs to be rejected by an admissibility checker.


\section{Subtyping for size}
It would be quite nice to have subtyping.
There is natural order on size expressions.


\chapter{Adding coinductive types}

\begin{itemize}
\item
coinductive datatype:
\[\codata \; \DD \; \tau : t \; \vec{\gamma}\]  

\item
coinductive function:
\[\cofun \; \ff \; : \; t \; \vec{\kappa}\]
\end{itemize}

\section{Syntax}

\section{Arity}

\section{Admissibility}

\section{Examples}


