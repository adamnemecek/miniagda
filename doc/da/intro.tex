\chapter{Introduction}

\paragraph*{Dependent type theory.}

Dependent types are an important part of Martin-L\"of type theory \cite{mart84}, which 
was conceived as a foundation for constructive mathematics.

Through the proofs-as-programs paradigm, they are fundamental to proof assistants 
like Coq \cite{coq}, Lego \cite{pollack94theory} and Twelf \cite{pfenning99system}.
As programming languages receive more and more powerful type systems, there have been renewed efforts to use dependent types in programming languages: Cayenne \cite{augustsson98cayenne}, Agda \cite{norell:thesis} and Epigram \cite{epigram} strive to be seen more as practical programming languages rather than logical frameworks.

\paragraph*{Inductive families.}
Dependent type theory\footnote{for an introductory text, the reader is referred to \cite{NPS:promlt}}
 is often described as an open theory: new constants can be added to the system.
As a programming language, there is the need to define inductive types like lists and trees.
For logical purposes, inductive predicates have to be defined. 

Inductive families \cite{dybjer94inductive} offer a powerful general scheme for defining inductive types.  
A nice example for programming is matrix multiplication.
One could define a inductive family of types \verb+Mat+ that is \emph{indexed} by two natural numbers.
The type for matrix multiplication could then be given as 
\begin{quote}
\begin{verbatim}
mmult : Mat k m -> Mat m n -> Mat k n 
\end{verbatim}
\end{quote}
which is not possible with a simple type system.
\paragraph*{Pattern Matching.}
In traditional type theory, so-called elimination constants for each data type provide primitive recursion on objects. 
As an alternative, it was suggested in \cite{coquand92pattern} to allow definitions by pattern matching, a well-known concept from functional languages \cite{DBLP:conf/fpca/Augustsson85}.

Pattern matching does allow for more readable and intuitive definitions, which is essential for type theory \footnote{Pattern matching is actually not a conservative extension to a theory with elimination constants. 
Pattern matching definitions can only be translated to a traditional type theory by adding the so-called \emph{K Axiom} \cite{hofmann95groupoid,GoguHMcBrCM2006}. 
} to be useful as a programming languages. 
It has to be ensured that the pattern clauses cover all possible cases.
\paragraph*{Termination}
Furthermore, pattern matching allows unrestricted recursion, and functions with non-terminating computations have to be disallowed for the following reasons: 
\begin{itemize}
\item
Proofs as programs need to be total functions to be correct.
\item
As types can depend on terms, computation is performed during type checking.
Thus, for type checking to be decidable, non-terminating computation cannot be allowed. 
\end{itemize}
To be more concrete, let us look at a simple recursive definition by pattern matching on a list in the functional programming Haskell: 
\begin{verbatim}
length :: [a] -> Int
length [] = 0
length (x:xs) = 1 + length xs 
\end{verbatim}
The only recursive call is \texttt{length xs}.
The argument \texttt{xs} is a structural part of the input argument on the left hand side \texttt{(x:xs)}.
We conclude, if the list data type is well founded (i.e there are only finite lists), then the program \texttt{length} is guaranteed to terminate on all input lists.

\paragraph*{Term-based termination.}
We note that we looked only at the defining clauses of the function to decide termination, its type declaration was irrelevant. Such methods are called \emph{term-based} to differentiate them from the \emph{type-based} methods discussed below.

When pattern matching for dependent types was introduced in  \cite{coquand92pattern},
the test that was outlined above was suggested as the criterion for termination:
at least one argument needs to get \emph{structurally smaller} in each recursive call.
But not all defined functions have that clear \emph{structurally recursive} form.

In \cite{abelAltenkirch:predStRec}, a decision procedure for a simply typed language was given that also handles lexicographic orderings on arguments and mutual definitions.
The \emph{size change principle} \cite{lee01sizechange} subsumes this effort and also handles functions with so-called permuted arguments. 

\paragraph*{Type-based termination.}
The above methods require that a declaration does not only need to be accepted by the \emph{type-checker}, but also pass a separate syntactic \emph{termination check}.
In contrast, \emph{type-based termination}\footnote{for an introduction, see \cite{abel:PhD}} refers to methods where the type-checker itself is in charge of checking termination: It is ensured by the typing rules. 

The \emph{sized types} approach is an instance of type-based termination:
Inductive types are decorated with size annotations that denote an upper bound on the height of the objects inhabiting them. 
It is then checked that this size is getting smaller during recursive calls.

In comparison with term-based termination, sized types are more robust to syntactical changes to the program and the type system is able to use the information that certain functions are \emph{size-preserving}.
Some effort \cite{bgp:lpar06} has been made to automatically infer these size annotations.

\paragraph{Infinite objects.}
Often there arises the need for potentially infinite objects, for example to model a continuous stream of network packages.
Streams, infinite sequences of elements, are the prime example of \emph{coinductive} types, where inhabitants can have 
\emph{infinite height}. 

While \emph{corecursive definitions} that create or manipulate infinite objects are inherently non-terminating, they still should be \emph{productive}. For productive definitions, looking at a finite portion of the object is well-defined.
As for termination, there are syntactic tests to guarantee productivity \cite{coquand-infinite}, but here again sized types offer a worthwhile alternative.  

\paragraph*{Mugda.}
Our work started with an investigation of how a sized type approach could be added to a system like Agda.
The current version of Agda supports inductive families and mutual recursive function declarations by pattern matching. 
The Agda termination checker was, at that time, based on the work of \cite{abelAltenkirch:predStRec}.

In the description above, sized types sound like a very different approach compared to term-based termination.
But, with inductive families and pattern matching, these approaches can be naturally combined.

As a result, the system \mugda was developed.
It supports inductive families and pattern matching. 
As special features, the language supports coinductive definitions and also offers a built-in \emph{Size type} to form size annotations. 
We outline here how termination and productivity checking of definitions is handled by \mugda:
\begin{itemize}
\item
a syntactic termination checker based on the size-change principle is employed.
\item
in addition, the Size type can be used to create sized data types, enabling advantages of the 
sized type approach.
\item
productivity of corecursive functions is also guaranteed by the Size type.
\item
the usage of the Size type is controlled by the type checker.
\end{itemize}  
\section{Overview}
We introduce the syntax and semantics of the \mugda language in the next chapter, and show some example programs.
The language described is missing the Size type that is later added in chapter 5. 

Type-Checking for $\mugda$ is presented in chapter 3.
We cannot just focus on termination-checking because a termination-checker can easily be tricked by ill-typed declarations.

In the fourth chapter, a termination criterion for inductive function declarations based on the size-change principle is presented. Then we motivate and present an extension that increases its strength for definitions with nested patterns.

In chapter 5, the Size type is added to the language, which enables the formation of sized data types. It is shown how sized data types help to allow more recursive definitions pass the termination check of the third chapter.
It also enables the declaration of a broad range of productive corecursive definitions to pass this check.
Finally, we first motivate why the usage of size annotations needs to be restricted, and then give an \emph{admissibility criterion} to achieve this.

In the conclusion, we take a look back and list ideas for future work. The appendix contains a description of the implementation of $\mugda$.






