\chapter{Preliminaries}

\section{Dependent Types}
While minimal logic corresponds nicely with the simple typed lambda calculus,
a more complex type system needs to be used to handle quantification used in predicate logic.

\section{Dependent pattern matching}
Traditionionally,elimination rules were used for introducing new constants in type theory.
Coquand suggested defining by pattern matching.
This adds to the usability of the language, and it is not a conservative extension to traditional type theory
with eleminators,as the K Axiom has to be added to the theory (\cite{hofmann95groupoid,GoguHMcBrCM2006}

\section{Inductive Families}

Using depentend types, you can a type Vec that is a list that carries its size with it.
A function of type \(\Pi n : Nat . Vec n \rightarrow Vec n \) is thus known to be size preserving. 
\section{Coinductive Types}

The prime example for an coinductive type are streams, the dual to list.
While it's always possible to apply a constructor to a list and get a bigger list, it's always
possible to deconstruct a stream, getting an element of the stream.
So streams are essentially infinite lists.  

\section{Sized Types}

Many termination checking techniques rely on \emph{terms} getting structurally smaller on recursive calls.
The idea of sized based termination \cite{abel:PhD} is to move this to types:
A recursive function is terminating if the \emph{type} of the recursive argument is getting smaller.
Type based termination treats data types as a family of types indexed by it size.
Special size expressions are introduced that decorate types.
Type based termination plays to its strength with higher order functions. 
Type based termination arose in the context of non-depentend lambda calulus.
As dependent types smear the distinction between types and terms, so they already offer much of the power needed for type based termination.